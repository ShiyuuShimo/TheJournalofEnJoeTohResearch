\documentclass[10pt, a5paper, twoside]{jsarticle}

\usepackage{okumacro}
\usepackage{enumitem}
\usepackage{amssymb}
\usepackage{amsmath}{}
\usepackage{amsfonts}
\usepackage{amsthm}
\usepackage{bm}
\usepackage{url}
\usepackage{here}
\usepackage[dvipdfmx]{graphicx}
\usepackage{wrapfig}
\usepackage{makeidx}
\usepackage{braket}
\usepackage{ascmac}
\usepackage{fancyhdr}
\usepackage{tikz}
\usepackage[top=20truemm,bottom=20truemm,left=15truemm,right=15truemm]{geometry}

\pagestyle{fancy}
	\fancyhead{}
	\fancyhead[RE]{円城塔研究}
	\fancyhead[LO]{資料:コンメンタール「ムーンシャイン」}
	\fancyhead[LE, RO]{\thepage}
	\fancyfoot{}
	\fancyfoot[LE, RO]{\footnotesize{The Journal of EnJoeToh Research, Vol.2, No.1, 2024} }

\theoremstyle{definition}
	\newtheorem{axi}{公理}
	\newtheorem{dfn}{定義}
	\newtheorem{thm}{定理}
	\newtheorem{hyp}{予想}
	\newtheorem{ths}{提唱}
	\newtheorem{prn}{原理}

\setcounter{page}{21}

\begin{document}

	~ %強制改行

	\begin{center}

		\Large{コンメンタール「ムーンシャイン」}

		\vspace{3mm}

		\large{Commentary of short story \textit{Moonshine}}

		\vspace{3mm}
		
		\large{下村思游}

	\end{center}

	\vspace{3mm}

	\section{はじめに}

		本論は,円城塔の短編小説「ムーンシャイン」の註解である.本作を構成する要素は,どれもひどく複雑に絡み合っており,それを逐一解説しようとする本論もまたひどく混み入らざるを得ない.本論の記述がしばしば混線するのは,そもそも本作の記述が混線していることに由来し,また本作が扱う諸概念同士が,様々な歴史的経緯や数学理論.あるいは自然そのものの構造によって複雑かつ相互に関係し合っていることにも由来する.この複雑さは明らかに円城塔の意図したものである:一見関係なさそうに見えるもの同士が関係し,しかも異常に複雑であるムーンシャイン理論,異常な直感力で手続きなしに数学的事実を把握したラマヌジャン,数が計算機として振る舞うFRACTRAN,複雑なものが計算機となるウルフラムの提唱,そして計算機の能力に関連するコルモゴロフ複雑性が,それぞれ装いを変えながら入れ替わり立ち替わり登場する構成となっている本作を,偶然の産物であるとは到底言えない.

		先に挙げた以外にも,本論を通じて,本作そして円城塔作品全体を貫いて,コンウェイ,ハーディ,タオ,サール,ラッセル,ヒルベルト,ゲーデル,チャーチ,チューリング,オイラー,フォン・ノイマン,カントといった人々が陰に陽に影響力を行使する様を見ることになるだろう.

		また,本作を語る上で避けられないのは,本作の根底を担う不可欠な概念であり,それゆえに円城塔本人に「現代的観点からすると不適切な部分がある」と断言される原因となった,自閉スペクトラム症である.この註解は妥当か,円城塔による評は妥当か,本論が複雑怪奇なのは必然か,読者による批判を待ちたい.

	\section{解説}

		頁数・行数は『ムーンシャイン』紙版\cite{moonshine}に準拠した.

		\subsection{題名}

			冒頭で示されている通り,密造酒あるいは戯言・法螺話・馬鹿話といった意味の英単語,moonshineに由来する.

			冒頭で示されている3番目の意味は英単語としてのmoonshineには存在しない語義\footnote{例えば,ウェブスター\cite{webster}では「月光/馬鹿話/密造酒」とのみある.}であり,辞書的に正しい語義を示しているのではなく,本作における“ムーンシャイン”という言葉の位置付けを示したものと考えられる.“ムーンシャイン”という語について,円城塔は,本書の「あとがき」において,「単語を借用しているだけ」と説明する.後述の通り,これは素直に受け取ってよい.本作において,ムーンシャイン理論は本義を離れ,十分難しいために便利に扱ってよいSFガジェットとしてのみ振る舞っている.

		\subsection{p.71, l.1-2}

			\begin{quote}

				「千九百十一という数について長々と語りはじめることは僕などにはなかなか難問であり,それがお前の才能の限界であると言われればその通り.」

			\end{quote}

			難問であるという割に,この直後のタクシー数の紹介以降,“僕”は1911という数について饒舌に語る.これは,本作の語り手の一人である“僕”が数学的な能力に優れているということの示唆だろう.数について具体的に知っていることは,現実においても数学の能力に寄与することがある.ガチャガチャと計算した結果出てきた数列をOEIS\footnote{Online Encyclopedia of Integer Sequences\cite{oeis}.数に関する記述を集積したサイトで,適当な数列で検索すると,その数列の性質の一覧や,その数列を部分にもつ数列とその性質の一覧を知ることが出来る.}で検索して関連しそうな既知の理論や定理を知ることを数学研究の手がかりとすることがある\cite{ysn}.数学も所詮は人間による知の営みであり.プロの数学者だからといって,数学の扱う対象すべてについての情報を網羅しているわけではない.数学は,他者や機械との交流や支援を通じて足りないものを補い合いながら進展していく有機的な体系である.

			なお,本作のもう一人の語り手である“少女”は,十分強大な直感的数学力を有し,1911よりも遥かに大きなモンスター群の位数について具体的に知っていることが後々示唆される.モンスター群の位数が1911より遥かに大きいことから,“少女”の数学的能力もまた“僕”より遥かに強力であることが示唆される.

		\subsection{p.71, l.2-5}

			\begin{quote}
				
				「これがたとえば,千七百二十九とかであれば話は別で,〜」

			\end{quote}

			“インドの魔術師”の異名で知られるインド出身の数学者,シュリニヴァーサ・ラマヌジャン(Srinivasa Ramanujan)の有名なエピソードに由来する.

			ラマヌジャンの師である英国の数学者ハーディによれば,ハーディが病床のラマヌジャンを見舞ったとき,「1729というつまらないナンバープレートのタクシーに乗ってきた」と述べたが,ラマヌジャンは「それは非常に面白い数です;それは二つの立方数の和として異なる二通りに表し得る最も小さい数です」と指摘したという\cite{ram}\footnote{入院中の大病人(ラマヌジャンはこのまま体調が回復せず32歳で病死する)にこんな会話をするだろうか,という疑念に対しては,しない方が不自然であるとすら反論する.数学者や物理学者というものは,そういう生き物である.}.このエピソードに因み,このような性質の数のことをタクシー数という.

			\begin{dfn}

				タクシー数

				タクシー数$Ta(n)$とは,2つの異なる正の立方数の対の和で$n$通り書けるような最小の自然数である.
				
			\end{dfn}

			本文にある通り,1729は$1729 = 12^3 + 1^3 = 10^3 + 9^3$と確かにタクシー数$Ta(2)$である.

			また,ラマヌジャンのエピソードを冒頭で示したのは,本作に登場する異常な直感的数学能力を有する“少女”の印象を強調する意図があると考えられる.ラマヌジャンは正規の数学教育をほとんど受けたことがなく,証明や定理といった数学の基本的な概念や操作を用いなかったにも関わらず,極めて複雑な数学的主張を直感によって得ることが出来るという特異な能力を持っていた.

		\subsection{p.72, l.1-2}
		
			\begin{quote}
				
				「ファイト--トンプソンの定理の元になったバーンサイド予想〜」

			\end{quote}

			ファイト--トンプソンの定理およびバーンサイド予想のどちらも,実在する数学の定理と予想である.

			\begin{thm}
				
				ファイト--トンプソンの定理(奇数位数定理)

				すべての奇数位数の有限群は可解である.

			\end{thm}

			\begin{hyp}
				
				バーンサイド予想

				非可換な有限単純群の位数は偶数である.

			\end{hyp}

			いずれもその主張は極めて明解だが,その証明と検証が極めて困難であったことで知られている.バーンサイド予想はファイトとトンプソンによって拡張され,ファイト--トンプソンの定理として200頁以上にも及ぶ長大な証明\cite{ft}が与えられた.この証明に対しては,発表直後から論理の飛躍や誤謬と思われる箇所の指摘がなされていたのだが,その複雑さと難解さによって,人間の数学者による検討はもはや不可能であるとされた.結局,2012年になってようやく定理証明支援系の一つ,Coq\footnote{フランス国立情報学自動制御研究所を中心に開発・管理されている定理証明支援系.プログラミング言語OCamlで書かれたソフトウェアであり,電子計算機上で動作する.ファイト--トンプソンの定理のほか,四色定理の証明に使われたことでも有名.}を用いた機械支援によって,千ページを超える長大かつ精密な証明\footnote{実際のCoqによる証明の過程は\cite{oo}を参照されたい.}が発表され,ファイトとトンプソンによる仕事は50年越しに完成された.

		\subsection{p.72, l.6-7}

			\begin{quote}
				
				「大多数の人々にとってそんなことは暮らしに活かしようのないこと〜」

			\end{quote}

			事実ではあるのだが,本作を精読するにあたっては,大多数でない人について考えなければならない.

			本作に登場するモンスター群は,これを対称性として持つような量子重力理論を構築しようという試みがあり,標準模型を超える新物理の候補の一つとして有力視されている.確かにモンスター群は日常生活にまったく馴染みがないかもしれないが,もしかすると,この宇宙を記述する物理法則の一部として私たちと不可分の数学的対象かもしれない.

			他にも,超弦理論の一部に関連があるのではないかと考えられているマシュームーンシャインというものもあり,これは長らく未完成のままとなっている超弦理論への数学側からのアプローチとなっている.

		\subsection{p.72, l.10}

			\begin{quote}

				「現在の僕の手の中には,何故か千九百十一がある.」
				
			\end{quote}

			勿論,コルト・ガバメント1911のことだが,これに留まらず,後々“少女”が数を実体として扱えている描写に対応するものと考えるべきだろう.

		\subsection{p.72, l.15}

			\begin{quote}

				「マシュー群」
				
			\end{quote}

			実在する数学用語.有限散在型単純群のうち,マシューによって最初に発見された,より簡単な5つの群のこと.モンスター群も有限散在型単純群なのだが,あまりにも位数が大きく複雑であったため,マシューによる探索から逃れていた.

			マシューから始まった有限散在型単純群の探索は,ノートンによってその全数が発見されたことが証明され,終結した.コンウェイ,ノートンらは,それまでの探索で得られた有限散在型単純群に関するすべての知識を“地図帳”\cite{atlas}にまとめて出版した.

		\subsection{p.73, l.1}

			\begin{quote}
				
				「群の定義」

			\end{quote}

			作中の記述がそのまま正しいが,数学的な記法は十分に示されていない\footnote{これは意図的なものだろう.そもそも,本作は,一見主題のように見えるムーンシャインも語の雰囲気だけを間借りしているだけであり,本来数理的概念の本質を深く理解した上で相当程度数理的に正しい法螺を吹く円城塔作品において極めて特異な作品である.群の定義をあえて書かないことで.数学的に正しく理解する必要がないことを示唆しているものと推察される.数学を小説に導入する必要性という観点においても,本作の数学的な要素であるムーンシャインが実は雰囲気だけであるため,説明のための数学を導入する必要がない.}.群の定義は以下の通り\cite{yke}.

			\begin{dfn}

				群

				$G$を空集合ではない集合とする.$G$上の演算$\cdot$が定義されており,次の性質をすべて満たすならば,$G$は群である.

				\begin{enumerate}
					
					\item 単位元と呼ばれる元$e \in G$があり,すべての$a \in G$に対して$a \cdot e = e \cdot a = a$

					\item すべての$a \in G$に対して$a^{-1} \in G$が存在し,$a \cdot a^{-1} = a^{-1} \cdot a = e$

					\item すべての$a, b, c \in G$に対して$(a \cdot b) \cdot c = a \cdot (b \cdot c)$が成り立つ.

				\end{enumerate}
				
			\end{dfn}

			物理学において,群は系の対称性を記述する.特に,素粒子物理学では,ネーターの定理を指導原理とし,系の連続的な対称性に注目して理論が構築されてきた経緯がある.

			\begin{thm}

				ネーターの定理

				ある座標変換に対して,作用積分が対称性を持つならば,その対称性に対応する保存量が存在する.
				
			\end{thm}

			系の空間並進対称性(並行移動)が運動量保存則,空間回転対称性が角運動量保存則,時間並進対称性がエネルギー保存則に対応するように,対称性と保存量の対応関係は物理学において極めて重要な手掛かりとなっている.

		\subsection{p.73, l.16}

			\begin{quote}

				「おお,バーリン.死んでしまうとは情けない.」
				
			\end{quote}

			“おお,〇〇.死んでしまうとは情けない.”という構文はビデオゲーム『ドラゴンクエスト』シリーズのお約束.プレイヤーが戦闘で全滅した際.拠点に戻されたときにかけられる台詞が元ネタ.

			バーリンというのはJ・R・R・トールキン『ホビットの冒険』に登場するドワーフ,バーリンに由来するか.

		\subsection{p.73, l.17}

			\begin{quote}

				「ああ,窓に,窓に.」
				
			\end{quote}

			H・P・ラヴクラフト「ダゴン」\footnote{「いや,そんな! あの手は何だ! 窓に! 窓に!」\cite{dgn}p.18}が元ネタ.

		\subsection{p.74, l.11}

			\begin{quote}

				「歩いているうちに井戸に落ち込む間抜け野郎」
				
			\end{quote}

			古代ギリシアの哲学者,タレスのエピソードが元ネタ\cite{dl}.

		\subsection{p.74, l.12}

			\begin{quote}

				「空から降って来た亀に頭を割られる不幸な人物」
				
			\end{quote}

			古代ギリシアの悲劇詩人,アイスキュロスの死因と伝えられるエピソードが元ネタ.プリニウス『博物誌』などによって現代まで伝わっている\cite{gonoji,naka}.

		\subsection{p.74, l.13-14}

			\begin{quote}

				「テュルプ博士の解剖学講義」
				
			\end{quote}

			オランダの画家,レンブラントの絵画.

		\subsection{p.74, l.19}

			\begin{quote}

				「僕はどちらかといえば〜」
				
			\end{quote}

			“僕”がいわゆる一般的な読者に近いことを示唆している.

		\subsection{p.76, l.8}

			\begin{quote}

				「証明ギャップ」
			
			\end{quote}

			実在する数学用語.証明における自明でない論理の飛躍のことをギャップともいう.これを回避する方法が,先述の定理証明支援系の活用である.現代数学では,機械支援を得て,証明を一段ずつ分解し,間隙を埋めていく,という人間と機械が融合した知的活動が珍しくない.

		\subsection{p.77, l.17}

			\begin{quote}

				「あと魔法陣さえ描けば〜」
				
			\end{quote}

			魔方陣\footnote{$n\times n$枡の枡目の中に1つずつ数字が書かれているものであって,すべての縦・横・斜め$n$枡の和が同じ値となるものをいう.}ではなく,魔法陣.作中後半で,なぜ魔法陣が必要だったのかが語られることになる.

			余談だが,$n$次の魔方陣のパターン数の正確な値は,つい最近まで5次までしか知られていなかったが,2024年5月にようやく6次の魔方陣のパターン数が得られたという報告\cite{mino}があった.なお,モンテカルロ法を用いた確率論的な推計が30次まで行われている\cite{ktj}\footnote{魔方陣は元々数学者の扱う対象であったが,現代では物理学者による計算機を用いた計算が主流となっている,ここで紹介した美濃英俊,北島顕正,菊池誠はいずれも物理学者である.}.

		\subsection{p.78, l.10}

			\begin{quote}

				「無慈悲な夜の女王」
				
			\end{quote}

			ロバート・A・ハインライン『月は無慈悲な夜の女王』が元ネタ.特に意味はなく,適当な引用だと思われる.

		\subsection{p.78, l.16}

			\begin{quote}

				「なあ,やっぱりモジュラス側からムーンシャイン経由で」
				
			\end{quote}

			特に意味はなく,ムーンシャイン理論から耳障りのいい単語を適当に散りばめたものと思われる.

		\subsection{p.79, l.2-4}

			\begin{quote}

				「正確には,〜」
				
			\end{quote}

			モンスター群の位数に一致する.この記述はすべて数学的に正しい.

			モンスター群の位数は,$2^{46} \cdot 3^{20} \cdot 5^9 \cdot {11}^2 \cdot {13}^3 \cdot 17 \cdot 19 \cdot 23 \cdot 29 \cdot 31 \cdot 41 \cdot 47 \cdot 59 \cdot 71 = 808017424794512875886459904961710757005754368000000000$

		\subsection{p.79, l.5-6}

			\begin{quote}

				「もう,地球を構成する全ての原子の数よりも多い塔.」
				
			\end{quote}

			作中の記述は正しい.簡単な見積もりによって,以下のように確かめられる.

			地球の全質量は$5.97217(13) \times {10}^{27} $g\cite{pdg},アボガドロ数が$6.02214076 \times {10}^{23}$である\cite{si}から,簡単のため地球がすべて水素から構成されていると仮定する\footnote{実際には,地球の組成は質量比で鉄,酸素,ケイ素,マグネシウム,ニッケルの順である\cite{ichi}が,すべて水素から成ると仮定した場合に地球を構成する原子の数が最大となる(地球を実際に構成する原子の数は計算結果より少ない)ので,結論に影響はない.}と,地球を構成する水素原子の数は,$(6.0 \times {10}^{27}) \times (6.0 \times {10}^{23}) = 3.6 \times {10}^{51}$より,約$3.6 \times {10}^{51}$個である.

			これに対して,モンスター群の位数は約$8.1 \times {10}^{53}$であり,地球を構成するすべての原子の数より,モンスター群の位数の方が大きい.

		\subsection{p.79, l.6}

			\begin{quote}

				「それでも私は,塔の数を正確に把握している.」
				
			\end{quote}

			語り手である“少女”の数学的能力が異常に強力であること,またその数学的能力は直感的な把握能力であることを明示している.

			これは真に驚くべきことである.数であればいかに巨大であっても直感的に扱えるとするならば,少女はゴールドバッハ予想を直ちに解決することが可能である.ここで,ゴールドバッハ予想は以下の通り.

			\begin{hyp}

				ゴールドバッハ予想

				$n$が4以上の偶数ならば,$n$は2つの素数の和として表される.
				
			\end{hyp}

			一見単純で簡単そうだが,ゴールドバッハ予想はゴールドバッハがオイラーとの書簡で定式化して以来,280年以上も未解決のまま残されている史上稀に見る難問である.

			ここにビジービーバー(busy beaver)を導入する\footnote{河出文庫『シャッフル航法』に収録された短編「Beaver Weaver」は,ビジービーバーと関連するのではないかと考えられている.}.ビジービーバーは$n$状態2記号チューリングマシンであり,ビジービーバーが空のテープから始動して停止し,しかもテープに出力する1の総数が最も多いものである.$n$状態ビジービーバーが出力する1の総数をビジービーバー関数といい,$BB(n)$と表す\cite{bbf}.このビジービーバー関数は$n$が増大するにつれて急速に爆発する.実際,人類は2024年になってようやく$BB(5)$の正確な値を知ることが出来た\footnote{この証明にはCoqが用いられた.実際の証明は\cite{bb5}を参照されたい.}.

			ビジービーバーを用いれば,反例さえあげればいいような問題,例えばゴールドバッハ予想のような問題について,ビジービーバーでその反例を探索するようなプログラムを用意すれば,$BB(n)$回だけ動作するまでに反例が見つかり停止するか,反例が見つからず停止しないかという形で解決される.ただし,そのためには$n$が20〜50は必要ではないかと(素朴に)考えられている一方で,$BB(20)$はZFC公理系の健全性を仮定すると値が決定不能であるとの(数学的な)予想\cite{bb20}がある.要するに,力技での解決は(現実では)不可能である.しかし,“少女”は明らかに可算無限以上の大きさの数を扱えることが作中で明記されていることから.この力技の解決が可能である.

			余談だが,ビジービーバー関数の値のような巨大な数を文字通り巨大数といい,アマチュア数学者を中心に盛んに研究がなされている.プロの数学者ではコンウェイのチェーン記法,クヌースの矢印表記などによる貢献がある.さらに余談だが,このクヌースは数理系に特化した組版ソフト \TeX の開発者でもある.本書の組版には,その拡張である \LaTeX 言語を用いており, \TeX 言語および \LaTeX 言語はチューリング完全である\footnote{\LaTeX の文法,特に数式の記法は複雑で扱いにくいことでも知られており,\LaTeX は十分複雑だからチューリング完全なのは当然,というようなギャグも流通している.}ことが知られている.

		\subsection{p.79, l.7-9}

			\begin{quote}

				「聞こえるように.〜」
				
			\end{quote}

			“少女”が共感覚によって数を把握している,という描写に加え,共感覚によって発生した高階の感覚が数を感得できる“数覚”となっていることを示す.

			最後の「青い」という文言は,宮沢賢治『春と修羅』の序文\footnote{「わたくしといふ現象は/仮定された有機交流電燈の/ひとつの青い照明です」\cite{myzw}}が元ネタ.“青い照明”は他にも「Boy's Surface」\footnote{ハヤカワ文庫JA『Boy's Surface』収録.“青い照明”ならぬ“青い証明”が登場する.自動定理証明・定理証明支援系にも注意したい.}や「ドルトンの印象法則」にも登場する.

		\subsection{p.79, l.19}

			\begin{quote}

				「剣によって薙ぎ払われた草原は〜」
				
			\end{quote}

			記紀神話における,倭建命と草薙劔のエピソード\footnote{「ここにまづその御刀もちて草を苅り發ひ,〜」\cite{fkf}p.138}が元ネタ.

		\subsection{p.80, l.1}

			\begin{quote}

				「十七と十九,双子の兄弟と〜」
				
			\end{quote}

			17と19が双子素数であることから.双子素数とは,$p$と$p+2$が同時に素数であるような素数の組$(p, p+2)$のことをいう.素数が無限に存在することの証明は古代ギリシアの頃から知られていたが,双子素数が無限に存在するか否か(双子素数予想)は未解決問題となっている\footnote{双子素数予想の周辺には,興味深いが未解決となっている問題が多く存在する.文献\cite{seisu}には,グリーン--タオの定理をはじめ,双子素数予想についての解説と研究課題が豊富に示されている.また,ハーディは双子素数予想に大きく貢献している.}.

		\subsection{p.80, l.19-p.81, l.1}

			\begin{quote}

				「私が見ていない間も月はある.私が見ていない間も,月は私を見つめている.」
				
			\end{quote}

			アインシュタインが,量子力学の不可解さ\footnote{本質的には,量子力学が不可解なのではない.世界は本質的に量子力学的なのであり,古典力学は(低過ぎも高過ぎもしない)低エネルギー領域という極めて特殊な条件で近似的に成り立つ特殊理論なのである.20世紀以降の物理学は,人間の直感と観測結果に矛盾が生じるなら,観測結果を採用して人間の直感の方に修正を迫る.}について,友人の物理学者・科学史家,アブラハム・パイスに語った言葉\footnote{「彼は突然立ち止まって私にふり向き,月は君が見ているときにしか存在しないと本当に信じているかね,と尋ねた.」\cite{pais}p.3}が元ネタか.観測するごとに変化する観測対象,姿を変える月,というモチーフは「内在天文学」\footnote{河出文庫『シャッフル航法』収録.観測する度にその向きを変えるオリオン座の観測を通して,人類と認知の主導権を争う未知の知性体の存在が示される.}でも現れる.

		\subsection{p.82, l.4}

			\begin{quote}

				「こうして見,触れることができるものに対して.外観だけを描写するのに.」
				
			\end{quote}

			直感的理解ができる“少女”からすれば,当然の反応といったところか.円城塔は,“少女”が直感によって数を把握していることを繰り返し強調している.

		\subsection{p.83, l.9}

			\begin{quote}

				「異国の言葉」
				
			\end{quote}

			この異国の言語こそ,われわれが日常で用いている自然言語である.直後の描写から,“少女”は五感がはっきり分割されておらず,混交していることもわかる.双子は,要するに「“数覚”を捨てて,通常の五感と通常の数学的能力程度しか可能でない認知機能が標準的に持つような一般的な日常言語を身に付けよ」と言っている.さもなくば、われわれのいるこの一般的な世界に戻れなくなるぞ,とも.

			あるいは,双子が“少女”を引き止めようとするのは,自身らを規定する“少女”の“数覚”が失われるのを防ごうとする保身であるとも解釈出来る.“少女”との別れを惜しむふりをしながら,実のところ自我が失われるのが恐ろしいだけなのかもしれない.クライマックス近辺で語られる,“少女”と数との相剋は,この保身説を支持する.

			円城塔は,自然言語と数理的言語,日本語と外国語,自然言語とその機械翻訳など,異なる基底をもつ言語同士がそれぞれ張る“言語空間”の範囲を探るというモチーフの作品を多く創作している.なお,この“異なる言語”として,生成AIが自然言語と画像が高次元ベクトルを中間言語として容易に結びつけられたことは非常に驚くべきことであり,円城塔本人も「(AI関連の技術進展で)一番ショックだった」\footnote{括弧内下村補註.}\cite{zdn}と発言している.

		\subsection{p.83, l.17}

			\begin{quote}

				「本当は僕らのこの言葉の方が,異国の言葉のはずなのに」
				
			\end{quote}

			「この言葉」は直前の台詞を指す.このような,日本語であるのに過度に説明的な言い回しは,本論をはじめ,物理学や数学など,翻訳語の影響を強く受けた学術語によく見られる.物理学や数学は“異国の言葉”とされることが多いが,それは日常語とは異なる厳密な語用であること,また西洋語の翻訳体が根幹にあることの二段階の差異に起因するものと考えられる.

		\subsection{p.84, l.17-18}

			\begin{quote}

				「これ以上多くを望みたくなってしまったら,〜」
				
			\end{quote}

			ラストシーンにおける“少女”の旅立ちを示唆している.

		\subsection{p.85, l.2-3}

			\begin{quote}

				「偽アポロストス・ドキアディス「全異端論駁」」

			\end{quote}

			人名については,現代ギリシャの作家,アポロストス・ドキアディスから.『ペトロス伯父と「ゴールドバッハの予想」』\cite{gr}という数学小説が早川書房から刊行されている.同書は,本作の元ネタとして直々に挙げられているもののひとつ\footnote{「『ペトロス伯父と「ゴールドバッハの予想」』、『ぼくには数字が風景に見える』、Conway's Prime Producing Machine とモンスターを混ぜるとだいたいあんな感じになる」\url{https://x.com/EnJoeToh/status/1287650436938268672}}である.

			書名については,ヒッポリュトスの著作『全異端反駁』\cite{papa}が元ネタか.

		\subsection{p.85, l.8}

			\begin{quote}

				「奇妙な石蹴り遊び」
				
			\end{quote}

			フリオ・コルタサル『石蹴り遊び』が元ネタ.コルタサルの名は「コルタサル・パス」\footnote{ハヤカワ文庫JA『バナナ剝きには最適の日々』収録.長編『エピローグ』の前日譚.題名はフリオ・コルタサル+オクタビオ・パス.}にも引かれている.

		\subsection{p.85, l.9}

			\begin{quote}

				「ペトロス・パパクリストス」
				
			\end{quote}

			先述の『ペトロス伯父と「ゴールドバッハの予想」』の登場人物,ペトロスと,そのモデルとなったギリシャ出身の数学者,クリストス・パパキリアコプロスが元ネタ.

		\subsection{p.86, l.7}

			\begin{quote}

				「ロンディニウムのダニエル」
				
			\end{quote}

			『ぼくには数字が風景に見える』の著者,ダニエル・タメットのこと.タメットはロンドン出身であり,ロンドンの古名がロンディニウムであることから,タメットを古風に表現したものである.

			タメットは自閉スペクトラム症かつサヴァン症候群の当事者であり,数と色の共感覚をもつ.同書はその特異な能力を通して世界を描写する当事者文学である.

		\subsection{p.87, l.9-10}

			\begin{quote}

				「無理数の存在を認めていなかった」
				
			\end{quote}

			古代ギリシアの数学者,ピタゴラスとその弟子たちからなるピタゴラス教団の主張が元ネタ.

			なお,自然数が$n = 0, 1, 2, \cdots \in \mathbb{N}$ならば,自然数は加法についてモノイド\footnote{四則のうち,自然数が除法についてモノイドでないのは容易に分かり,また減法についてもモノイドでないのが分かる.一方で,乗法についてモノイドでないのは分かりにくいが,自然数に0(加法単位元)が含まれていることからモノイドでないことが分かる.加法と乗法を同時に扱うのは極めて難しく,フェルマーの最終定理が300年以上も未解決であったのも,またABC予想が未解決となっているのも,このことが原因である.}である.ここで,モノイドとは以下の通り.

			\begin{dfn}

				モノイド\footnote{モノイドの定義は,群の定義から逆元の存在(可逆律)を除いたものである.}

				$G$を空集合ではない集合とする.$G$上の演算$\cdot$が定義されており,次の性質をすべて満たすならば,$G$はモノイドである.

				\begin{enumerate}
					
					\item 単位元と呼ばれる元$e \in G$があり,すべての$a \in G$に対して$a \cdot e = e \cdot a = a$

					\item すべての$a, b, c \in G$に対して$(a \cdot b) \cdot c = a \cdot (b \cdot c)$が成り立つ.

				\end{enumerate}
				
			\end{dfn}

		\subsection{p.88, l.1}

			\begin{quote}

				「シモニデスは体系的な記憶の術を編み出し,〜」
				
			\end{quote}

			古代ギリシアの詩人,シモニデスのエピソードに由来する記憶術のこと.建築物の配置と記憶したい事項を結びつけ,思い出したい時は記憶の中の街を歩いて建物に結びつけられた事項を引き出す,という形で記憶する技である.これは「良い夜を持っている」\footnote{新潮文庫『これはペンです』収録.完全記憶を持つ父の足跡を辿ることで,父を理解しようとする.}に主題として登場する.

			余談だが,検索技術が高度に発達した現代においても,調査系司書の第一歩として,参考図書目録とその中の主要図書の概要を5000タイトル暗記することが推奨されていたりする.

		\subsection{p.91, l.1}

			\begin{quote}

				「数を数の持つ性質として直感できる人間」
				
			\end{quote}

			直後に説明があるが,サヴァンのことである.

		\subsection{p.92, l.2}

			\begin{quote}

				「サヴァン・コンピューティング」
				
			\end{quote}

			大塚英志『木島日記』に登場するSFガジェット.強大な計算能力を持つサヴァン症候群の脳を並列接続し,生体コンピュータとして利用する,というもの.いかにも悍ましいアイデアであるが,そもそもコンピュータという語は元々人間の計算手を指していたものであり,脳を直接生体部品として用いるという点を除けば,なんらおかしなところはない.


		\subsection{p.92, l.11}

			\begin{quote}

				「現代の暗号理論の基礎部分は,〜」

			\end{quote}

			作中の記述は数学的に正しい.ここで挙げられているのはRSA暗号であり,巨大な数の因数を暗号通信の鍵として用いる手法である.

			なお,RSA暗号が登場した初期に必要とされていた計算量はそこまで大きいものではなく,現代の計算機を用いれば十分現実的な時間で突破可能となっており,現在では初期方式のRSA暗号の利用は禁忌となっている.

		\subsection{p.94, l.9-10}

			\begin{quote}

				「白と黒の斑点が乱舞する図を見せられれば,〜」

			\end{quote}

			イギリスの心理学者,リチャード・L・グレゴリーによる認知心理学の有名な教科書『脳と視覚』\cite{gre}に掲載された,ゲシュタルト心理学における実験のこと.白と黒の斑点で構成された画像の中に実はダルメシアン犬が写っており,一度ダルメシアン犬を認知すると認知する前の感覚に戻れなくなることを示した.

		\subsection{p.94, l.13}

			\begin{quote}

				「5と2」

			\end{quote}

			2で2と書いてある素数\cite{seisu,2of2}および1でPRIMEと書いてある素数\cite{seisu,prime}と,心理学におけるネイヴォン課題(Navon test)から.ネイヴォン課題は自閉スペクトラム症者の認知機構の研究のために用いられた\cite{ide}.

		\subsection{p.96, l.5}

			\begin{quote}

				「非能力者が圧倒的に遅れをとる〜」
				
			\end{quote}

			ケン・リュウ「天球の音楽」\footnote{ハルコン・SF・シリーズ『天球の音楽』収録.数学的直感力を拡張するインプラントが普及した世界において,体質的にインプラントを導入出来なかった数学者志望の主人公の挫折と再生を描く.同人誌収録作品のため現在は入手困難.}の主題.

		\subsection{p.97, l.13}

			\begin{quote}

				「多重共感覚者」
				
			\end{quote}

			ここからしばらく,共感覚と,それを重ねた多重共感覚に関する説明がなされる.作中における多重共感覚については後ほど詳しく説明する.

		\subsection{p.99, l.5}

			\begin{quote}

				「十七」

			\end{quote}

			本作において,17と19は“少女”の強大な計算能力の上を走る計算機である.ただの数が計算機になる,というのは到底信じがたいが,難解プログラミング言語FRACTRAN\footnote{コンウェイによって開発されたプログラミング言語.ソースコードと入力値がすべて自然数と自然数で構成された分数の列から成り,極めて難読かつ単純だがチューリング完全であることが知られている.PythonによるFRACTRANインタープリターの実装例は\cite{pyfra}を参照されたい.}ではただの素数がメモリとして機能しており\cite{fra},場合によってはたった1つの素数が計算機として振る舞うことすらある.これを考慮すれば突飛すぎるとまではいかない法螺と言える.

			ここで,FRACTRANに関連する話題として,数学の有名な未解決問題であるコラッツ予想\footnote{2022年に,コラッツ予想はほとんどすべての正の整数に対してほとんど正しいことを証明したタオの論文が出版されている.このタオは先述の双子素数予想に関するグリーン--タオの定理のタオ.}を紹介したい.

			\begin{hyp}

				コラッツ予想

				任意の正の整数$a$に対して,以下の整数列は,必ず1に到達する.

				\begin{equation*}
					x_0 = a, x_{n+1} = \begin{cases} \displaystyle \frac{x_n}{2} & \text{if~} n \equiv 0 \\ 3 x_n + 1 & \text{if~} n \equiv 1 \end{cases} (\text{mod~}2)
				\end{equation*}
				
			\end{hyp}

			このような数列をコラッツ数列といい,コラッツ数列は,1次関数$g_i(n) = a_i n + b_i$を2つ用いた以下のような2部1次関数で表される.

			$$g(n) = g_1(n) | g_2(n)$$

			$g(n)$は$g_1(n)$,$g_2(n)$の順に評価され,最初に整数となった返値がそのまま$g(n)$に返される.$(a_1, b_1) = (\frac{1}{2}, 0), (a_2, b_2) = (3, 1)$となる場合,これは確かにコラッツ数列を満たす.また,一般の$k$部1次関数(一般化コラッツ問題)は次のように表される.

			$$g(n) = g_1(n) | g_2(n) | g_3(n) | \cdots | g_k(n)$$

			$a_i \in \mathbb{Q}$,$b_i = 0$ならば,$g(n)$はFRACTRANのコードである.すなわち,FRACTRANのコードは一般化コラッツ問題である\footnote{ただし,コラッツ予想は一般化コラッツ問題に含まれるが,FRACTRANのコードではない.}.FRACTRANはチューリング完全であり,そのコードは決定不能\footnote{$\because$チューリングの停止性問題.停止性問題は,ゲーデルの第一不完全性定理と数学的に等価であることが知られている.}なので,一般化コラッツ問題もまた決定不能である.ここに至っても,コラッツ予想自体は何も解決していないが,コラッツ予想を含む一般コラッツ問題が決定不能であることがわかった.このことから,コラッツ予想が決定不能である可能性が存在する.

		\subsection{p.99, l.16}

			\begin{quote}

				「少女の形をしたものの表面に浮き出た何か」

			\end{quote}

			先述の,17と19は“少女”の強大な計算能力の上を走る計算機であるという(作中における)事実を強調している.

		\subsection{p.99, l.17}

			\begin{quote}

				「ユニバーサル・チューリング・マシン」
				
			\end{quote}

			Universal Turing Machine.日本語では万能チューリング機械ともいう.任意のチューリングマシンをシミュレート出来るチューリングマシンのこと.

		\subsection{p.99, l.18-19}

			\begin{quote}

				「まあ,\ruby{能力}{クラス}的には.でもチューリング完全なんてのは,容易く実現できるものでね.ライフ・ゲームだって見方によってはチューリング完全」

			\end{quote}

			クラスというのは,計算複雑性理論におけるクラス,あるいは物理学者スティーヴン・ウルフラムによるオートマトン\footnote{オートマトンの定義をする}の分類のことを指すか.前者について,クラスPは以下のように定義される\cite{sip3}.

			\begin{dfn}

				クラスP

				Pは決定性単一テープチューリングマシンによって多項式時間\footnote{多項式時間とは,時間を計算のステップ数で表したとき,多項式で表される時間のことである.例えば,判定に$n, n^2, n^3$回の計算が必要ならば,多項式時間で判定出来るが,$2^n$回の計算が必要ならば,多項式時間では判定出来ない(指数時間で判定可能).}で判定出来る言語のクラスである.

				$$\mathrm{P} = \bigcup_{k} \mathrm{TIME}(n^k)$$
				
			\end{dfn}

			前者の説をとるならば,本来,チューリングマシンで多項式時間で判定可能$\Rightarrow$クラスPである,という論理になるところ,クラスPの問題を多項式時間で判定可能$\Rightarrow$チューリング完全である,と転倒させているものと考えられる.

			後者について.ウルフラムは,オートマトンは以下の4つの動作パターン\footnote{「この定性的な分類クラスは,複雑系の振舞いにおける(1)アトラクター点,(2)リミット・サイクル,(3)カオティック・アトラクタ,(4)非常に長い軌道を描く(時間と相関のある)アトラクタにそれぞれ対応している.」\cite{bio}}に進化すると考えた\cite{bio}.

			\begin{quote}

				\begin{enumerate}
					\item 単調でつまらない「死滅」状態

					\item 単純で安定な周期的状態

					\item 非常に多くの安定点を持ち,さらに不安定な状態にもなるカオス的なパターン

					\item 複雑な局所的構造で,動的に進化する.ある期間後に突然パターンを変えるような状態.
				\end{enumerate}
				
			\end{quote}

			ウルフラムは,4番目に挙げた,十分複雑で“生きた”計算論的振舞いをするオートマトンは大体チューリング完全である,と主張した.このような主張をウルフラムの提唱といい,90年代から00年代の複雑系研究者の間で一種の常識として扱われていた\footnote{円城塔本人による複雑系の文脈説明にも同様の文言が見える.「「複雑に見えるものにはだいたいUTMが入ってるんでは予想」が行われるようになり、ルール110のチューリング完全性が証明されたりしました。」\url{https://x.com/EnJoeToh/status/1361841991143677956}}.

			\begin{ths}

				ウルフラムの提唱

				十分複雑なものはチューリング完全であり,それは生命である.
				
			\end{ths}

			この後者の説をとるならば,17は,17自身が“少女”という極めて複雑な計算能力\footnote{“少女”は十分複雑な計算機ではないことに注意.“少女”の計算能力は,通常の計算機やその単純外挿のもつようなものではなく,原理的にまったく異なる機序によって得られている直感的能力である.}の上を走る十分複雑な構造であり,チューリング完全である,すなわち自身は生命体であると主張していると解釈出来る.

			なお,チューリング完全とは,万能チューリングマシンと同じ計算能力をもつことである\footnote{この定義は宙に浮いたようなものとなっている.数学的には,チャーチ--チューリングの提唱によって,“計算能力”という曖昧な概念を計算可能関数という数学的な対象に一致させている.}.ウルフラムの提唱の下,複雑そうに見えるもののチューリング完全性の証明が流行し,ライフゲーム\footnote{コンウェイが考案したセルオートマトン.マス目を白黒2色で塗り分け,色の変化に関する簡単なルールを定めるだけで,極めて複雑な様相が現れる.}のルール110や,ウルフラムが最も単純な十分複雑なオートマトンであると考えたウルフラムの2状態3記号チューリングマシンのチューリング完全性が証明された\cite{mth,smth}.

		\subsection{p.100, l.17}

			\begin{quote}

				「平常の人間の認知過程が多重に暴走している小娘.」
				
			\end{quote}

			通常の五感で必要十分,そこに数と色の共感覚を載せるだけで異常な数学的能力を発揮出来るのだから,多重に暴走すれば文字通り人智を超えた超強力な数学的能力を得るのは当然である,という感じか.感覚の拡張によって世界が拡張されるという論理は,飛躍でありながら,カントを踏まえた議論となっている.また,ニューラルネットワークや中国脳も連想される.

		\subsection{p.101, l.5}

			\begin{quote}

				「孤立して虚空に浮かぶ巨大な万華鏡」
				
			\end{quote}

			他から切り離され,自身が自身のみを指示規律し,独立して存在するもの,というイメージは,円城塔作品に頻出するモチーフである.「Self-Reference ENGINE」\footnote{ハヤカワ文庫JA『Self-Reference ENGINE』収録.連作短編集の最後に収められた,書名と題名を同じくする短編.}におけるSelf-Reference ENGINE,「$\varnothing$」\footnote{河出文庫『シャッフル航法』収録.1段落ごとに1字ずつ減少していく文体芸とともに,収縮する作品宇宙をその内部から考察する.}における作中宇宙,「考速」\footnote{ハヤカワ文庫JA『後藤さんのこと』収録.有限から簡単なアルゴリズムによって容易に構成可能な無限,語るよりも速く伸長してしまい未来永劫語りえなくなってしまうもの,について考察する.}におけるスピノザ『エチカ』の“公理二”\footnote{「他のものによって考えられえないものはそれ自身によって考えられなければならない.」\cite{ethica}p.43}などが好例.

			また,“万華鏡”という表現は,モンスター群の司る対称性を鏡像対称性に仮託したもの.直後の“群論的アルファベット”というのも,モンスター群が“既約”であることを文字列の最小単位であるアルファベットで例えたものである.

		\subsection{p.101, l.12-13}

			\begin{quote}

				「群論と,コンピュータの間に直感的な関連なんてないのでは,〜」
				
			\end{quote}

			ごもっともな指摘だが,直後にある通り,現実にそういう機構が現れているので,(作中においては)群論と計算機の間に関連はあると認めざるを得ない.ウルフラムの提唱の通り,モンスター群は十分複雑であるからチューリング完全である,ということである.物理学に顕著な,経験主義の態度がよく現れている.

		\subsection{p.102, l.6}

			\begin{quote}

				「あなたのボスが描いたあの紋様,彼女の裡で渦巻く認知過程を乗り切って,ポストのUTMの認知的概念図を届けたあれ.」
				
			\end{quote}

			“あの紋様”とは,前半でボスが描いた魔法陣\footnote{繰り返しになるが,魔方陣ではない.}のこと.ボスの描いた魔法陣は,どうしてか“少女”の中の17という万能チューリングマシンを起動させ\footnote{無秩序な状態の物質に,外部から刺激を与えることで,秩序構造を取り出せることがある,例えば,過冷却状態の水に衝撃を与えると氷になる.},“少女”は17というインターフェースを介して外界の“異国の言葉”を用いて意思疎通を図ることが出来るようになっている.

			ポストは人名であり,数学者・論理学者・計算機科学者のエミール・レオン・ポストを指す.ポストはチューリングとは独立にチューリングマシンと等価な計算モデル(ポスト--チューリングマシン)を発表した,

		\subsection{p.102, l.13}

			\begin{quote}

				「不動点とか安定性とか,〜」
				
			\end{quote}

			ウルフラムによるオートマトンの分類で既に示したように,複雑系では,ある系について考えるとき,不動点や安定性を手がかりとすることが多い.これは素粒子物理学で対称性を手がかりとすることに似ている.

		\subsection{p.102, l.18}

			\begin{quote}

				「記述の束」
				
			\end{quote}

			“記述の束”という語は,哲学者ジョン・サールが従来の記述の理論である“フレーゲ--ラッセル見解”に修正を加えた際に導入したもの.フレーゲおよびラッセルによって一旦完成された記述の理論の詳細と,それへの批判については,文献\cite{iid1,iid2,iid3}を参照されたい.

			作中において,“記述の束”はサールやクリプキが用いたような,哲学的に正しい意味で用いられていると考えられる.なお,“束”はサールによる英語の原表記ではclusterである.このことから,17が計算機として動作するのは,“少女”の17に関する記述の束が十分複雑なクラスターであることから,ウルフラムの提唱より,17自体がチューリング完全となるため,という感じの馬鹿論理だと考えられる.

			先にも述べたが,数学的対象(ここでは特に数)に関する十分な情報を有することは,数学的能力に寄与することがある.実際,ムーンシャイン理論が発見されたきっかけは,作中にもある通り,モンスター群の既約表現の次元の線型結合と,j-不変量のフーリエ展開の項に対応関係があることに気づいたことだった.

			さて,17に関する記述の束を確認するため,作中で示されている17の性質のうち,自明でないものを列挙して説明する.

			\begin{dfn}

				プロス素数\cite{proth}

				プロス素数とは,以下の式で表され,以下のすべての制約を満たす自然数のうち,素数であるものである.

				$$N = k \cdot 2^n + 1$$

				\begin{enumerate}
					\item $k = 1, 3, 5, \cdots$

					\item $n = 1, 2, 3, \cdots$

					\item $2^n > k$
				\end{enumerate}
				
			\end{dfn}

			$(n, k) = (4, 1)$のとき$N = 17$となり,確かに17はプロス素数.また,プロス素数が無数に存在するかどうかは未解決問題.

			\begin{dfn}

				フェルマー素数\cite{seisu}

				フェルマー素数とは,非負整数$n$に対して$F_n = 2^{2^n} + 1$を満たし,かつ素数であるものである.
				
			\end{dfn}

			$n = 2$のとき$F_n = 17$となり,確かに17はフェルマー素数.また,フェルマー素数が無数に存在するかどうかは未解決問題.

			\begin{dfn}

				整数の分割\cite{ram}

				整数$n$の分割とは,$n$を任意の個数の非負整数の和に分けることである.

				このときの分割の個数を$p(n)$と表す.
				
			\end{dfn}

			0から17までの分割の個数を列挙すると,以下のようになる\cite{17}.

				$$1, 1, 2, 3, 5, 7, 11, 15, 22, 30, 42, 56, 77, 101, 135, 176, 231, 297$$

			したがって,$p(17) = 297$であることが確かめられた.

			また,ラマヌジャンは,以下の2式が恒等式であることを証明することなく直接言明した\cite{ram}.

			\begin{align*}
				p(4) + p(9) x + p(14) x^2 + \cdots = 5 \frac{ \{ (1 - x^5)(1 - x^{10})(1 - x^{15}) \cdots \}^5 }{ \{ (1 - x)(1 - x^2)(1 - x^3) \cdots \}^6 }
			\end{align*}

			\begin{align*}
				p(5) + p(12) x + p(19) x^2 + \cdots &= 7 \frac{ \{ (1 - x^7)(1 - x^{14})(1 - x^{21}) \cdots \}^3 }{ \{ (1 - x)(1 - x^2)(1 - x^3) \cdots \}^4 } \\ &+ 49 x \frac{ \{ (1 - x^7)(1 - x^{14})(1 - x^{21}) \cdots \}^7 }{ \{ (1 - x)(1 - x^2)(1 - x^3) \cdots \}^8 }
			\end{align*}

			ラマヌジャンは分割の個数$p(n)$を知らず,ハーディと共同して漸近公式を得るに留まっていたため,上記2式が恒等式であるという事実は一般式を知らないまま直感的にもたらされている.このエピソードは,ラマヌジャンの異常性がはっきりと理解出来る好例である.なお,$p(n)$の一般式は1937年にラーデマッハーによって与えられたのち,2011年に小野とブルーニエによって有限個の代数的数を用いた証明が与えられた.整数の分割についての詳細は,文献\cite{ae}を参照されたい.また,代数的数の定義は以下の通り.

			\begin{dfn}

				代数的数・超越数

				0でない有理数係数の多項式の根となるような複素数を代数的数という.

				また,代数的数でない複素数を超越数という.
				
			\end{dfn}

			超越数の代表例としては,自然対数の底$e$\footnote{エルミートによって.エルミートは量子力学におけるエルミート演算子(数学ではエルミート作用素という)にも名を残す.},円周率$\pi$\footnote{リンデマンによって.これによって,ある円と同じ面積をもつ正方形を作図出来るか否かという円積問題は否定的に解決された,}が挙げられる.また$e^{\pi} = (-1)^{-i}$\footnote{オイラーの等式$e^{\pi i} = -1$の両辺をそれぞれ$-i$乗すれば容易に得られる.これの超越数性はゲルフォントとシュナイダーがそれぞれ独立に証明した.}も超越数だが,$e + \pi$や$e - \pi$が超越数か否かは未解決問題となっている.

			余談だが,任意の自然数について,奇数のみへの分割の個数と,相異なる自然数への分割の個数が一致することはオイラーによって証明されている\cite{ono}.

			\begin{dfn}

				ジェノッキ数\footnote{作中では“ジェノッチ”と表記されているが,ジェノッキとの表記の方が一般的なようだ.}\cite{seki}

				ジェノッキ数$\{ G_n \}^{\infty}_{n=0}$は,以下の母関数で表される.

				$$\frac{2 t}{e^t + 1} = \displaystyle \sum^{\infty}_{n=1} \frac{G_n}{n !} t^n$$

				あるいは,以下の漸化式で表される.

				$$\displaystyle \sum_{0 \le k \le \frac{n}{2}} \begin{pmatrix} n \\ 2k \end{pmatrix} G_{2n-2k} = 0$$
				
			\end{dfn}

			$n = 8$のとき$G_8 = 17$となり,確かに17はジェノッキ数.

			なお,作中では「(17は)唯一の正のジェノッチ数」とあるが,正のジェノッキ数は無数に存在するため,誤りである.おそらく,“唯一の正のジェノッキ素数”あるいは単に“ジェノッキ素数”と書こうとしたものと考えられる.ジェノッキ素数は17しか存在しないことが証明されている\cite{seki}.

			\begin{dfn}
				
				ユークリッド群,壁紙群(文様群)

				2次元平面におけるユークリッド群$E$は,以下のいずれかの要素を満たす.

				\begin{enumerate}
					\item 恒等変換

					\item 一点を中心とする回転変換

					\item 平行移動

					\item 平面上のある直線を軸とする鏡映変換

					\item 平面上のある直線を軸とするすべり鏡映変換
				\end{enumerate}

				壁紙群とは,二次元平面におけるユークリッド群$E$の部分集合$W$であって,2次元平面に敷き詰められた単位図形(文様)を別の単位図形に重ね合わせられるような集合であり,以下の2条件を満たすものをいう.

				\begin{enumerate}
					\item $W$は異なる2方向への平行移動を含む.

					\item $W$に属する平行移動の移動距離の極小値は0ではない.つまり,適当な正の実数$x$について,$W$に属する平行移動の移動距離は必ず$x$より大きい. 
				\end{enumerate}

			\end{dfn}

			これを満たす壁紙群は,17通りしか存在しないことが知られている\cite{urabe}.

		\subsection{p.103, l.9-10}

			\begin{quote}

				「中国語の辞書を与えられ,〜」
				
			\end{quote}

			ジョン・サールが提唱した哲学における思考実験,中国語の部屋を指す.

		\subsection{p.103, l.11-p.104, l.8}

			\begin{quote}

				「どういうなりゆきなのかは知らないが,(中略)ゆえに少女は解放されて,こうしてボスを訪ねてきている.」
				
			\end{quote}

			直後にある通り,この記述は作中における“少女”争奪競争の顛末を率直に記述したものである.

		\subsection{p.105, l.7-8}

			\begin{quote}

				「誰かにとって黒い2は,〜」

			\end{quote}

			クオリアの話でもあるし,同じ数字をトリガーとする共感覚であったとしても,共起される色は別でありうるし,そもそも視覚ではなく別の五感の可能性だってありうる,という話.

		\subsection{p.105, l.12}

			\begin{quote}

				「素因数分解によるコードブレイク.」
				
			\end{quote}

			先述の,強大な計算能力によってRSA暗号を突破することを指す.

		\subsection{p.106, l.2-3}

			\begin{quote}

				「人間に抱き締められた最初のコンピュータの名前は,セントラル・コンピュータと言う.」
				
			\end{quote}

			真偽未確認.一般名詞すぎて調査しきれなかった.

		\subsection{p.107, l.7}

			\begin{quote}

				「これは手じゃないわけじゃない」
				
			\end{quote}

			6本指の手についての言だが,意味が不明瞭.豊臣秀吉の手は6本指であったと伝えられており,そこから天才・大物の象徴として扱われることもあるため,“少女”の天才性を強調するために導入された描写ではないかと推察される.

		\subsection{p.109, l.8}

			\begin{quote}

				「一つ一つの数字が,計算機としての性質を持ち,相互に作用する網目.」
				
			\end{quote}

			共感覚の複雑なネットワークを構成することで数が計算機として機能し,数もまた複雑なネットワークを構成してさらに強力な計算能力を持つことを示している.直後のアロステリック蛋白質は実在する蛋白質で,作中の記述の通り,複数の蛋白質が結合することで分子の構造が変化し,失活・活性化するなど,単独での生化学的作用とは異なる作用を発現させる蛋白質のこと.

			要するに,複数のものが組み合わさることで,単独では持ち得ない性質を持ち得ることがある,と言うことの事例を提示している.計算機が単純に集積しただけでは指数時間の計算を多項式時間に引き戻すことは出来ないが,思わぬ事物同士が思わぬ関係性で結ばれている,という事例としてムーンシャイン理論(の表層)を用い,ムーンシャイン理論におけるモンスター群が十分複雑であり,モンスター群と他の分野の繋がりもまた十分複雑なので,ウルフラムの提唱によってチューリング完全であるとみなしてよく,“少女”の数学的能力の上を走る計算機の存在が許される.また,たった1つの数字が計算機として振る舞うFRACTRANを考慮すれば,“少女”の上を走る計算機が数字であることもまた許され,数字同士の記述の束の集積体としての網目はさらに複雑であり,それがムーンシャイン理論(の表層)を介して“少女”自体の数学的能力を保証する.ウルフラムの提唱の下,ムーンシャイン理論--共感覚--FRACTRANは互いに互いを規定し,宙に浮く形で自立する.この空中浮遊する論理の輪廻がどのようにして生まれたかといえば,ボスの魔法陣による外部からの刺激だった.全方位に対して暴走する無秩序が,外部からの微小な秩序構造の流入によって無秩序の中の秩序構造を獲得する複雑系の過程として“少女”の覚醒を解釈することが出来,またここには系の自発的対称性の破れ\footnote{系の対称性が自発的に破れている例として,磁石が挙げられる.磁化を記述する式に対称性は陽に現れないが,磁場の向きが自発的に現れ,系の対称性は自発的に破れる.詳細は\cite{tsk}を参照されたい.}というモチーフも見える.

		\subsection{p.109, l.17-18}

			\begin{quote}

				「頂点作用素代数」
				
			\end{quote}

			Vertex Operator Algebra.リチャード・ボーチャーズ\footnote{ボーチャーズは,自身を自閉スペクトラム症であると自己診断している.ボーチャーズと実際に対面した発達障害の専門家である児童心理学者のサイモン・バロン-コーエンによれば,ボーチャーズには確かに自閉症的傾向が認められるという\cite{sym}.ただし,この記述は自閉スペクトラム症的な傾向を数学者として好ましい資質として捉え,数学者の奇異な行動を殊更面白く誇張して描こうとする書物の記述であることに注意せよ.}によるムーンシャイン理論の証明に使われた実在する数学的概念\cite{miya}.元々は物理学における場の量子論という分野のうち,特に高い対称性を持つ場合の場の量子論である共形場理論を数学的に扱う際に用いられた概念なのだが,ムーンシャイン理論にも使えることがわかり,物理数学から純粋数学に逆輸入された経緯を持つ.

			共形場理論を数学的に扱う流儀には2通りあり,1つが頂点作用素代数を用いる流儀,もう1つが作用素環の族を用いる流儀である.これらの流儀は本質的には同じことをしており,$S^1$\footnote{空間1次元+時間1次元の2次元ミンコフスキー空間に対して,共系変換を考える.空間変数を$x$,時間変数を$t$としたとき,新たな変数$t+x, t-x$を考えると,この新しい変数について研究の対象となる代数系を2つに分解することが出来る.この新しい変数$t \pm x$の動く1次元空間に無限遠点を加えてコンパクト化した一次元円周$S^1$が,カイラルな共形場理論の“時空”に相当する.\cite{kawa}.}上の作用素環のネット\footnote{ネットという語は現在では数学的に不適切なのだが,歴史的経緯から使われ続けている.実際に網目を成していたり,何らかのネットワークを司る数学的構造であることは意味しない.}と,頂点作用素代数は一致する\cite{kawa}.この頂点作用素代数は($S^1$上の作用素環の)ネットである,という数学的事実が本作の発想の元にあったのではないかと考えられる.

		\subsection{p.109, l.18}

			\begin{quote}

				「既知宇宙最大の複雑さの果ての向こう側.」
				
			\end{quote}

			巨大基数ではないかと考えられる.巨大基数の存在は,集合論の公理であるZFC公理系からは証明出来ない.また,弱到達不能基数・強到達不能基数は一般連続体仮説の下で一致することが知られている.

			“少女”は,任意の数の性質をすべて感覚的に把握出来る能力をもつ.この能力で巨大基数そのものを把握出来るのだから,“少女”は元からZFC公理系(ここでは既知宇宙)に留まることが原理的に出来ない.これが,17と19が“少女”に“数覚”を捨てるよう再三促していた理由である.“数覚”を捨てて既知宇宙に留まるにせよ“数覚”を発展的に捨てて既知宇宙を去り零から宇宙を作り直すにせよ,“少女”は結局“数覚”を捨てなければならない.ならば,既知宇宙から去る方を選ぶのは当然である.これは“少女”が大した葛藤もなく旅立ちを選んだラストシーンの描写をよく説明する.したがって,当該箇所を巨大基数を示す描写であると比定することは十分に正当化される.

		\subsection{p.109, l.18-19}

			\begin{quote}

				「月光に照らされる橋を渡った,弦理論の深奥にかかるカーテンのあちら側.」
				
			\end{quote}

			“弦理論の深奥にかかるカーテン”は意味不明な記述に感じられる.強いて挙げるならば,AdS/CFT対応\footnote{場の量子論において,$(d+1)$次元の反ド・ジッター(Anti-de Sitter, AdS)背景上の重力理論が,$d$次元の共形場理論(Conformal Field Theory, CFT)と等価である,という対応関係のこと.このような,高次元の重力理論が,低次元の重力を含まない場の理論と等価であるという対応をホログラフィといい,AdS/CFT対応はホログラフィの中で最も有名である\cite{hkt}.}があるだろうか.高次元の理論のヴェールが低次元の世界の端にかかっている,というイメージを用いて,“少女”があげるべきヴェール,進み出すべき高次世界を詩的に表現しているか.この立場に立つならば,この描写は,現代物理学の最先端の高尚で神秘的なイメージと伝統的な自然科学観\footnote{例えば,ルイ-エルネスト・バリアスによる彫刻『科学の前にヴェールを脱ぐ自然』,ヘラクレイトスによる箴言「自然は隠れることを好む」.}を折衷させ,人間による壮大な知の営みの歴史を文学の上で一覧させようとする試みとして解釈出来る.

			あるいは,直後にビッグバンを示唆する語があることから,宇宙の晴れ上がりのことか.宇宙が誕生してしばらくの間,宇宙は高エネルギー状態にあり,低エネルギー帯では結合しない電子と光子が結合してしまうことで,光が直進出来なくなってしまう.宇宙の膨張とともに宇宙が冷えていき,電子と光子が脱結合(decouple)し,光が直進出来るようになり澄み渡っていく時代のことを,宇宙の晴れ上がりという.光年という単位が示す通り,天文学において,遠くを観測することは過去を観測することと等価である.高性能の望遠鏡でより遠くを観測するのはこのためなのだが,宇宙の晴れ上がり(ビッグバンから約38万年後)より以前の領域は,光が直進出来なかったためにヴェールがかかっているように観測され,その先の過去を観測することは原理的に不可能である.いかなる努力によっても原理的に探究不能な領域が存在することの実例がここにも明示的に現れる.

		\subsection{p.109, l.10}

			\begin{quote}

				「宇宙の始点.」
				
			\end{quote}

			ビッグバンが想起される.ビッグバン以前の宇宙は,ビッグバン以降の物理が通用しないと考えられており,ビッグバン以前の宇宙について既知の物理学を適用することは出来ず,科学として言及出来ることは何もない.

		\subsection{p.110, l.1-2}

			\begin{quote}

				「計算機をいくら積み重ねてみても,ただの計算機でしかないことは言うまでもない.」
				
			\end{quote}

			繰り返しになるが,ただの計算機を集積したところで,元の計算機のクラスを超える計算機を作ることは出来ない\footnote{計算機を集積してより強大な計算力を持つ計算機を構築する手法(いわゆるスーパーコンピュータ)があるが,それでも高々多項式時間の範疇で計算を高速化しているに過ぎない.}.直後に整数全体の濃度の話が出ていることから,自然数全体からなる集合の濃度$\aleph_0$\footnote{$\aleph_0$とかいてアレフ0と読む.}とその次に大きな濃度$\aleph_1$の間に基数がないにも関わらず,$\aleph_0$をいくら足し上げても$\aleph_1$と同じ濃度にならないのと同様に,通常の計算機をいくら集積させてみたところで“少女”の計算能力に及ぶことはないということをここでも主張している.

		\subsection{p.110, l.4}

			\begin{quote}

				「ただし私は,計算機ではありえない.」
				
			\end{quote}

			ここまで,“少女”の能力を数学的能力・計算能力と表現してきたが,それは正確な表現ではない.“少女”の能力は,計算機による能力とは本質的に異なるという説明が随所でなされている.“少女”の持つ能力は,すべての数を平等にかつ負荷なく扱えるような,原理不明の生得的能力である.

			また,“少女”は任意の数についてのすべての性質を知っているので,先述の双子素数予想やゴールドバッハ予想は“少女”にとっては予想ではなく,自明な事実である\footnote{“少女”は,背理法など間接的な方法を用いることなく,“双子素数定理”の真偽そのものを提示することすら可能であろう.}.さらに,ゲーデル数化\footnote{クルト・ゲーデルによる不完全性定理の証明で導入された,形式言語に用いられる記号や論理式にそれぞれ一意な自然数を振る操作.平たくいうと,すべての単語に一意な自然数を割り振ることで,文を数字の羅列にしよう,というアイデア.}というアイデアを考慮すれば,“少女”は任意の文を知っている.このような認知体系はまさに異形であり,通常の認知機能によっては説明するどころか想像することすら出来ない.

		\subsection{p.110, l.19}

			\begin{quote}

				「何かを引き換えにしなければ手に入れることができないもの.」
				
			\end{quote}

			これは文字通りの意味.様々な創作によって人口に膾炙した表現でもある.

			このような,何かを失うことで何かを得ること,逆に何かを得ることで何かを失ってしまうことに,円城塔はしばしば関心を寄せる.特に,何かを記述すること,何かを記述することによって何かを不可逆かつ不可避に失ってしまうこと,記述しようとすることで記述出来なくなってしまうこと,というモチーフは円城塔を貫く中心的な主題である.初期作品からは「Japanese」\footnote{ハヤカワ文庫JA『Self0Rference ENGINE』収録.土中から平仮名,片仮名,平平仮名,平片仮名,片平仮名,片片仮名,$\cdots$が発見される.}や「Goldberg Invariant」\footnote{ハヤカワ文庫JA『Boy's Surface』収録.既知の数学と,それと相矛盾するパラ数学の戦いによって世界が変容する様を描く.グレッグ・イーガン「ルミナス」,ホルへ・ルイス・ボルヘス「トレーン,ウクバール,オルビス・テルティウス」から着想を得たか.}が挙げられ,近年の作品では「誤字」\footnote{新潮文庫『文字渦』収録.文字コード上における文字たちの戦乱を描く.}「かな」\footnote{新潮文庫『文字渦』収録.物語の末に声を得た“かな”は,みずから言葉を語りだす.}が挙げられる.

			先に列挙した作品とは異なり,やや分かりにくい形で登場する主題の変奏として,破壊的観測とランダウアーの原理が挙げられる.

			破壊的観測とは,観測することによって,観測対象を不可逆に破壊してしまう観測のこと.量子力学における観測がこれにあたる\footnote{このような立場が(いわゆる)コペンハーゲン解釈.現代の物理学者の多くは,このコペンハーゲン解釈に同意している.}.これが登場するのが「バナナ剝きには最適の日々」\footnote{ハヤカワ文庫JA『バナナ剝きには最適の日々』収録.バナナには皮が3枚に剥けるものと4枚に剥けるものがおり,互いに互いを憎んでいるが,何枚に剥けるかは実際に剥いてみるまでわからず,剥いてしまうとバナナは死んでしまう.}.

			一方で,ランダウアーの原理とは,以下の通り.

			\begin{prn}

				ランダウアーの原理

				不可逆な計算は必ず熱の発生を伴う.
				
			\end{prn}

			ランダウアーの原理は,不可逆計算の実行時にはエネルギーが消費されることを主張している.逆に,実行時にエネルギーが消費されないような計算として可逆計算があり,これが主題となるのが「ガベージコレクション」\footnote{ハヤカワ文庫JA『後藤さんのこと』収録.可逆計算と情報熱力学を手がかりに,時間の流れが存在することは忘却と等価であると主張.}.可逆計算の詳細については,文献\cite{kgk}を参照されたい.また,完全記憶とは記憶情報が熱的に失われないことと仮定すれば,完全記憶力をもつ人物のエントロピーは(情報論的には)増大しない\footnote{ここでは,熱の交換など熱力学的な状態量の変化は無視する.}ので,世界は時間停止する.このような主題が扱われるのが「いい夜を持っている」\footnote{新潮文庫『これはペンです』収録.完全記憶を持つ父の足跡を辿ることで,父を理解しようとする.この父に自閉スペクトラム症の表象が強固に現れていることに注意.}.

			先に挙げた2例は物理学における例だったが,物理的実体を仮定せずとも,集合論をおくだけでラッセルの定理という例を挙げることが出来る.ラッセルの定理は以下の通り.

			\begin{thm}

				ラッセルの定理

				任意の集合$A$に対して,$R_A \notin A$なる集合$R_A$が存在する.
				
			\end{thm}

			ラッセルの定理は,素朴集合論においてラッセルのパラドックスとして知られる問題を公理的集合論で整理して得られる.証明については文献\cite{russell}を参照されたい.

			ラッセルの定理は,いかに巧みに囲い込もうとも,そこから零れ落ちてしまう要素が必ず存在することを主張する\footnote{ここで石川五右衛門の辞世の句「石川や 浜の真砂は尽きるとも 世に盗人の種は尽きまじ」を連想するのは流石に考えすぎか.}.

		\subsection{p.111, l.17}

			\begin{quote}

				「生命.」
				
			\end{quote}

			“少女”は,自らの数に関する能力をすべて捨て去ろうとしている.さらに,数を捨て去ることは,生命を知ることにほかならないという.ここで,作中では物理帝国主義(物理学帝国主義,あるいは還元主義とも)\footnote{物理帝国主義,物理学帝国主義,還元主義といった語の成立や意味的差異,用例については\cite{isd}を参照されたい.}を仮定しているとするならば,この数から生命への帰結をよく説明出来る.以下,用語として還元主義を選ぶ.

			還元主義とは,ある階層の学問における問題を,より基礎的と考えられる学問の問題へと還元する姿勢のことである.例えば,生命を記述することを考えてみる.生命を規律するのは化学物質であり,生命はそれぞれの化学物質について,質量,濃度などの物理量の十分な集まりによって記述される\footnote{言語哲学における“フレーゲ--ラッセル的見解”が想起される.}ことが期待され,生物学から化学へと還元される.これら化学物質は,運動方程式や熱拡散方程式によってその振る舞いが記述されるので,化学から物理学へと還元される.こうして,生命をいかに記述するかという生物学の問題は,物理学とその記述体系である数学まで還元される.逆に,この宇宙における数学は物理のあり方を決定し,化学,生物学をも決定してしまうだろう.

			この宇宙の数学がゼータ関数の正規化を許しているがために,この宇宙ではカシミール効果という不可思議な物理現象\footnote{極少の隙間を空けて平行に配置した真空中の金属板間に引力が発生する現象.数学におけるゼータ関数の正規化を実験的に示す現象として有名.量子力学を用いて極めて正確に説明出来る物理現象なのだが,しばしば疑似科学的に語られることでも有名.}が観測されてしまう.ここから飛躍して考えれば,その体系に数を真に含まないような数学だけが許される宇宙においては,そのような数学で可能な物理が展開され,可能な物理によって規定された可能な化学が展開され,その先には可能な生命が存在するはずである\footnote{このような試みは一見意味不明かもしれないが,この宇宙における数学・物理学によって可能な生命すべてを探索する試みは,生物物理学・合成生物学・理論生物学といった分野で実際に行われている.なお,理論生物学は,円城塔の博士課程における指導教官,金子邦彦の専門分野\cite{knk}である.}.


		\subsection{p.112, l.1}

			\begin{quote}

				「どこで誰が保証しているのか誰も知ることのない同一性」
				
			\end{quote}

			チャーチ--チューリングの提唱,あるいは寝ても覚めても私は私であり続けるという同一性を指すか.

		\subsection{p.112, l.9-10}

			\begin{quote}

				「ここに既に宇宙があるせいで,新たに生まれ出ることのできない,別の宇宙たちの極小の泡.まだそこに,数は姿を見せていない.」
				
			\end{quote}

			多宇宙理論\footnote{多世界解釈ではないことに注意.}.初期宇宙のモデルのひとつで,我々の宇宙はビッグバンの際に生まれた無数の宇宙のひとつであり,その周辺には別の宇宙が存在しているとする.多くの場合,異なる宇宙間の観測は不可能であり,因果律的にも独立していることを認める.宇宙の晴れ上がり以前の宇宙の観測が困難であるため,観測による検証は不可能であると考えられる\footnote{近年,多宇宙理論(マルチバース)を支持する物理学者による意見表明が盛んに行われているが,この宇宙に含まれない,この宇宙からは観測不可能な他の宇宙を語るという反証可能性をまったく欠いた営みに,私は科学的な意義があるとは思えない.多宇宙理論を主張する書籍は多数出版されているが,ここで紹介するべき定評ある文献はないと判断したため,これ以上多宇宙理論には立ち入らず,参考文献等の紹介もしないこととした.}.

		\subsection{p.112, l.13}

			\begin{quote}

				「接触した塔を粉微塵に砕いて進み.〜」
				
			\end{quote}

			文字通り,“少女”が塔を物理的に破壊し,その根源である自身の数学能力を壊している.ホルへ・ルイス・ボルヘス「『ドン・キホーテ』の作者,ピエール・メナール」\footnote{「(e) 塔のひとつを除くことによってチェスをより豊かなものにする可能性についての技術的考察.メナールはこの改良を提案し,推奨し,議論し,そして最後にしりぞける.」\cite{jorge}p.55}や,アポトーシスを連想する.

		\subsection{p.112, l.18}

			\begin{quote}

				「数か,私か.」
				
			\end{quote}

			創成期宇宙を語っているという文脈から,物質--反物質間の対生成・対消滅を数--私間で置き換えたものと断定される.この現実宇宙において,CP対称性は自発的に破れている.元々物質と反物質は正確に1対1の割合で対生成・対消滅を繰り返していたのだが,CP対称性の破れによって,物質の生成率が反物質の生成率より僅かに大きくなってしまった\footnote{このとき,バリオン数保存則とレプトン数保存則が破れるが,実はバリオン数からレプトン数を引いた数が新たな保存量となり,これに対して保存則が成り立つ.このような保存量を$B - L$数(baryon minus lepton number)という.}ために,やがて宇宙には物質のみが存在するようになった.これがこの現実宇宙に物質のみが存在するようになった経緯である.

			ここから,“少女”の内部は,一見安定的だが,実のところ“少女”と数が対生成・対消滅を繰り返すことで平衡状態を保っている系として解釈出来る.

			一方,数という構造に注目しよう.数という数学的構造は,極めて容易に構成出来る.例えば,ペアノの公理は集合論の上の自然数のモデルである.

			\begin{axi}

				ペアノの公理

				\begin{enumerate}
					\item 自然数$0$が存在する.

					\item 任意の自然数$a$に対して,その後者である$S(a)$が存在する.

					\item $0$はいかなる自然数の後者でもない.

					\item 異なる自然数は異なる後者をもつ.

					\item $0$がある性質をもち,$a$がある性質を満たせばその後者$S(a)$もその性質を満たすとき、すべての自然数はその性質を満たす.
				\end{enumerate}
				
			\end{axi}

			ここで,ペアノの公理の5は,数学的帰納法として知られている.逆に,数学的帰納法を認める程度でしかないような集合論のみを認めた場合でも,集合論のみから自然数が出てきてしまう.ペアノの公理を満足する自然数の具体的な構成方法として,フォン・ノイマンによる構成が知られている.

			この他にも,ある体系の上で自然数として振る舞うモデルとしては,型なしラムダ計算の上のチャーチ数,圏論における自然数対象(natural number object)などが挙げられる.

			これらを素朴に考慮すれば,“少女”はかなり分の悪い賭けをしているものと考えられるが,実のところ,そのような心配は不要であろう.我々が認知出来ないような公理,そもそも数を許さない宇宙など,いくらでも回避策は考えられる.この考察が形而下の反証可能な極めて狭い範囲を対象とする数学・物理学を基盤としている以上,ここから先の領域で語り得ることは何もない.形而下の言葉しか持ち得ない我々にとっては言及不能だが,“少女”にとっては今眼前に確かに存在する新たな世界へと,“少女”は月光とともに橋を渡っていく.

		\subsection{p.113, l.2}

			\begin{quote}

				「月光に照らされ乱反射して降り注ぐ不可視の破片の雨の中,私は橋を渡りはじめる.」
				
			\end{quote}

			クライマックス.円城塔は,自身のCosense\footnote{旧称Scrapbox.}\cite{hamete}において,“ハメ手を持つこと”について紹介している.ここで用いられているのは,“2人の登場人物が,時空的に離れた場所で,それぞれモノローグする”と,“理詰めで押し続けるように見せて,限界に達したところで破綻させる”である.

			ここまでで見てきた通り,本作は数学・物理学・情報理論・言語学・哲学の諸概念が象徴的かつ正確な意味を帯びて散りばめられ,しかもそれらが極めて複雑に絡み合い,絡み合った関係性自体が新たな意味を獲得して更なる連想をもたらすような構成となっている.そのような構成で語られるのは,思わぬ物事同士が関連しあい.関係性のネットワークが暴走して計算機を構成し,十分複雑な構造を成す計算機が生命として振る舞うという物語だった.ここで,我々は,複雑なテクストが,複雑なものはそれ自体が生命として振る舞うと主張する複雑な物語を語り,その物語の主張を自ら体現している様を目撃する.すなわち,複雑なものが,複雑なものを複雑に記述し,その複雑な記述によって自分自身の存在が規定されているという,自己言及構造を認める\footnote{このような自己言及構造は,最初の入力がなければ,不安定点上にあるものの駆動しないこともまた分かる.最初の入力とは,読者による読書である.このような,テクスト--読者から成る読書という系の形式化と数理的考察は,円城塔作品に頻出する主題である.また,逆に,一度読まれて(駆動させられて)しまったがゆえに,どうしようもなく在らされてしまう物語というものが考えられる.これもまた,円城塔作品に頻出する主題である.}.

			このような自己言及構造を伴う複雑な本作において,読者は理解の限界に至る.このことによって,読者は抑圧され,その抑圧が張り詰めていった最後で,“私”の語りは急に視覚的になり,美しい情景が描写される.この視覚的描写は読者にとって非常に理解しやすいものであり,それまでの数理科学的に難解な描写による抑圧は解放される.また,既存の世界に留まらざるを得ない“僕”,自らが消滅することを知りながら“少女”を見送る17と19,そして自己破壊によって新たな宇宙へと旅立つ“少女”の間で交わされる会話は,相互対話としてではなく,モノローグとして振る舞う.

			円城塔は,複雑な物語を複雑な構造の上に構成することで,物語自身が自身を規律する構造を設定した.さらに,このような複雑な物語によって読者を巧みに抑圧し,ラストシーンで一気に解放させることで,カタルシスを強調している.これらのことから,本作は,膨大な人類知を文芸的技巧によって調和させた,科学と文学の融合たる優れたサイエンス・フィクションであると言える.

	\section{謝辞}

		本稿の執筆中,本項で扱おうとしている数学的対象について話題を振るたび,驚くほど多くの数学的事実を与えてくださった名古屋大学大学院多元数理科学研究科のI氏に深く感謝申し上げる.

	\begin{thebibliography}{99}

		\bibitem{moonshine} 円城塔, 『ムーンシャイン』, 創元日本SF叢書, 東京創元社, 2024

		\bibitem{webster} America's most trusted dictionary, Merriam--Webster, \url{https://www.merriam-webster.com}

		\bibitem{oeis} \textit{The On-Line Encyclopedia of Integer Sequences}, \url{https://oeis.org/}

		\bibitem{ysn} 吉永正彦, プロの研究者はどうやって研究をおこなっているのか, \textit{数学セミナー}, \textbf{57}(7)(669), 30-35, 2017

		\bibitem{ram} G・H・ハーディ, 『ラマヌジャン』, 丸善出版, 2016

		\bibitem{ft} Walter Feit, John G. Thompson, Solvability of groups of odd order, \textit{Pacific Journal of Mathematics}, \textbf{13}(3), 775-1027, 1963

		\bibitem{oo} odd-order, \url{https://github.com/math-comp/odd-order}

		\bibitem{atlas} J. H. Conway, R. T. Curtis, R. A. Wilson, S. P. Norton, R. A. Parker, \textit{ATLAS of Finite Groups}, Oxford University Press, 1985

		\bibitem{yke} 雪江明彦, 『代数学1』, 第2版, 日本評論社, 2023

		\bibitem{dgn} H・P・ラヴクラフト, 『ラヴクラフト全集3』, 創元推理文庫, 東京創元社, 1984

		\bibitem{dl} ディオゲネス・ラエルティオス, 『ギリシア哲学者列伝 上』, 岩波文庫, 岩波書店, 1984

		\bibitem{gonoji} 五之治昌比呂, 『吾輩は猫である』の二つの逸話の材源について : アイスキュロスの死とアグノディケ, 西洋古典論集, (24), 47-65, 2016, \url{http://hdl.handle.net/2433/217014}

		\bibitem{naka} 中務哲郎, 『イソップ寓話の世界』, ちくま新書, 筑摩書房, 1996

		\bibitem{russell} 下村思游, “集合の集合”が集合でないことの証明, \textit{円城塔研究}, \textbf{1}(1), 11-14, 2023

		\bibitem{mino} Hidetoshi Mino, The number of magic squares of order six counted up to rotations and reflections, \textit{The number of magic square of order 6}, 2024, \url{https://magicsquare6.net}

		\bibitem{ktj} Akimasa Kitajima, Macoto Kikuchi, Numerous but rare : an exploration of Magic Squares, \textit{PLoS ONE}, \textbf{10}(5), 2015, \url{https://doi.org/10.1371/journal.pone.0125062}

		\bibitem{pdg} Particle Data Group, Review of particle physics, \textit{Physical Review D}, (110), 030001, 2024, \url{https://doi.org/10.1103/PhysRevD.110.030001}

		\bibitem{si} 『国際単位系(SI) 第9版(2019) 日本語版』, 産業技術総合研究所, 2020, \url{https://unit.aist.go.jp/nmij/public/report/si-brochure}

		\bibitem{ichi} 一国雅巳, 『無機地球化学』, 培風館, 1972

		\bibitem{bbf} T. Rado, On non-computable functions, \textit{Bell System technical journal}, \textbf{41}(3), 877-884, 1962, \url{https://doi.org/10.1002/j.1538-7305.1962.tb00480.x}

		\bibitem{bb5} Coq-BB5, 2024, \url{https://github.com/ccz181078/Coq-BB5}

		\bibitem{bb20} Scott Aaronson, The Busy Beaver frontier, \textit{ACM SIGACT News}, \textbf{51}(3), 32-54, 2020, \url{https://doi.org/10.1145/3427361.3427369}

		\bibitem{myzw} 宮沢賢治, 『宮沢賢治全集1』, ちくま文庫, 筑摩書房, 1986

		\bibitem{fkf} 『古事記』, 改版, 岩波文庫, 岩波書店, 2007

		\bibitem{seisu} 小林銅蟲, 関真一朗, 『せいすうたん1』, 日本評論社, 2023

		\bibitem{pais} アブラハム・パイス, 『神は老獪にして…』, 産業図書, 1987

		\bibitem{zdn} 円城塔, 布施琳太郎, 平川綾真智, 「詩情」の変換 : AI時代に編む言葉, \textit{現代詩手帖}, \textbf{67}(9), 92-105, 2024

		\bibitem{gr} アポロストス・ドキアディス, 『ペトロス伯父と「ゴールドバッハの予想」』, 早川書房, 2001

		\bibitem{papa} ヒッポリュトス, 『全異端反駁』, キリスト教教父著作集19, 教文館, 2018

		\bibitem{gre} リチャード・L・グレゴリー, 『脳と視覚』, ブレーン出版, 2001

		\bibitem{2of2} 70000...00003(216-digits), Prime Curios!, \url{https://t5k.org/curios/page.php?number_id=7463}

		\bibitem{prime} 11111...11111(517-digits), Prime Curios!, \url{https://t5k.org/curios/page.php?number_id=2753}

		\bibitem{ide} 井出正和, 『科学から理解する自閉スペクトラム症の感覚世界』, 金子書房, 2022

		\bibitem{fra} Ronald T. Kneusel, 『ストレンジコード』, 秀和システム, 2024

		\bibitem{pyfra} 下村思游, PyRACTRAN, 2024, \url{https://github.com/ShiyuuShimo/PyRACTRAN}

		\bibitem{sip3} Michael Sipser, 『計算理論の基礎3』, 原書第3版, 共立出版, 2023

		\bibitem{bio} クラウス・エメッカ, 『マシンの園』, 産業図書, 1998

		\bibitem{mth} Matthew Cook, Universality in elementary celluar automata, \textit{Complex Systems}, \textbf{15}(1), 1-40, 2004, \url{https://doi.org/10.25088/ComplexSystems.15.1.1}

		\bibitem{smth} Alex Smith, Universality of Wolfram's 2, 3 Turing Machine, \textit{Complex Systems}, \textbf{29}(1), 1-44, 2020, \url{https://doi.org/10.25088/ComplexSystems.29.1.1}

		\bibitem{ethica} スピノザ, 『エチカ 上』, 岩波文庫, 岩波書店, 2011

		\bibitem{iid1} 飯田隆, 『言語哲学大全Ⅰ』, 増補改訂版, 勁草書房, 2022

		\bibitem{iid2} 飯田隆, 『言語哲学大全Ⅱ』, 増補改訂版, 勁草書房, 2023

		\bibitem{iid3} 飯田隆, 『言語哲学大全Ⅲ』, 増補改訂版, 勁草書房, 2024

		\bibitem{proth} Eric W. Weisstein, Proth numbers, \textit{The On-line Encyclopedia of Integer Sequences}, 2003, \url{https://oeis.org/A080075}

		\bibitem{17} N. J. A. Sloane, a(n) is the number of partitions of n (the partition numbers), \textit{The On-line Encyclopedia of Integer Sequences}, 2001, \url{https://oeis.org/A000041}

		\bibitem{ae} ジョージ・アンドリュース, キムモ・エリクソン, 『整数の分割』, 数学書房, 2006
		
		\bibitem{ono} Jan Hendrik Bruinier, Ken Ono, Algebraic formulas for the coefficients of half-integral weight harmonic weak Maass forms, 2011, \url{https://doi.org/10.48550/arXiv.1104.1182}

		\bibitem{seki} 関真一朗, ジェノッキ素数, \textit{数学セミナー}, \textbf{57}(7)(669), 20-22, 2017

		\bibitem{urabe} 卜部東介, 文様の群論, \textit{茨城大学卜部東介数学研究室数学博物館}, \url{https://www.ms.u-tokyo.ac.jp/~tsuboi/urabe/public_html/pattrn/Pattern2.html}

		\bibitem{tsk} 田崎晴明, 『統計力学Ⅱ』, 培風館, 2008

		\bibitem{sym} マーカス・デュ・ソートイ, 『シンメトリーの地図帳』, 新潮文庫, 新潮社, 2014

		\bibitem{miya} 宮本雅彦, 頂点作用素代数入門, \textit{数理解析研究所講究録}, \textbf{867}, 88-98, 1994, \url{http://hdl.handle.net/2433/83950}

		\bibitem{kawa} 河東泰之, 共形場理論と作用素環,頂点作用素代数, \textit{総合講演・企画特別講演アブストラクト}, (2005), 2005, \url{https://doi.org/10.11429/emath1996.2005.Spring-Meeting_73}

		\bibitem{arai} 新井敏康, 『数学基礎論』, 増補版, 東京大学出版会, 2021

		\bibitem{hkt} 疋田泰章, 『共形場理論入門』, 講談社, 2020

		\bibitem{kgk} 森田憲一, 『可逆計算』, 近代科学社, 2012

		\bibitem{isd} 伊勢田哲治, 物理(学)帝国主義という言葉を使い始めたのはだれか, \textit{Daily Life}, 2023, \url{http://blog.livedoor.jp/iseda503/archives/1935568.html}

		\bibitem{knk} 金子邦彦, 『普遍生物学』, 東京大学出版会, 2019

		\bibitem{jorge} ホルヘ・ルイス・ボルヘス, 『伝奇集』, 岩波文庫, 岩波書店, 1993

		\bibitem{hamete} 円城塔, ハメ手を持つこと, 2018, \url{https://scrapbox.io/enjoetoh/%E3%83%8F%E3%83%A1%E6%89%8B%E3%82%92%E6%8C%81%E3%81%A4%E3%81%93%E3%81%A8}

		\bibitem{hmt} 下村思游, “ハメ手”を持つこと実例解説1 : ムーンシャイン, \textit{円城塔研究}, \textbf{1}(2), 7-9, 2023

	\end{thebibliography}

\end{document}