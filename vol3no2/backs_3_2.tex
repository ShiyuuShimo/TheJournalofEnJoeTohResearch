\documentclass[10pt, a5paper, twoside]{ltjsarticle}

%\usepackage{okumacro}
\usepackage{enumitem}
\usepackage{amssymb}
\usepackage{amsmath}
\usepackage{amsfonts}
\usepackage{amsthm}
\usepackage{bm}
\usepackage{url}
\usepackage{here}
\usepackage{graphicx}
\usepackage{wrapfig}
\usepackage{makeidx}
\usepackage{braket}
%\usepackage{ascmac}
\usepackage{fancyhdr}
\usepackage{tcolorbox}
\usepackage{tikz}
%\usepackage{otf}
\usepackage{luatexja-otf}
\usepackage{luatexja-ruby}
\usepackage[top=20truemm,bottom=20truemm,left=15truemm,right=15truemm]{geometry}
\usepackage{mathrsfs}

\pagestyle{fancy}
	\fancyhead{}
	\fancyhead[RE]{円城塔研究}
	%\fancyhead[LO]{TITLE OF ARTICLE}
	\fancyhead[LE, RO]{\thepage}
	\fancyfoot{}
	\fancyfoot[LE, RO]{\footnotesize{The Journal of EnJoeToh Research, Vol.3, No.2, 2025} }

\theoremstyle{definition}
	\newtheorem{dfn}{定義}
	\newtheorem{thm}{定理}

\setcounter{page}{26}

\begin{document}

	%\begin{center}

		%\Large{\textit{memorandum}}

	%\end{center}

	%\newpage

	{\large 執筆者紹介}

	\vspace{3mm}

	下村思游

	 司書,SF研究者,SFレビュアー.専門は素粒子物理学,加速器科学.

	 著書に『ハヤカワ文庫JA1500総解説』(共著,早川書房,2022).

	 直近の仕事は『SFマガジン』2025年8月号ウィリアム・ギブスン特集.

	 Bluesky: @ss-scifi.bsky.social

	 HP: \url{https://sfpromenade.net}

	 mail: shiyuu.shimomura\_at\_gmail.com : \_at\_ $\rightarrow$ @

	\vfill

	\hrulefill

	\center

	{\Large 円城塔研究 3巻2号}

	2025年11月23日 発行

	\flushleft{編集 数理文学研究会\\    東京, 日本\\発行 柏屋\\    東京, 日本, \url{https://sites.google.com/view/kashiwaya}\\ISSN 2759-0178(print)\\ISSN 2758-9358(online)}

	\hrulefill

	内容に関する質問・意見等は shiyuu.shimomura\_at\_gmail.com : \_at\_ $\rightarrow$ @




\end{document}