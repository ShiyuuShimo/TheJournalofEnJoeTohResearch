\documentclass[10pt, a5paper, twoside]{ltjsarticle}

%\usepackage{okumacro}
\usepackage{enumitem}
\usepackage{amssymb}
\usepackage{amsmath}
\usepackage{amsfonts}
\usepackage{amsthm}
\usepackage{bm}
\usepackage{url}
\usepackage{here}
\usepackage{graphicx}
\usepackage{wrapfig}
\usepackage{makeidx}
\usepackage{braket}
%\usepackage{ascmac}
\usepackage{fancyhdr}
\usepackage{tcolorbox}
\usepackage{tikz}
%\usepackage{otf}
\usepackage{luatexja-otf}
\usepackage{luatexja-ruby}
\usepackage[top=20truemm,bottom=20truemm,left=15truemm,right=15truemm]{geometry}
\usepackage{mathrsfs}

\pagestyle{fancy}
	\fancyhead{}
	\fancyhead[RE]{円城塔研究}
	\fancyhead[LO]{京フェス2025円城塔賞部屋資料}
	\fancyhead[LE, RO]{\thepage}
	\fancyfoot{}
	\fancyfoot[LE, RO]{\footnotesize{The Journal of EnJoeToh Research, Vol.3, No.2, 2025} }

\theoremstyle{definition}
	\newtheorem{axi}{公理}
	\newtheorem{dfn}{定義}
	\newtheorem{thm}{定理}
	\newtheorem{hyp}{予想}
	\newtheorem{ths}{提唱}
	\newtheorem{prn}{原理}
    \newtheorem{clm}{主張}

\setcounter{page}{6}

\begin{document}

	~ %強制改行

	\begin{center}

		\Large{京フェス2025円城塔賞部屋資料}

		\vspace{3mm}

		\large{下村思游}

	\end{center}

	\vspace{3mm}

	\section{本編}

        今年の円城塔部屋は「円城塔作品の嘘について語る部屋」.

        円城塔作品の中でも,特に初期作品において,どこで嘘をついているのかを整理しようという試みだった.が,実はあまり嘘をついていないのではないかという疑惑が浮上.複雑系について勉強すればするほど,至って当然のことを言っているように感じるようになった.

        \subsection{Self-Reference ENGINE}

            みなさんご存知の通り,ひどく複雑なため一旦スキップ.

        \subsection{Boy's Surface}

            \subsubsection{Boy's Surface}

                実はほとんど全く嘘をついていない.

                \vspace{3mm}

                計算機は数学を理解していない,というのは真.すべての数学の証明は,それぞれある記号列に変換することができる.(ゲーデル数化)(この事実が定理証明支援系の存在を保証している)

                これによって,定理証明支援系は,その意味論を感知せずとも,連続的な記号操作という形式的な手続きによってすべての証明を扱うことができる.古典的には,ヒルベルトの名言「点,線,平面を椅子,机,ビールジョッキにそれぞれ置き換えても数学は何ら問題としない」が挙げられる.ソーカルが“数理科学の論文は「テクスト」ではない”と言ったのは,このあたりに源流がある.逆に,円城塔は,数理科学で述語として用いられている日常語を小説に持ち込み,述語としての意味と日常語としての意味を重ね合わせて使うことがよくある.

                \begin{quotation}
                    ex: この「重ね合わせ」も量子力学で使われる述語.意味が“重ね合わせ”られるならば“意味”というものは実は量子的であって,複数の意味を持つ単語を記述することで作中世界が分岐する可能性があり(多世界解釈),実際それを支持する描写が作中に存在していて,というように円城塔っぽい話を展開することが可能.
                \end{quotation}

                \vspace{3mm}

                (事物)は〜を感じない,という言い回しは,少なくとも素粒子物理学では多用する.例:ニュートリノは光を感じない(=ニュートリノは電磁場と相互作用しない)ここで,ニュートリノにとって,光はこの世界にあってもなくてもどうでもいい.なぜなら直接相互作用しないから.

                そもそも相互作用(観測)出来ないのであれば,それは観測者にとってあるともないともどちらとも言えない.この見解は円城塔が好んでいる内在物理学と一致するように思う.円城塔作品には,描写されたものはすべて(作中世界において)真であり,しかも実在する,という特徴があるかも.

                \vspace{3mm}

                昔はよくわからなかったけど,物理学の理解,特に複雑系への理解が進むと,至って自明なことを並べているだけのように感じられる.物理学を血肉にした人(無論それはマイノリティである)による当事者文学としての検討も必要か.

        \subsection{オブ・ザ・ベースボール}

            \subsubsection{オブ・ザ・ベースボール}

            嘘はついていないが,テキトーなことしか言っていない.純文学の公募新人賞を通った作品だからか,他の初期作品に比べて明らかに数理を使っていない.文体自体もかなり毛色が違う.分量がアイデアに対して長すぎる印象がある.間延びした印象なのは,本来使われる数理がないことと,規定枚数を満たすためと推察される.数理の欠如も純文学向けのファインチューンであろう.池澤夏樹による芥川賞選評でもアイデアに対して長すぎると指摘されている.

            「グライダーは生命だ」という記述がある.これはウルフラムの提唱の現れ.ウルフラムのテーゼ:十分複雑なものはチューリング完全であり,それは生命でもある.

            \vspace{3mm}

            90年代の複雑系研究者の間で流行っていたジャーゴン.複雑系における宗教的な信念とも言える.木口まこと(菊池誠)の小説「わたしとわたしが旅するところ」にもウルフラムのテーゼが現れている.

            円城塔曰く,

                \begin{quotation}
                    魔方陣もので,最初のあたりにある,\\ 「じゃあ,あそこには生命がいるのね」\\ にどこまで乗れるか次第,というところではないかと思います.
                \end{quotation}

            この 「じゃあ,あそこには生命がいるのね」というのがウルフラムのテーゼそのもの.そしてそれはノレるかノレないか怪しい言説でもある.

            ただし,ウルフラムのテーゼは作品の本質には関与していないように思われる.

        \subsection{これはペンです}

            \subsubsection{これはペンです}

                叔父は文字だ.文字通り.」という出だしから始まる.最初に一見意味不明だが興味深い命題を示し,それが正しいことを丁寧に確認し,最後にその命題の帰結として得られる驚くべき事実を示して終わる.

            \subsubsection{良い夜を持っている}

                記憶の街が勝手に時間発展してしまう,というのが嘘.最後の破綻が印象的.「ムーンシャイン」に似た構築のされ方のように思う.謎の多い父の跡を追う,という文学的な主題を主軸に,嘘か本当か分かりにくいが実は正しいアイデアを散りばめ,物理学的な考察を進めて物語を構成していく.

                \vspace{3mm}

                「目覚めると,今日もわたしだ.」という出だしから始まる.グレッグ・イーガン「貸金庫」

                    \begin{quotation}
                        ありふれた夢を見た.わたしに名前がある,という夢を.ひとつの名前が,変わることなく,死ぬまで自分のものでありつづける.それがなんという名前かはわからないが,そんなことは問題ではない.名前があるとわかれば,それだけでじゅうぶんだ.(山岸真訳)
                    \end{quotation}

                \vspace{3mm}

                完全記憶を持つ父は,自らの住む街をそっくりそのまま記憶し(記憶の街),さらに絵として出力することができた.父の記憶力は十分強力なので,記憶の街は現実の街とは独立に時間発展し,双方の状態は乖離していく($\because$初期値鋭敏性).

                13歳にしてはじめて覚醒した,というのはいささか過剰な描写ではあるが,嘘とまでは言いにくい.江戸川乱歩,現世は夢,夜の夢こそまこと.

                また,父には自閉スペクトラム症の特徴が明確に現れていることに注意.遠い過去の記憶と近い過去の記憶が両方同程度に明晰なため,時間感覚が消失するという特徴が明らかに現れている.「ガベージコレクション」の主題,忘却と時間発展は等価であるという主張を裏返したもの.完全記憶をもつ者は,時間発展を認識しにくいとする.自閉スペクトラム症の表象は天才表象の一部として「ムーンシャイン」にも現れている.精神医学の専門家の意見を仰ぎつつ,表象研究の側面からも検討すべき.脳神経に関する複雑系の研究では,定型発達の脳神経構造の比較対象として,自閉スペクトラム症者の脳神経構造をしばしば用いる.

                \vspace{3mm}

                素朴物理学という概念がある.これは人間が自然現象を認知するときに用いる素朴な知識体系で,人間が生得的にもつものであるとされている.

                例えば,今,ボールを上に投げ上げるとする.このとき,ボールに働く力の向きはどの方向か? もちろん答えは下向きである($\because$ボールに働く力は重力のみであるから).これは中学理科の内容であり,もちろん全国民が習得しているべき知識だが,即答できただろうか?間違える人もいるだろうし,正解した人も少し時間がかかったのではないか?間違ったり時間がかかったりしても恥ではない.このように,簡単なはずの問題なのに,いかに学力的に優秀な集団(東大,東北大,ハーバード大,MIT)であっても,専門家(物理学科の学生,物理学者など)以外では3割程度は間違えてしまうことが知られている.このように,教室で学習した内容と,日常生活での認知が乖離してしまうことを,提示事例の個別学習という.

                このような,素朴物理学・素朴生物学・素朴心理学の研究は近年欧米で盛んに行われているが,欧米に先駆けて,東北大教育学部ではこの提示事例の個別学習について研究が極めて盛んに行われていた.逆に,専門家はこの素朴物理学を訓練によって克服している.一方で,物理学で博士号を取得した人間にも,このような素朴物理学の痕跡がわずかに残っていて,素朴物理学が知識に先行して発火することが知られている.素朴物理学が現代の人類に残ってしまっているのは,物理学的に正確な世界認識が生物の生存競争で特に有利ではなかったということに由来すると思われる.20世紀初頭以降,人類は数学・論理学・物理学でそれまでの素朴な直感を裏切る重要な成果を手にした.それらが一向に世間に広まらないのは,この素朴物理学・素朴数学が邪魔をしているのではないか? 理学部の課程は,この素朴物理学・素朴数学を乗り越えるための知的訓練なのではないか?

                \vspace{3mm}

                数学・物理学の学生は,数学・物理学を自らの血肉とするように徹底的な教育を受ける.複数の教官が“勉強しようと思って勉強する奴は話にならない,起きてる時は常に数学・物理学を考えるように生きろ”ということを言っている.
                \begin{quotation}
                    「朝起きた時に,きょうも一日数学をやるぞと思ってるようでは,とてもものにならない.数学を考えながら,いつのまにか眠り,朝,目が覚めたときは既に数学の世界に入っていなければならない.」佐藤幹夫(佐藤超関数,概均質ベクトル空間の理論の創始者)\footnote{\url{https://program.math.tsukuba.ac.jp/emeritus/kimurata/}}
                \end{quotation}

                \begin{quotation}
                    「今,週40時間と書きましたが,アルバイトなどがなければ,院生がその2倍くらい勉強する(こともある)のは何もおかしくないのではないでしょうか.時間的余裕があるのにそういうことを苦痛だと思う人は,数学には向いていないと思います.」河東泰之(作用素環論)\footnote{\url{https://www.ms.u-tokyo.ac.jp/~yasuyuki/sem.htm}}
                \end{quotation}

                これらの恐ろしい主張は,人類が生得的に有する素朴な観念を捨て去るために必要な心構えなのではないか?

                \vspace{3mm}

                提示事例の個別学習が発生する,つまり日常生活で素朴物理学等が優越する事象が発生するのは,脳が認知資源を節約しようとするからである.日常生活でいちいち物理学的に厳密な理解をしていたら,脳が疲れ果てて食事を取れずそのまま死んでしまうだろう.

                一方で,はじめから認知資源の節約という機能が不全を起こしている人々がいる.自閉スペクトラム症者である.特徴的な構造への異常な執着が数学と相性がいいとの指摘がある.実際,ムーンシャイン予想を証明したリチャード・ボーチャーズは自閉症的傾向が著しく,本人が苦痛を覚えているのであれば間違いなく自閉スペクトラム症と診断されるとの見解が専門家からなされている.ボーチャーズ本人も自身を自閉症であると自己診断しているが,生活に困難を覚えていないために診断は確定されていない.自閉スペクトラム症は症状が客観的に識別されることをその診断の前提とするが,本人がその症状によって苦しんでいなければ病気であるとは診断されない.人間関係に興味がなく,抽象化された概念を好む傾向も,数学者・物理学者として好ましい素質である

                とすると,物理学者や数学者に自閉的傾向が強い人間が多いのは,先天的な要素によって数学・物理学に惹かれる人間も多いが,徹底した知的訓練によって後天的に自閉的な傾向を叩き込まれているという側面があるのではないか.理学部物理学科出身者として,自閉的傾向のある人間が明らかに多いことは事実であると明言する.確かに世間のイメージは正しいのだが,それはもしかしたら専門家になるための教育によって獲得させられた態度かもしれない.

                \vspace{3mm}

                複雑系の人は子供が好き,という話を聞いた.人間の脳は,特に学習を強いられているわけではないのに,勝手に言語を学習し,勝手に歩き始める.今やニューラルネットワークが当たり前になってしまったが,それでもNNの学習には膨大な計算資源で莫大なエネルギーを費やしながら長時間学習をぶん回す必要がある.しかし,人間の脳は,遥かに計算能力が劣るにも関わらず,一人でに学習してしまう.物理学でもそれを説明する理論を考えているが,やはり実物を見ると,物理学者としての心が動かされるとのこと.

        \subsection{道化師の蝶}

            \subsubsection{道化師の蝶}


        \subsection{ムーンシャイン}

            \subsubsection{パリンプセスト}

                嘘はあまりついていない.よくわからない8つのお話の集積体.結局のところ,全体としてもよくわからない作品だが,わたしたちはある対象についてどこまでわかることが出来るだろうか,という話をしているように思う.最後は科学の営みを「群盲象を評す」で一言に表している.

            \subsubsection{ムーンシャイン}

                この中では一番嘘をついているし,述語の用い方がやや不適当.異常な計算能力をもつ少女とのボーイミーツガール.最後の破綻が印象的.理詰めて押し進めて読者の理解を拒みつつ,最後に視覚的な情景を構成して落としにかかる.この作品にはウルフラムのテーゼが色濃く出ている.

    \section{コメント}

        複雑系の勉強をするにつれて,自分が強烈な素粒子物理学の信奉者であることを初めて自覚した.円城塔は,典型的な素粒子物理学の盲信者であるワインバーグを批判している\footnote{\url{https://note.com/note_engine/n/n881b8da440e7}}.

        つまるところ,円城塔と私とでは,物理学に対する宗教的な信念が生まれつき異なる.物理学者は2種類に分けられる.熱力学が好きな者と,嫌いな者である.物性・複雑系は前者で,素粒子・原子核・宇宙は後者.無論,円城塔は前者であり,私は後者.

        (あらゆる研究に対して言えることではあるが)この発表内容を鵜呑みにせず,別の視点から検討されることを期待する

    \section{質疑応答}

        この節は後日追記された.

        \begin{itemize}
            \item 定理証明支援系が同値変換だとするならば,最初から自明なのでは?

            \vspace{2mm}

             同値変換のみではない.イメージとしては,数学の証明をレゴブロックのようにパーツ分けして,決められたパーツ(=公理系)だけで証明しようという感じ.使えるパーツを予め制限し,使ってはいけないパーツを使ったらそれを検知するというのが定理証明支援系と言われる所以.なので,通常の数学の証明能力を損なうことなく,正確な証明を構成することが可能になる.

            (余談)現代数学の証明に穴はないのでは,という指摘がありえるが,あるということを明言する.数学者が無意識に導入してしまう重要な公理として,選択公理がある.これは非常に有名かつ強力な公理だが,あまりにも当たり前すぎて数学者の認知からしばしば零れ落ちてしまう.具体的には,無限集合と無限集合の直積をとる操作には選択公理が必要.このような,選択公理と同値な命題を一覧したい場合は\cite{algd}を,数学の形式化の概説については\cite{garrigue}を参照されたい.

            \vspace{2mm}

            \item 円城塔作品には生命を扱う作品が多いと説明があったが,納得できない.実例を挙げてほしい.

            \vspace{2mm}

             「ムーンシャイン」がまさにそう.同作では,17という数字が知性をもち,しかも人格を獲得してキャラクターとして振舞う.これは素朴には意味不明だが,ウルフラムのテーゼを認めれば,これを自然に解釈可能.すなわち,17という数は,無数の数学的性質に関する「記述の束」(フレーゲ-ラッセル-サール)によっている.この「記述の束」というカスケードは十分複雑であり,したがってそれはチューリング完全であり,生命であって,それが人格を獲得して語り始めたとしても何ら不思議ではない,といった感じ.ただし,ここにはウルフラムのテーゼによる第一の飛躍と,人格を獲得するという第二の飛躍がある.

            「ムーンシャイン」は,素朴に読むと,最後のシーンで急に生命について語りだす意味不明な作品のように読める.しかし,ウルフラムのテーゼ(とさらにもう一段の飛躍)を認めるならば,最初から生命の話しかしていない.他にも,「微字」「緑字」「捧ぐ緑」「お父さんの娘」も挙げられる.

            \vspace{2mm}

            \item 17以外も十分複雑でありうるか?

            \vspace{2mm}

             任意の数についていくらでも構成可能だろう.17や19が作中でピックアップされたのは,それらが特徴的な性質を多く持つ素数であって,小説内において性質の説明を自然に展開可能であるからという文芸的要請によるものであると思う(特に,素数でなければならない理由とは,同作がFRACTRANをモチーフにしているから).例えば,1ですらいくらでも性質を列挙可能.実際.作中の作法に則って,1は乗法単位元である,最小の正の整数である,自身の階乗と自身が一致する最小の数である,約数の和が自身となる唯一の数である,最小のメルセンヌ数である,最小のカタラン数である,最小のハーシャッド数である,最小のリュカ数である,など.

            \vspace{2mm}

            \item AIとの対話を通じた研究はしないのか

            \vspace{2mm}

             文献調査にGeminiやGoogle検索のAIモードを使うことはあるが,研究そのものの相談はしない.ニッチな分野の情報は既に各種LLMに取り込まれてしまっており,特に私が定式化した「ウルフラムの提唱」などは私の言っていることがそのまま出力として返されてしまう.文学研究,特に現代作家に対する同時代的な作家研究においては,作品のテクストがLLMに取り込まれていない点,定説がなく素人が好き勝手に発信した情報が取り込まれている点などから,LLMの回答精度が著しく低いことが容易に想定されるため,そもそも試みてすらいないし,今後試みることもないだろう.

            (後日談)多少試行してみたが,やはり私自身の主張を学習して鸚鵡返しになっているので無価値だった.

            \vspace{2mm}

            \item (円城塔によるワインバーグ批判に対して)単純なものの集まりが全体を成す,というのはやはり当然のことなのではないか.

            \vspace{2mm}

             それがまさに素朴物理学に由来する妄念.20世紀を通じて,人類は素朴な還元主義が成り立たないことを繰り返し確認してきた.具体例としては熱力学,統計力学,複雑系が挙げられる.還元主義のみでこの世界が説明できないのは,もはや自明であるとしたいし,円城塔はそのようなことを繰り返し述べている(が,難しすぎてほとんど誰も気付いていない).

        \end{itemize}

    \begin{thebibliography}{99}
        \bibitem{algd} alg-d, 選択公理, \textit{壱大整域}, \url{https://alg-d.com/math/ac/}

        \bibitem{garrigue} Jacques Garrigue, 「Coq : 型理論から来た証明支援系」, 『数学とAIのこれまで(とこれから)』, 数学セミナー増刊, 日本評論社, 2025
    \end{thebibliography}

    \section*{注記}

        本資料は,2025年に開催された京都SFフェスティバル2025で行われた「円城塔の法螺話を語る部屋」のために制作された2025年10月の原資料\footnote{\url{https://scrapbox.io/allreferenceengine/%E4%BA%AC%E3%83%95%E3%82%A7%E3%82%B92025%E5%86%86%E5%9F%8E%E5%A1%94%E9%83%A8%E5%B1%8B%E8%B3%87%E6%96%99}}を2025年11月にPDFとして再編集したものである.

        質疑応答については後日補ったが,それ以外の箇所は,特に断りのない限り,全て現資料が成立した2025年10月当時の記述をそのまま転記している.

\end{document}