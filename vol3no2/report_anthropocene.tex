\documentclass[10pt, a5paper, twoside]{ltjsarticle}

%\usepackage{okumacro}
\usepackage{enumitem}
\usepackage{amssymb}
\usepackage{amsmath}
\usepackage{amsfonts}
\usepackage{amsthm}
\usepackage{bm}
\usepackage{url}
\usepackage{here}
\usepackage{graphicx}
\usepackage{wrapfig}
\usepackage{makeidx}
\usepackage{braket}
%\usepackage{ascmac}
\usepackage{fancyhdr}
\usepackage{tcolorbox}
\usepackage{tikz}
%\usepackage{otf}
\usepackage{luatexja-otf}
\usepackage{luatexja-ruby}
\usepackage[top=20truemm,bottom=20truemm,left=15truemm,right=15truemm]{geometry}
\usepackage{mathrsfs}

\pagestyle{fancy}
	\fancyhead{}
	\fancyhead[RE]{円城塔研究}
	\fancyhead[LO]{講演「人新世で残らないもの」聴講レポート}
	\fancyhead[LE, RO]{\thepage}
	\fancyfoot{}
	\fancyfoot[LE, RO]{\footnotesize{The Journal of EnJoeToh Research, Vol.3, No.2, 2025} }

\theoremstyle{definition}
	\newtheorem{axi}{公理}
	\newtheorem{dfn}{定義}
	\newtheorem{thm}{定理}
	\newtheorem{hyp}{予想}
	\newtheorem{ths}{提唱}
	\newtheorem{prn}{原理}
    \newtheorem{clm}{主張}

\setcounter{page}{16}

\begin{document}

	~ %強制改行

	\begin{center}

		\Large{講演「人新世で残らないもの」聴講レポート}

		\vspace{3mm}

		\large{下村思游}

	\end{center}

	\vspace{3mm}

    本資料は,2024年2月22日に国立国会図書館関西館(京都府相楽郡精華町)で開催された円城塔の講演会「人新世で残らないもの」の聴講レポートである.当時記していたメモからの再構成であり,本資料は講演の内容と一言一句対応するわけではないことに注意されたい.

	\section{はじめに}

    普段はオフラインでの講演が主で,オフラインで80人以上集まったことはない.人が多すぎて当惑している\footnote{当日は180人程が来場した.}.

    \section{人新世とは}

    人新世とは,地層年代,地質年代のひとつとして提案されているもの.人の影響が顕に見える層のことをいう.プラスチックがやたらと出てくる,鉄がすごくいっぱいある,など.

    恐竜が絶滅した境のあたりに,イリジウムなどの特徴的な金属が多く入っている変な層がある.そんな感じで,なんか変な物質が多く残る時代に我々は生かされている.

    \section{日々消えている}

    とはいいつつ,日々色々なものが消え続けている.

    図書館は基本的に炎上するということになっている.アレクサンドリアの大図書館,『薔薇の名前』の図書館.

    ハーバード大の図書館に人皮装丁本がある\footnote{\url{https://ask.library.harvard.edu/faq/82429}}.倫理に反するということで剥がされ,中身と表紙の人皮が別に保管されている.元々の図書館資料は破壊されているわけで,一般的倫理と図書館としての倫理の狭間にある.ハーバードに行った際に,お前も人皮装丁本を見に来たのか,と言われたことがある.人皮装丁本を見にくる日本人が多いらしい.

    直近では,Internet archiveが著作権に関する裁判の二審で敗訴した\footnote{IAは,コロナ禍において,著作物をスキャンし,その電磁的複製物(スキャンデータ)を一度に複数貸し出すようにした.IA側は緊急時のフェアユースを主張していたが,一度に複数の人間に同時に貸し出すことを可能としていたことは著作権を不当に侵害していると判断された.\url{https://wired.jp/article/internet-archive-loses-hachette-books-case-appeal/}}.

    Internet archiveは非常に有名な電子図書館だが,実は私設団体である.

    \section{青空文庫}

    これも実は私設.

    検索に使いやすいのはシカゴ大のサイト\footnote{\url{https://artflsrv04.uchicago.edu/philologic4.7/aozora/}}かもしれない.これはシカゴ大のホイト・ロング\footnote{日本文学者.著書に『数の値打ち : グローバル情報化時代に日本文学を読む』(フィルムアート, 2023).}が作っているもの.

    データが増えるのは,最近では文豪系ゲームのファンによる貢献.各地の近代文学館はそれで大いに助かっているとのこと.国木田独歩とか,思いもよらない人のファンが増え,青空文庫も充実してきている.

    \section{基本的には残らない}

    人名辞典以上の情報が残る人はそうない.円城塔という名前も,人の名前としてまずない名前だが,名前くらいしか残らないのでは.

    シカゴのトリビューン・タワー\footnote{シカゴ・トリビューンの本社ビル.}の壁面には世界各地の石が埋められている.マニラの宝石も埋められているが,それは明らかにマニラで戦没した日本人のもので,日本語が彫られている.名前が残されていたとて,その人の情報は名前以上のものを得られないだろう.

    骨も,現代では火葬してしまうので,残らない.縄文人などの骨が残っているのは,土葬だから.したがって,現代では地球は墓だらけにすらならない.何も残らないので.残るとすれば,熱心なファンがいて何でも残そうとした場合とか.

    \section{残していないもの}

    書簡.

    書簡はメールになった.ただ,メールを残したいだろうか? 砕けすぎているので残したくない.前後の文脈がわからないと何もわからない.

    日記.

    日記はウェブ日記に移行したが,そこからさらにSNSに移行してわかりづらくなった.みんな,意外とちょっとした短文を書くことで満足するようになった.SNSは他者との交流のネットワークとして存在するので,どこまでが誰の記述なのかがわかりづらく,誰々の日記です,誰々から誰々への書簡です,というパッケージ性が急激に薄れた.どこまで収集すればいいのかが非常に不明瞭.

    \section{残していないもの2}

    原稿,草稿.

    円城塔は,ペンと紙では小説を書けない.キーボードとエディタがないと書けず,したがって草稿というものが残らない.芥川賞を獲ると,菊池寛記念館に芥川賞受賞作の草稿か原稿を2,3枚送って収蔵してもらうことになっている.
    が,円城塔は草稿も原稿も紙媒体では存在しない.なのででっち上げることになるのだが,手直しの跡がない草稿はいかにも不自然なので,手直しした風の小細工を加えた手書きの出来立てほやほやの草稿を送りつけることになる.

    \section{残せなかったもの}

    iPhone

    かつては,持ち主が死ぬと遺族であってもロックを外すことができなかった.当然苦情が相次いだので,ロックを外せるようになった.今では遺言を残したり,死亡時のアカウント相続人を指名することでロックを解除できるようになった.ではロックを外したいか,というのはまた別物.

    \section{(iPhoneの使用者死亡時のApple公式の文書)}

    \section{GitHub}

    残す方法として,Gitなどのバージョン管理システムがある.円城塔は,GitHubを利用して小説を公開している.誤字の指摘と修正もここで行なったことがある.

    一方で,出版社の人は興味をもたない.まあ,だろうね,と.

    \section{GitHubにアップされたワシントンD.C.の法律文}

    法律の変更履歴を追跡出来るように可視化している.上がっているし,誰でも見れるし,変更履歴も見れるが,xmlなので人に対する可読性が低い.可視化できるものは可視化しないか? 報道も,いつ何を報道したかをきちんと可視化出来るはず.

    \section{デジタル化}

    コミックの制作フローは完全に電子化された.そしてKADOKAWAのBook walkerに集約された.

    一方,書籍は電子化に失敗した.原稿をPCで作っても,校正や版組で必ず紙に印刷するため電子化できない.校正者が電子での校正を嫌うため,紙への印刷から脱却できない.人間サイドの問題では?

    印刷所だけが最終版面のデータを持っているため,出版社も著者も蚊帳の外.電子書籍のデータを収集したいなら,DNP,凸版あたりを攻めるといいだろう.しかし,印刷会社としては,書籍をするより壁紙を刷った方がはるかに儲かるので,もはや印刷会社にとって書籍印刷は趣味である.

    また,個人出版社が大規模化した.文学フリマ,コミケなどでの頒布が増えた.銀河帝国が崩壊して群雄割拠という状況が近づいてきている.

    \section{摂取の多様化}

    コンテンツを摂取する方法が多様化している.円城塔は,12-2月で恋愛もの,ラブコメ,異世界,悪役令嬢の漫画を570冊読んだ.これは紙の時代では考えられないことである.500冊も買ってしまったら,読む以前にどこに置けばいいのか.このようにして,コンテンツが電子化されたことである種のタガが外れ,以前では考えられなかった質と量のコンテンツ摂取が可能になった.

    \section{とはいえ読んでいた過去の人}

    フーコーは,『狂気の歴史』を執筆するため,仏国立公文書館にこもって“狂気”という語が出てくる本を全て読んだという.司馬遼太郎が『坂の上の雲』を執筆した際は,神田から“日露”という文字列が入った資料がすべて消え去ったという.

    まあ,彼らは猛烈めくりタイプの読書を行なったとされており,多分まともに読んでいないはず.電子化された資料は,確かにアクセスが容易いが,本当に総浚いする気なら,まだ図書館に分があるだろう\footnote{国立国会図書館デジタルコレクションの全文検索を使えば,さらに綿密な調査が行えることだろう.詳しくは下村思游「国立国会図書館デジタルコレクションの全文検索を用いた「奇想」および「奇想小説」の語誌の概観」(カモガワGブックス(5))を見よ.なお,これもまた図書館の優位性を強調するものであることに注意せよ}.

    \section{文字を読むのか?}

    人は小説を読まなくなったかもしれない.とはいえコンテンツを摂取する量は変わらないのでは.TikTok,配信,ソシャゲになっただけでは.

    \section{ゲームの実行系}

    有名ゲームなら,あったりする.任意バージョン実行は無理っぽい.Minecraftは任意バージョン実行可能だが,そもそもゲーム本体のソースコードがオープンなので例外.

    Pythonはかなり書き方が激しく変化する言語.10年前のコードとかは動きようもない.じゃあ100年後とかなら,もっとどうしようもないだろう.

    \section{我に返る}

    文人の書簡が残っているのも,選択が働いた結果である.漱石は庭で書簡を焼いていた.

    一方,子規は自分の手紙の写しをとるなど熱心に残していた.これによって漱石の書簡は周囲の人間が所持していたものが残ったために間接的に残ることになった.このようにして残された漱石の書簡の編集作業を見ていた芥川の書簡の書き方が,これ(漱石全集編集事業)を機に変化したことが知られている\footnote{この議論は近代作家旧蔵書研究会のシンポジウム「『これはペンです』か?」での話とまったく同一}.

    粗忽者が勝手にデジタルに移行して残らないと言っているだけなのでは.

    \section{残したいものは本当に実データか?}

    残すとして,本当に残したいのは表面的な実データなのだろうか.死んだ後に残ったメモ帳を本当に見たいものか.故人のメモを読み,この人はこんなことを秘めていたのかとか,こんなことを考えていたのか,知らなかった,という反応をするのはまあ当然である.なぜなら,その故人のメモは他人に公開することを想定していなかったのだから.残したくなかったものが残ってしまった例.

    文芸誌に載る対談記事は書き起こしではなく,編集されたものである.同様に,表に出るものはすべて自己演出,自己編集の産物である.細川家のように,代々日記を残して歴史や史観を残す人たちもいる.

    \section{失われるもの}

    人が死んだあとも,編集されてないデータが膨大に残る.その編集されていないデータを,人は本人そのものだと思い込む.

    \section{フェイク}

    データが増えると,故人を再現出来るような気がしてくる.一方で,クリエイターを機械学習によって模倣・復活させる試みはパロディになりがち.

    実際,ここ10-15年くらいの流行りでもある.機械学習で誰かしらを模倣するときは,その人が残した出力を学習するのではなく,その人が摂取したコンテンツを学習する方が正道だろう\footnote{この議論は,先掲のシンポジウム「『これはペンです』か?」の際にも,(円城塔を再現するなら)司馬遼太郎と田中芳樹を学習させたAIを混ぜる方が正気である,と主張していたことの流れを汲む.正確には,司馬遼太郎・田中芳樹に,英語圏の数理系の教科書の翻訳文体と法律文を混ぜるべき.具体的には,ファインマン物理学,田崎晴明,日本国憲法.}.故人のメモから復元した故人とされる何者かを観測して,これは故人らしい振る舞いをしている,などとするのはどうにも気狂いじみている.美空ひばりが歌いはじめたあたりで誰か止めるべきだったのでは.

    \section{バベルの図書館}

    真理が書かれた本のありかを示した本がある図書館.現代的な視点では,そこにあらゆるフェイクが含まれているというのが興味深い.普通に探すと,まず確実にフェイクに行き当たるだろう.

    \section{(無時間の思想)}

    (講演ではスキップされた)

    \section{野生化した図書館}

    図書館という巨大なデータが日々誕生している.違法アップロードサイトが事実上の電子図書館として機能している.海外の怪しい学習モデルは,こういった違法アップロードサイトのデータを元に作成されていることが多い.エヴァのキャラを描かせてみて実際に出力出来たなら,確実に“やってる”,という判定方法がある.

    あるいは無料かつ合法でコンテンツがあるコミック配信サイト,イラスト投稿サイト,小説投稿サイト.もしくは印刷所も事実上の電子図書館.昨今,機械学習は訓練用データの枯渇に苦しんでいる.ここらへんの大元のデータを持つ会社は日本にしかないので,日本のコンテンツに特化したAIを作るのに最も有利なのは日本である.

    大阪の公立図書館に朝早くから行くと,異世界小説を山積みにして読み漁っている老人をよく見かける.予習?と思ったりする.

    \section{書き方も変わる}

    小説新人賞の投稿者を見ていると,かつてとは様子が変わってきた.かつての投稿者は,いかにも文学志向で,俺を認めなければお前を殺すくらいの勢いがあった.今では,書く方が楽しく,読んでもらえるかどうかは気にしない,という人がかなり増えた.小説投稿サイト,zine文化の隆盛もこれによるところが大きいであろう.出来た,嬉しい,楽しい,だけで完結している世界.少しちいかわっぽい.

    そもそも売れないのでそれでいいのかも.10年で累計100万部売りました,はすごい.が,1年では1000万にしかならない.これなら,大企業に就職して真面目に勤めた方が良さそうである.

    \section{思想}

    データの残り方が変わる.

    そこから想像される人間の姿も変わる.

    ネット上の現実もまた変わる.

    何を残すか,何を残さないか,史観や思想といった主体的なものが問われている?

    \section{質疑応答}

    \begin{itemize}
        \item ハーバードの人皮の本の中身は?

        \vspace{2mm}

        大したものではない.ネクロノミコンとかではない.ただ,地元の知人曰く,ハーバードとかのあたりはマジでインスマスらしい.

        \vspace{2mm}

        寝ますか? とか聞かれるのでなんでもいいですよ.寝ます.笑いますかとも聞かれますが,笑ってますからわかるでしょ(笑)

        \vspace{2mm}

        \item (カルロ・ギンズブルグ『チーズとうじ虫』を念頭に)人間というものは元々残そうという意志はないのでは?

        \vspace{2mm}

        見方による.ギンズブルグは変な人.世界観の均一化の方が円城塔としては気になる.現代社会で『チーズとうじ虫』が出来ないのは気にかかるかもしれない.

        \vspace{2mm}

        \item コンテンツが残っていない.サ終したソシャゲのテキストの冊子体が発行された例がある.残らないから紙に戻すということはいかがか

        \vspace{2mm}

        デジタルは残らない.いっそ崖に刻むべき.伊藤計劃は,“くーまん”\footnote{“くーまん”とはこれのこと.\url{https://ja.wikipedia.org/wiki/%E3%81%8F%E3%83%BC%E3%81%BE%E3%82%93}}を人は覚えているかどうかということを気にしていた.

        \vspace{2mm}

        \item (漫画原作者から)漫画の消費スピードが早い,アイデア出しはどうしているのか

        \vspace{2mm}

        依頼次第.色と長さで決める.アニメだったら監督次第.

        \vspace{2mm}

        \item 異世界転生についてどう思うか

        \vspace{2mm}

        意外と緻密.現代と結びつく要素があった方が何かと便利.現代語や現代的価値観,科学技術を一切持ち込まないストロングスタイルのファンタジーを構成するのは物語的に困難.既に軌道に乗っているので,意外と死ななそう.ちなみに,アニメ制作会社は最近馬車のモデルを作らされすぎて,めちゃめちゃ上手くなってきている.

        \vspace{2mm}

        \item なぜ漫画の電子化はbook walkerに集約されたのか?

        \vspace{2mm}

        KADOKAWAの中の人曰く,出来る人が出版界にいなかったかららしい.出版社の動きが鈍く,KADOKAWAが電子化に力を入れ始めた頃にそのまま乗り合わせる形になって今に至る.

    \end{itemize}

\end{document}