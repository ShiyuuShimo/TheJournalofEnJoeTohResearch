\documentclass[10pt, a5paper, twoside]{ltjsarticle}

%\usepackage{okumacro}
\usepackage{enumitem}
\usepackage{amssymb}
\usepackage{amsmath}
\usepackage{amsfonts}
\usepackage{amsthm}
\usepackage{bm}
\usepackage{url}
\usepackage{here}
\usepackage{graphicx}
\usepackage{wrapfig}
\usepackage{makeidx}
\usepackage{braket}
%\usepackage{ascmac}
\usepackage{fancyhdr}
\usepackage{tcolorbox}
\usepackage{tikz}
%\usepackage{otf}
\usepackage{luatexja-otf}
\usepackage{luatexja-ruby}
\usepackage[top=20truemm,bottom=20truemm,left=15truemm,right=15truemm]{geometry}
\usepackage{mathrsfs}

\pagestyle{fancy}
	\fancyhead{}
	\fancyhead[RE]{円城塔研究}
	\fancyhead[LO]{京フェス2024円城塔賞部屋資料}
	\fancyhead[LE, RO]{\thepage}
	\fancyfoot{}
	\fancyfoot[LE, RO]{\footnotesize{The Journal of EnJoeToh Research, Vol.3, No.2, 2025} }

\theoremstyle{definition}
	\newtheorem{axi}{公理}
	\newtheorem{dfn}{定義}
	\newtheorem{thm}{定理}
	\newtheorem{hyp}{予想}
	\newtheorem{ths}{提唱}
	\newtheorem{prn}{原理}
    \newtheorem{clm}{主張}

%\setcounter{page}{22}

\begin{document}

	~ %強制改行

	\begin{center}

		\Large{京フェス2024円城塔賞部屋資料}

		\vspace{3mm}

		\large{下村思游}

	\end{center}

	\vspace{3mm}

	\section{「パリンプストあるいは重ね書きされた八つの物語」}

        \subsection{重要概念}

        \begin{itemize}
            \item ファインマン\insertkanjiskip--\insertkanjiskip ホイーラー吸収体理論
            \item 到達不能基数
            \item 文脈から切断され,投げられてあるテクスト
        \end{itemize}

        \subsection{課題}

        (下記の質問を受けて)到達不能基数が作品内でどのような役割を持つのか説明する.

        \subsection{質疑応答}

            到達不能基数に見立てることは,文芸的にどのように作用するのか?

            \begin{quotation}
                現時点ではあまり気にしてませんでした.到達不能基数に見立てること自体は誤っていないはずなので,文芸的作用を考察します.
            \end{quotation}

    \section{「ムーンシャイン」}

        \subsection{重要概念}

        \begin{itemize}
            \item ウルフラムの提唱
            \item 有限散在型単純群
            \item 定理証明支援系
            \item 共感覚
            \item “天才”表象
        \end{itemize}

        \subsection{読解}

            \subsubsection{科学哲学的解釈}

                古来より、人類は自然現象を各個の観測事実の単純な集積として具体的に認識してきた(素朴物理学).ニュートン、ライプニッツ、ガリレオらによって、各個の観測事実の背景にある物理法則が数学を用いて抽象的に体系化された(古典物理学).これによって、人類はなんでも際限なく知ることができるのではないかと過信した(カントのいうところの理性の暴走).

                しかし、群論で有限群を扱うなかで、人間が扱える大きさではない巨大な数を具体的に扱わなければ成らない事態が発生した.しかし、“少女”にとっては、このようなものは造作もない。具体を克服した抽象を、“少女”は具体によって飛び越え、さらにもう一段飛躍してわれわれの観測可能な世界から去っていく.

            \subsubsection{“少女”=カントール説}

                カントールとは、無限について初めて数学的な定義を正しく与えた数学者であり、素朴集合論(ラッセルのパラドックス発見以前の数学)を完成させて数学の基礎の構築に極めて多大な貢献をした19世紀最大の数学者である.カントールは、平たくいうと、他の数学者たちが有限ごときで四苦八苦しているなか、独りで勝手に無限を定義・理解し、無限のその先にある無数の無限の敷き詰められた砂浜で遊んでいた孤高の数学者である.

                作中の数学者たちが有限ごときで四苦八苦しているなか、一足跳びにこの世界から去っていった“少女”の姿は、このカントールの姿に酷似している.

            \subsubsection{“少女”の行動は自明説}

                そもそも“少女”は異なる公理系に立っており、この世界から去るのは自明である.

                “少女”は、巨大基数を具体的に把握している.巨大基数の存在はZFC公理系からは証明不可能.また、巨大基数のうち、強到達不能基数の存在公理はグロタンディーク宇宙の存在公理と同値.つまり、“少女”が依拠する“宇宙”は元から異なる(物理学的宇宙と数学的宇宙を混同していることに注意せよ).

        \subsection{課題}

            円城塔は本作を「現代的観点からすると不適切」とするが、これを別の視点から擁護できないか?

            数理科学における知的訓練の特殊性.午前中の橋本さんの講演\footnote{素粒子を専門とする理論物理学者,橋本幸士の講演.日常で気づいた不思議なことをとことん調べる,物理学者の奇癖を紹介した.}にもあったように、物理学者は日常のすべてを物理学の探究の対象であるとし、過剰な態度をとりがち.この過剰な探究心は、他者から奇妙であるように見えないか?

            奇妙に見え、実際に奇妙に振る舞うのであれば、その人物は奇妙なのでは?

            数学者が自然言語で数学的対象に“触れる“ことができるようになる過程は、言語学的な観点から研究の対象となっていたりする.

        \subsection{質疑応答}

            「現代的観点からすると不適切」というのはボーイミーツガールを指すのでは?
            \begin{quotation}
                それもそうだが,そもそも円城塔は,古典的な文芸的手法と数理的モチーフをうまく混ぜ合わせるのが非常にうまい作家だと思う.

                スパイものっぽいネタとかギャグとかは何度でも擦り倒す人でもあるのに,これだけ問題視するのはまだ自分の中で納得できていない.

                要するに,私は円城塔のボーイミーツガールに問題があるとは考えていないため,ボーイミーツガールではなく天才表象に絞って議論を試みたということ.
            \end{quotation}

    \section{「遍歴」}

        \subsection{重要概念}

            \begin{itemize}
                \item エルゴード仮説
                \item すべての山口の夢を見た山$\infty$
            \end{itemize}

        \subsection{課題}

            \begin{itemize}
                \item テイヤール・ド・シャルダンの思想をきちんと確認する.
                \item オープンソースの文化や思想をきちんと確認する.
                \item 円城塔の言うところの“単調性”を数学的に検証する.\\特に,“単調性”が互いに互いの夢を見合うようなサイクリックな構造を許すのかどうかを数学的に確認する.
            \end{itemize}

    \section{「ローラのオリジナル」}

            \subsection{重要概念}
                \begin{itemize}
                    \item 作られて“ある"ことを強いられている
                \end{itemize}

            \subsection{課題}
                参考文献が円城塔本人の文献に偏りすぎている.本作の読解が不十分であることを強く示唆している.

    \section*{注記}

        本資料は,2024年に開催された京都SFフェスティバル2024で行われた「円城塔『ムーンシャイン』を語る部屋」のために制作された2024年10月頃成立の原資料\footnote{\url{https://scrapbox.io/allreferenceengine/%E4%BA%AC%E3%83%95%E3%82%A7%E3%82%B92024%E5%86%86%E5%9F%8E%E5%A1%94%E9%83%A8%E5%B1%8B%E8%B3%87%E6%96%99}}を2025年11月にPDFとして再編集したものである.

        質疑応答については後日補ったが,それ以外の箇所は,特に断りのない限り,全て現資料が成立した2024年10月当時の記述をそのまま転記している.

\end{document}