\documentclass[10pt, a5paper, twoside]{jsarticle}

\usepackage{okumacro}
\usepackage{enumitem}
\usepackage{amssymb}
\usepackage{amsmath}{}
\usepackage{amsfonts}
\usepackage{amsthm}
\usepackage{bm}
\usepackage{url}
\usepackage{here}
\usepackage[dvipdfmx]{graphicx}
\usepackage{wrapfig}
\usepackage{makeidx}
\usepackage{braket}
\usepackage{fancyhdr}
\usepackage{tikz}
\usepackage[top=20truemm,bottom=20truemm,left=15truemm,right=15truemm]{geometry}

\pagestyle{fancy}
	\fancyhead{}
	\fancyhead[RE]{円城塔研究}
	\fancyhead[LO]{資料:コンメンタール「ローラのオリジナル」}
	\fancyhead[LE, RO]{\thepage}
	\fancyfoot{}
	\fancyfoot[LE, RO]{\footnotesize{The Journal of EnJoeToh Research, Vol.2, No.2, 2024} }

\theoremstyle{definition}
	\newtheorem{axi}{公理}
	\newtheorem{dfn}{定義}
	\newtheorem{thm}{定理}
	\newtheorem{hyp}{予想}
	\newtheorem{ths}{提唱}
	\newtheorem{prn}{原理}

\setcounter{page}{13}

\begin{document}

	~ %強制改行

	\begin{center}

		\Large{コンメンタール「ローラのオリジナル」}

		\vspace{3mm}

		\large{Commentary of short story \textit{The original of Lora}}

		\vspace{3mm}
		
		\large{下村思游}

	\end{center}

	\vspace{3mm}

	\section{解説}

		頁数・行数は『ムーンシャイン』紙版\cite{moonshine}に準拠した.

		\subsection{題名}

			ウラジーミル・ナボコフの遺作となった長編小説『ローラのオリジナル』\cite{laula}が元ネタ.138枚のカードに記された断片的なメモ群で構成されており,その順序や真の全要素は不明なままとなっている.

			円城塔は,あとがきで「ナボコフによる同名の遺作との関係はない」としているが,「節をカード風にはしてみた」ともあり,関係がまったくないとは言えない.

		\subsection{p.163, l.4}

			\begin{quote}
				
				「その革新性からではなく凡庸さから」

			\end{quote}

			“わたしのローラ”を生成する際に用いた,実在する機械学習の学習手法,LoRA (Low Rank Adaptation)\cite{lora}を指す.LoRAに関する作中の説明は,まったく正しい.機械学習に使われる他のアルゴリズムに比べ,LoRAは小説の中で説明出来る程度に簡単な手法である.

		\subsection{p.163, l.10-11}

			\begin{quote}
				
				「テキストの大半はテキストファイルではなく,PNGファイルのtEXtチャンク内にコードされていたものだが,〜」

			\end{quote}

			PNGファイルは画像ファイルだが,その内部にはテキストデータを保存するための領域を確保することが出来る.このような領域には3種類あり,tEXtチャンクはその中の1つである.tEXtチャンクにはLaten-1\footnote{より正確には,ISO/IEC 8859-1.いわゆるアルファベットを構成する文字を指す.}のみしか格納出来ないので,日本語の文字列をそのまま格納していたのであれば,実際にはiEXtチャンクが使われていたのだろう.

			また,直後のアルファチャンネルとは,PNGファイルで画像の一部を透明化したいときに用いるマスク画像を指定するデータ領域のこと.無論,テキストデータを保存するべき箇所ではない.

		\subsection{p.163, l.12-p.164, l.1}

			\begin{quote}
				
				「ここに集め,並べ直した文章は当座の収集によるものであり,配置の順序についても独自の判断によった.」

			\end{quote}

			本作は,先述の画像データ群に埋め込まれていたテキストデータを適当な配列で並べ直したものである.時間順序が乱れていると思われる作品として『Self-Reference ENGINE』があるほか,他者による(物理的に時間順序が乱れる)先行作として先述のナボコフ『ローラのオリジナル』やフリオ・コルタサル『石蹴り遊び』が挙げられる.

			本文テクストを選択・配列したのは“業者”であるが,ここで,“業者”が“わたし”として登場する00は“業者”が選択・配列したテクストに含まれるのかどうかという問題が生じる.00を基底現実であるとして含めないとする解釈もあるだろうが,ここでは,含まれるとして解釈したい\footnote{$\because$ 2桁の通し番号は00から振られるべきであるから.}.

		\subsection{p.164, l.19}

			\begin{quote}

				「当時の画像生成技術が完全に非可逆的〜」
				
			\end{quote}

			作中でいう“当時”の技術水準は,現実における2023年の技術水準にほとんど一致するものと考えてよい.現実における機械学習は不可逆的な学習であり,学習の結果得られたモデルから元の入力を復元することは不可能である.

			一方,本作では,学習に使う入力データにはハッシュによる“タグつけ”が義務づけられており,データの法的・倫理的健全性の担保が図られている.

		\subsection{p.166, l.18}

			\begin{quote}
				
				「探し出すことのできた断片は,全体のほんの一部であって,〜」

			\end{quote}

			本文テクストは,膨大な画像群である“わたしのローラ”の一部から回収された部分集合である.この部分集合からいかに驚くべき真実を見出そうとも,それ以外の巨大な集合の中には,その真実と矛盾する“真実”が無数に含まれているだろう.つまり,本作が膨大なテクストの部分集合であるという主張を素直に認めるのであれば,本作は本質的に読解不能である.

			しかしながら,そもそも小説とはすべての可能な読みを許容するメディアであり,真に正しい一意な読みなどないという立場もあることを考慮すれば,本作が本質的に読解不能なことは自明.

			また,無数の要素をもつ集合からいくつかの要素を選んでくることを許していることから,ただちに選択公理が連想されたが,これは考えすぎかもしれない.

		\subsection{p.167, l.9}

			\begin{quote}
				
				「わたしはわたしのローラを見かける.」

			\end{quote}

			“わたしのローラ”なのか,わたしの“ローラ”なのかが不明瞭である.これは「これはペンです」,「$\varnothing$」,「パラダイス行」などで見られる,冒頭で印象的だが意味不明な命題を示し,最後にその命題から導かれる信じがたい真実を明かす,という手法の変種か.つまり,あえて“ぎなた読み”的な,係り受けを判断しづらい読みにくい文章を提示することで,読者の注意をひく意図がある\footnote{ぎなた読みについては,「$\text{ATLAS}^3$」で用いられた実績がある.}ものと考えられる.

		\subsection{p.171, l.3}

			\begin{quote}

				「機械によってはじめて可能となった執着の技術」
				
			\end{quote}

			これとはやや異なるが,“機械によって可能となった芸術”および“ゴミの山から立ち上がる物語”は,かつてギブスンが「冬のマーケット」で描いたヴィジョンであった.定理証明支援系であるCoqによって可能になった数学が実在する\footnote{余談だが,定理証明支援系の登場は,人類による数学という営みの終焉を意味しない.なぜなら,計算機はある特定の数学的体系1つにしか依拠出来ないが,任意の数学的体系には,そこで証明出来ない真である算術文が存在するから(第一不完全性定理)である.つまり,いかに計算機を用いようとも,証明不可能な数学的命題が常に存在する.ある体系から証明不可能な命題を扱うためには,その命題を扱えるような適当な体系を計算機に設定しなければならない.これが出来るのは計算機ではなく人間だけなのだが,計算機に扱えない数学的対象を人間が扱える理由を,人類は未だ理解していない.第一不完全性定理の存在は,数学自身が永遠のフロンティアであり続けることを主張するのである.}ように,十分発達した機械による支援によって可能になる学問や技術がある.ごく一部に限られていた機械支援という手法が,一般にも適用されはじめたということかもしれない.

		\subsection{p.175, l.4-5}

			\begin{quote}

				「箱の中身が人であろうと機械であろうと本質的な違いはない」
				
			\end{quote}

			ジョン・サールが提唱した哲学における思考実験,中国語の部屋を指す.

		\subsection{p.175, l.19}

			\begin{quote}
				
				「画像も言葉も同様に,高次元のベクトルであるにすぎない.」

			\end{quote}

			機械学習によって得られたモデルは,高次元空間への写像である.画像も言葉もすべて高次元空間内のある一点として同じように表されるので,言葉による入力によって画像の出力を得られるのである.本文にもある通り,言語--言語間の変換と言語--画像間の変換は本質的に等価であり,基底を取り替えているに過ぎない.

		\subsection{p.190, l.14}

			\begin{quote}

				「最終的には不幸の意味を幸福に書き換えてしまうだろう.」
				
			\end{quote}

			ジョージ・オーウェル『1984年』に登場するニュースピークのこと.

		\subsection{p.190, l.15-16}

			\begin{quote}

				「見るものの方を透明に消してしまう薬やマントだ.」
				
			\end{quote}

			後述の,“ローラ”の生成規則と,“わたし”と“ローラ”の相互作用則を指す.

		\subsection{p.207, l.13}

			\begin{quote}

				「わたしのローラには親がいるに違いなく,〜」
				
			\end{quote}

			「墓の書」の主題である,創作中の登場人物の家族に関する議論からの派生と考えられる.「墓の書」において,円城塔は,後期クイーン的問題から派生して,登場人物の死や,登場人物の祖先について考察する.登場人物が作中で生きているとするならば,その登場人物はいつか必ず死ぬのであり,その墓所となるべき場所も登場人物が創作されるとともにその作中に当然用意されているはずである.また,登場人物が生きているとするならば,生まれた瞬間というものがあるはずであり,さらには親をはじめとした祖先が必ず存在していたはずである.

			登場人物を指す固有名詞は記述の束であるから,それを支える無数の記述が創作物に必要になる.たった1人の登場人物を仮定しただけで,創作者は無数の記述的基盤を与えなくてはならなくなる.

		\subsection{p.213, l.16-17}

			\begin{quote}

				「フロイト的錯誤であるところの児童ドア」
				
			\end{quote}

			“児童ドア”は原文ママ.もちろん“自動ドア”の誤りであるが,これをフロイト的錯誤であると仮定するならば,そのドアは“児童”しか入れないドアであると“わたし”が深層心理で判断していることにより,“その子”と“わたし”は分たれてしまったのだと解釈される.

		\subsection{p.214, l.13}

			\begin{quote}

				「トロフィー」
				
			\end{quote}

			犯罪者がその実績を誇示するため,被害者から奪うなどして獲得した物品を指す.写真や映像データなど,無形物である場合もある.

		\subsection{p.216, l.7}

			\begin{quote}

				「その場では,「検索」と「生成」がほとんど同義となる.」
				
			\end{quote}

			「Self-Reference ENGINE」の冒頭,「全ての可能な文字列.全ての小説はその中に含まれている.」\cite{sre}や,ボルヘス「バベルの図書館」が本質的に孕む問題.無数の要素から任意の対象1つだけを抽出する検索式を構成することは,その対象を構成することにほかならない.

		\subsection{p.217, l.12}

			\begin{quote}

				「そこには一本の腕が,体からではなく文脈から切り離されて転がるだけだ.」
				
			\end{quote}

			ハイデガーからの影響が強く見られる.

		\subsection{p.226, l.4-}

			\begin{quote}

				「わたしは,わたしのわたしを生成し,〜」
				
			\end{quote}

			“わたし”と“ローラ”の生成規則と相互作用則を明示している.

			第$n$階層の“わたし”と“ローラ”をそれぞれ$W_n, L_n$とすると,自然言語との対応は以下のように与えられる.

			\begin{equation*}
				\begin{cases}
					W_n \to (\text{わたしの})^n \text{わたし} \\ L_n \to (\text{わたしの})^n \text{ローラ} \\ L_0 \to \varnothing
				\end{cases}
			\end{equation*}

			無限鏡による片思いの連鎖,と喩えることが出来るかもしれない.

			また,指数が際限なく大きくどこまでも続いていく様からは,「決定論的自由意志改変攻撃について」\footnote{ハヤカワ文庫JA『異常論文』収録.}の階層型表記が連想される.この「決定論的〜」は,本人曰く“「それを見ることにより、二度目醒めることのできる夢が存在するが、自力で叶うものではない”\footnote{\url{https://twitter.com/EnJoeToh/status/1451160872559120384}}と要約可能」だが,“その詳細は今後10年の課題である”としている.本作は,その詳細について語った作品であると考えられる.

			なお,円城塔が自身のCosense\cite{nkm2}で“こないだ書いた奴”として挙げているのが本作である.

		\subsection{p.228, 10-}

			\begin{quote}

				「わたしは無数のわたしのローラの死を生成してきた.〜」
				
			\end{quote}

			これも「墓の書」の主題の発展系である.

		\subsection{p.234, l.19}

			\begin{quote}

				「その子と初老の人物は店のドアをくぐりぬけ,あなたの視界から消え去って行く.〜」
				
			\end{quote}

			このドアは,213頁に登場した“フロイト的錯誤”としてのドアだろう.“わたし”は“わたし”であり,“わたしのわたし”ではないので,“わたしのローラ”と交流をもつことが出来ない.この原理的な交流不可能性が,本作の救えなさを生み出している.

	\begin{thebibliography}{99}

		\bibitem{moonshine} 円城塔, 『ムーンシャイン』, 創元日本SF叢書, 東京創元社, 2024

		\bibitem{laula} ウラジーミル・ナボコフ, 『ローラのオリジナル』, 作品社, 2011

		\bibitem{lora} Edward Hu, Yelong Shen, Phillip Wallis, Zeyuan Allen-Zhu, Yuanzhi Li, Shean Wang, Lu Wang, Weizhu Chen, LoRA : low-rank adaptation of large language models, v2, 2021, \url{https://arxiv.org/abs/2106.09685}

		\bibitem{png} Portable Network Graphics (PNG) Specification, 3rd edition, W3C, 2024, \url{https://www.w3.org/TR/png-3/}

		\bibitem{haka} 円城塔, 「墓の書」, 新潮, \textbf{118}(9), 239-247, 2021

		\bibitem{sre} 円城塔, 『Self-Reference ENGINE』, ハヤカワ文庫JA, 早川書房, 2010

		\bibitem{nkm2} 円城塔, 中身について考える2(2023), 2023, \url{https://scrapbox.io/enjoetoh/%E4%B8%AD%E8%BA%AB%E3%81%AB%E3%81%A4%E3%81%84%E3%81%A6%E8%80%83%E3%81%88%E3%82%8B%EF%BC%92(2023)}

	\end{thebibliography}

\end{document}