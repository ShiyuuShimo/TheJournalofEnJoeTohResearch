\documentclass[10pt, a5paper, twoside]{jsarticle}

\usepackage{okumacro}
\usepackage{enumitem}
\usepackage{amssymb}
\usepackage{amsmath}{}
\usepackage{amsfonts}
\usepackage{amsthm}
\usepackage{bm}
\usepackage{url}
\usepackage{here}
\usepackage[dvipdfmx]{graphicx}
\usepackage{wrapfig}
\usepackage{makeidx}
\usepackage{braket}
\usepackage{fancyhdr}
\usepackage{tikz}
\usepackage[top=20truemm,bottom=20truemm,left=15truemm,right=15truemm]{geometry}

\pagestyle{fancy}
	\fancyhead{}
	\fancyhead[RE]{円城塔研究}
	\fancyhead[LO]{資料:コンメンタール「遍歴」}
	\fancyhead[LE, RO]{\thepage}
	\fancyfoot{}
	\fancyfoot[LE, RO]{\footnotesize{The Journal of EnJoeToh Research, Vol.2, No.2, 2024} }

\theoremstyle{definition}
	\newtheorem{axi}{公理}
	\newtheorem{dfn}{定義}
	\newtheorem{thm}{定理}
	\newtheorem{hyp}{予想}
	\newtheorem{ths}{提唱}
	\newtheorem{prn}{原理}

\begin{document}

	~ %強制改行

	\begin{center}

		\Large{コンメンタール「遍歴」}

		\vspace{3mm}

		\large{Commentary of short story \textit{Ergodic}}

		\vspace{3mm}
		
		\large{下村思游}

	\end{center}

	\vspace{3mm}

	\section{はじめに}

		本論は,円城塔の短編小説「遍歴」の註解である.本作は,その不可欠な構成要素として政治的な話題を含むが,諸事情により私はこのような議論に言及する権利を制限されているため,事実の指摘のみに留めた箇所や,意図的に言及を行ってない箇所が存在する.

		明らかに註解が欠落している箇所について逐一明言することもしないので,不足一切について他者による批判と補充を期待する.

	\section{解説}

		頁数・行数は『ムーンシャイン』紙版\cite{moonshine}に準拠した.

		\subsection{題名}

		遍歴,と書いてエルゴディック,と読む.これは物理学の一分野,統計力学におけるエルゴード仮説(ergodic theory)に由来する.

		エルゴード仮説は,物理量の長時間平均と,位相平均が一致することを主張する.統計力学が整備され始めた当初,この性質は統計力学における原理である等重率の原理を基礎付けするものと考えられていたが,現在では否定されている\cite{tsk}\footnote{余談だが,物理学の教科書には,脚注が異常に長かったり,本質的な情報がなぜか脚注にしか書かれていなかったりと,脚注に特色があるものが多い.田崎晴明の著書はその好例で,同書のp.7-9は半分以上,p.12に至っては8割以上が脚注となっている.物理学における脚注芸は円城塔『烏有此譚』の脚注芸に影響を与えている.}.

		エルゴードという言葉自体は,ギリシャ語のエルゴン(ergon, 仕事)とオドス(hodos/odos, 道・経路)を合わせた造語として,統計力学を創始したボルツマンによって作り出されたものである.

		\subsection{p.117, l.5-6}

			\begin{quote}

				「ネイピア数の肩にやたらと分数が乗った形のものが多く見られる.」

			\end{quote}

			統計力学には,ネイピア数$e$の指数関数として表記される概念が多く登場する,例えば,以下の分配関数の定義が代表的である.

			\begin{dfn}

				分配関数

				分配関数$Z(\beta)$は,以下のように定義される.

				$$Z(\beta) = \sum_{i} e^{- \beta E_i}$$
				
			\end{dfn}

			ここで登場した$e^{- \beta E_i}$はボルツマン因子と呼ばれるもので,有限温度の物理の本質を表現しており,統計力学に頻出する因子である\cite{tsk}.この描写はこのボルツマン因子を含む数式,すなわち統計力学に関する数式がそのカフェの内装に多く使われていることを示している描写であると考えられる.このことは,本作が統計力学に由来するものであることから,無理なく支持される.

		\subsection{p.118, l.3}

			\begin{quote}

				「椅子は規格化された二種類」

			\end{quote}

			2準位系を示唆するものと考えられる.一般的な統計力学の教科書では,簡単な2準位系モデルから考察をはじめる.つまり,統計力学の話をはじめるための作法を再現しているか.

		\subsection{p.118, l.10}

			\begin{quote}

				「$k T \log{2}$」
				
			\end{quote}

			2準位系における系の内部エネルギーを,エントロピーと温度の積で表したもの.ここで,エントロピーは以下の公式(ボルツマンの原理)で表される.

			\begin{dfn}
				
				ボルツマンの原理

				$k$をボルツマン定数,$W$を状態数としたとき,エントロピー$S$は以下のように表される.

				$$ S = k \log{W}$$

			\end{dfn}

			2準位系でエネルギーを$U = ST$の標識で表すような状況として代表的なものは,有名なマクスウェルの魔(マクスウェルの悪魔,マクスウェルのデーモンとも)である.

		\subsection{p.118, l.13}

			\begin{quote}
				
				「みな一定の距離を保って,室内に偏り少なく座っている.」

			\end{quote}

			普通の描写でもあるのだが,物理学における条件設定を同時に示唆するように配置されているか.この系は,十分時間発展した平衡状態にあるものと考えられる.

			ここで,“偏りなく”ではないことに注意したい.系に偏りがないことは,極めて強力かつ不自然な条件設定である.

		\subsection{p.120, l.1}

			\begin{quote}
				
				「山口浩一の人生」

			\end{quote}

			平凡な人生,と言っていいだろう.

		\subsection{p.124, l.1}

			\begin{quote}
				
				「何かを治すことはできるのだが,何かを治し続けることはできない.」

			\end{quote}

			國分功一郎『中動態の世界』を連想する.

		\subsection{p.125, l.14}

			\begin{quote}
				
				「オープンソース教団」

			\end{quote}

			本作の主題.

			%オープンソースについての説明

			%政治の話まで繋げるかどうか

		\subsection{p.131, l.11-12}

			\begin{quote}
				
				「客観的な世界が存在するかどうかなどということは,改めて考えるまでもないことかもしれないのだが〜」

			\end{quote}

			カントの影響が色濃く見られる.

		\subsection{p.132, l.10}

			\begin{quote}
				
				「自分は差別と,エルゴード教団が嫌いだ」

			\end{quote}

			有名な(そして極めて悪質な)ジョーク,“私は差別と黒人が嫌いだ”に由来する.

		\subsection{p.134, l.6-7}

			\begin{quote}
				
				「エルゴード教団も,教団としての政治的主張を持つが,〜」

			\end{quote}

			宗教勢力と政治権力は,古来から非常に結びやすいものである.ここでは,仏教勢力である一向宗の信仰を背景とした一向一揆が,既存の権力者を除き,事実上一向宗が支配者になった事例として,戦国期の加賀の事例を挙げるに留める.

		% \subsection{p.137, l.6}

		% 	\begin{quote}
				
		% 		「エルゴード教団は,ライセンス型信仰集団の中で,生まれ変わりを認める一派である.」

		% 	\end{quote}

		\subsection{p.141, l.14-15}

			\begin{quote}
				
				「テイヤール・ド・シャルダン型の\ruby{叡智圏}{ノウアスフィア}」

			\end{quote}

			フランス出身のカトリック司祭・古生物学者・地質学者,ピエール・テイヤール・ド・シャルダンが広めた概念\footnote{元々は,ソビエト(ウクライナ)の独創的な鉱物学者・地球化学者ウラジーミル・ヴェルナツキーが扱っていた.このヴェルナツキーを通じて,テイヤール・ド・シャルダンの思想はロシア宇宙主義ともよく比較される\cite{tds}.}のこと.

			テイヤール・ド・シャルダンの著書は当初禁書とされたが,死後その禁を解かれ,歴代の教皇\footnote{パウロ6世,ヨハネ・パウロ2世,ベネディクト16世(神学者ラッツィンガーとして),フランシスコ.}から肯定的な言及がなされているという\cite{tds}.

			また,叡智圏という概念は,哲学や神学への影響だけに留まらず,オープンソースの基本的な思想にも大きな影響を与えている\cite{noo}.

		\subsection{p.141, l.17}

			\begin{quote}
				
				「推進者たちの作業は,エンジニアたちの作業が常にそうであるように冗談に彩られていた」

			\end{quote}

			この通り.最も利用されているプログラミング言語のひとつ,Pythonとその使用者はまさにその典型例である.このPythonの名前の由来はモンティ・パイソンであり,直後に出てくるスパムメールはモンティ・パイソンのスケッチ「スパム」に由来する.

		\subsection{p.142, l.6}

			\begin{quote}
				
				「匿名化ソフトウェアあれと神は言われた.誰が世界を作ったのかをわからなくするためである」

			\end{quote}

			先述のPythonには,The Zen of Python\footnote{Zenとは,仏教における禅のことである.ハッカー文化はヒッピー文化との関係が深く,ヒッピー兼ハッカーが好んだ禅は(やや誤解されつつ)ハッカー文化に根強く受け継がれている.}と呼ばれるウィットとジョークに彩られた標語\cite{zen}がある.Pythonというプログラミング言語は,PythonのZenを暗黙の了解とすることを使用者に要請し,(Zenを理解する者にとっては)読みやすく書きやすい一義的な言語である.

			余談だが,Pythonの(条件文における)三項演算子\footnote{もしAならばBせよ,そうでないならCせよ,という手順を一行で記述する構文のこと.プログラミング言語によって実装はそれぞれ異なるが,Pythonでは``B if A else C''と実装されている.}の実装はかなり直感に反したものとなっており,現在の表記の採用を強行したPython開発者のグイド・ヴァン・ロッサムを馬鹿にした文言がZenに書かれている\footnote{``Although that way may not be obvious at first unless you're Dutch.''\cite{zen}(そのやり方は,オランダ人じゃないと一見分かりづらいかもしれない)ヴァン・ロッサムはオランダ人であり,英語の慣用表現にやたら頻出するオランダ人ネタを踏まえたもの.}.いかに開発者といえども,その仕事がクソであればクソであると噛み付くのがソフトウェアエンジニアの習性であり,本作においてもこの習性がよく反映されている.

		\subsection{p.142, l.9-10}

			\begin{quote}
				
				「「無知のヴェール」や「トロッコ問題」のような思考実験〜」

			\end{quote}

			いずれも有名な思考実験.

			%無知のヴェールは,アメリカの哲学者,ジョン・ロールズが提唱したもので,

			%トロッコ問題は,イギリスの哲学者,フィリッパ・フットが提唱したもので,

		\subsection{p.142, l.10-11}

			\begin{quote}
				
				「密教スタイルよりも禅スタイルが好まれた.」

			\end{quote}

			PythonにおけるZenをもちろん含むが,密室での一子相伝の密教ではなく,開かれた場での公開討論の禅を重視する\footnote{密教が一子相伝,禅が公開討論であるという事実は特にないが,そういうイメージ同士を対比して,“スタイル”という言葉で濁しつつ対比されている.}ということだろう.

			直前のソクラテスとソフィストの対置も,質問厨\footnote{ネットスラング由来.ここでは,自身の現状を説明することなく質問だけを行い,コミュニティに対してまったく貢献しない者のことを指す.ソフトウェアエンジニア界隈で最も嫌われる言動のひとつ.}のソクラテスではなく,詭弁とはいえ議論自体は試みているソフィストを評価するということで,議論する姿勢を尊ぶ教義を強調する記述だろう.

		\subsection{p.143, 3-4}

			\begin{quote}
				
				「あらゆるものは放っておけば最終的に,熱力学的平衡状態へ到達する.しかしその平衡が実現されていないのが生命であり社会である以上,〜」

			\end{quote}

			任意の系は,十分に時間発展した先で熱平衡状態に到達する.熱平衡状態にある系について,その部分系に注目したとき,任意の部分系同士は互いに等質的である.宇宙全体を系に取ったとしてもこれは成り立ち,これを特に宇宙の熱力学的死という.平たくいえば,宇宙というものは,ムラがあるときが生きた状態であり,ムラが完全になくなり等質的になってしまうと,そこに意味は見出せなくなり,死んでいる状態となる.

			“宇宙の死”という言葉は比喩的に感じられるが,これは物理学の正式な用語であり,逆に,生命とは,熱平衡状態に到達するまでの熱力学的な過程として記述されると考えられている.特に,物理学的な系として生命を考察したのが\cite{shr}である.

			また,社会についても,競覇的贈与という機構を仮定した場合,経済格差・社会格差や社会構造が自然に現れるという,熱平衡とは逆の時間発展の結果が得られることが知られている\cite{knk}\footnote{この論文は,円城塔の博士論文指導教官である金子邦彦と,その指導学生(つまり円城塔の弟弟子)である板尾健司によるものである.なお,当該論文は社会構造が成立する初期段階を論じたものであり,その後の社会の発展は研究の範疇外であることに注意されたい.}.

		\subsection{p.145, l.2}

			\begin{quote}
				
				「\ruby{不合理ゆえに我信ず}{クレド・キア・アブスルドウム}」

			\end{quote}

			2世紀から3世紀にかけて活動した最初期のラテン教父であるキリスト教神学者,テルトゥリアヌスの思想を象徴する標語.テルトゥリアヌスはキリスト教とそれ以外の思索を厳密に切り分け,対比させることによってキリスト教の固有性を強調しようとした.

			テルトゥリアヌスは,著書『キリストの肉について』において,「神の子は十字架につけられた.私はそのことを恥としない,なぜならそれは恥ずべきことであるから.神の子は死んだ.それはまったくもって信ずべきことである,なぜならそれは不合理なことであるから.彼は葬られ復活した.これは確実なことである,なぜなら不可能なことであるから.」\footnote{\cite{phil}からの孫引き.}と主張した.これらの記述が後世要約されたのが“不合理ゆえに我信ず”であり,テルトゥリアヌスはキリスト教のテクストのみに従うべきであるとし,キリスト教以外のギリシャ哲学などとの調和を模索したユスティノスの見解を徹底的に批判した.

		\subsection{p.145, l.7-}

			\begin{quote}
				
				「山一が山口から目覚めるのと同様に,山二もまた,山一から目覚めた.〜」

			\end{quote}

			数式で表現するとわかりやすい.$x$を体験,その体験内容を引数で表現すると,作中の山口,山一,山二は以下のように記述される\footnote{この記法は,どこか型なしラムダ計算のような気配を感じる.要するに,お行儀がよくなさそう.}.

			$$\text{山二} ( \text{山一} (\text{山口}_1, \text{山口}_2, \cdots) )$$

			%後で詳細を詰める
			%孤立した人間がいれば反証になるが,孤立する条件が,夢を見ないことではなく夢に見られないことなので,能動的に回避する手段がない,つまり孤立した人間の存在が極めて難しい

			山$N$と書いたとき,$N \to \infty$としたときの山$\infty$はすべての人生の記憶をもつことになる\footnote{作中にあるような“単調性”を仮定するならば,これはほとんど自明($\because$(グラフとして)木であるから).また,誰からも孤立した人間がいれば反証になるが,孤立する条件が,夢を見ないことではなく夢に見られないことなので,能動的に回避する手段がない.つまり,孤立した人間の存在が極めて難しい.すなわち,続く本文の記述は相当確からしい.ここで,孤立した人間がいないとすれば,自分自身の夢を見る人間が存在する.このようにして,われわれは自己言及構造を得るに至る.}.

			無限遠方において人類全体がたった1人の人間によって代表される,というSF的転回は,星新一「最後の地球人」を連想する.「最後の地球人」は未来の先に過去があった,という言葉で要約可能で,未来と過去が円環構造を成すことは,ロシア宇宙主義とエルゴード理論\footnote{ここでは,統計力学における誤謬的仮説であるエルゴード仮説ではなく,数学理論としてのエルゴード理論を指す.以下,同様に使い分けを行う.}を通じて本作とも関連する.また,SFの文脈から独立していたものと考えられていた円城塔の作品が,現代神話としてのSF,そしてセンス・オブ・ワンダーとしてのSFという古典SFの特徴に回帰していることは,物語構造のメタ解析としても興味深い.

		\subsection{p.148, l.3}

			\begin{quote}
				
				「山8」

			\end{quote}

			恐らく山八だが,半角数字の縦中横との兼ね合いでこういう表記になってしまった山$\infty$のことでもあるだろう\footnote{漫画『サムライ8』が同様のオチだったが,これが本作に関係しているかについては,そうとも言えるしそうでもないとも言える.}.

			本来,エルゴード教団1.0の死生観の転換は,山$\infty$の登場時,つまり無限遠方の未来でようやく発生するものなのだが,何らかの外的要因\footnote{ここでは縦中横.}で8が横に倒れ$\infty$になったことによって,想定外に早く発生してしまった,という感じか.

		\subsection{p.155. l.1}

			\begin{quote}
				
				「散らかった部屋」

			\end{quote}

			“散らかり具合”はエントロピーの世間向け\footnote{エントロピーを“散らかり具合”とか“乱雑さ”で説明づけるのは相当程度正しい.ただし,エントロピー増大則についての説明はかなり怪しいものも多く見られる.}の説明によく使われる.文学においても,トマス・ピンチョンが「エントロピー」で扱っている.

		\subsection{p.155, l.9-10}

			\begin{quote}
				
				「人間A,B,Cがいた場合,その人生を追体験する順番は,$3!=6$種類ありうる.」

			\end{quote}

			体験する順番についての考察は,過去に『AUTOMATICA』\footnote{ハヤカワ文庫JA『バナナ剝きには最適の日々』収録.}でも行っている.

		\subsection{p.156, l.4}

			\begin{quote}
				
				「その体験は$N!$通りありえ,これは明らかに$N$より大きい.」

			\end{quote}

			これはまったく正しい.$N! = N$となるのは$N = 1$だけで,$N \to \infty$でその差分は単調に増加する.

			なお,$N$が十分大きいとき,以下の近似公式が成り立つ.

			\begin{dfn}
				
				スターリングの近似

				$N \gg 1$であるとき,
				$$\log{N!} \simeq N \log{N} - N$$

			\end{dfn}

			この近似は,統計力学の計算で頻出する.

		\subsection{p.158, l.1-2}

			\begin{quote}
				
				「人類の総人口が〜」

			\end{quote}

			ユングの集合的無意識,相転移,人間原理を混ぜ合わせた,非科学的でありながらSF的に極めて魅力的な主張.

			伊藤計劃『ハーモニー』との関連性を考慮するべきか.

		\subsection{p.159, l.12-}

			\begin{quote}
				
				「このとき浩二の眺める喫茶店の風景を,〜」

			\end{quote}

			ラストシーン.ここでは,異なる時空を生きる,ほとんど同質なもの同士が,同じようなことを独白する構成をとっている.これは,円城塔が「なんかわからんが泣ける」としている,“2人の登場人物が、時空的に離れた場所で、それぞれモノローグする”というハメ手\cite{tki}の実践例であり,エモさを意図的に生成することを狙ったものと考えられる.

			本作の主題であるエルゴード仮説とは,長時間平均と位相平均が一致することを主張する仮説である.本作は,エルゴード仮説を仮定する宗教の周囲を巡りながら,様々なトピックを語り,長い時間をかけた読書の末に.物語の出発点であるカフェに戻ってきて同じような話をする.つまり,長時間平均と位相平均が一致する物語を展開しつつ,その内部で複数の人生を巡り,無限遠方の未来を語り,元の場所に回帰する物語となっている.

			そもそも,円城塔の初期作品には,最初に命題を示し,それがいかに正しいかを順序立てて説明し,最後にその命題から導かれる最も驚くべきだが自明な事実を提示する作品が多かった\footnote{例えば,「これはペンです」「パリンプセストあるいは重ね書きされた八つの物語」「パラダイス行」「$\varnothing$」など.}.本作では,最初と最後の結論と場所が全く一致しており,上記の展開とは異なるにもかかわらず,なぜだか感情を動かされるし,成長した気分になる.

			対象の周囲をぐるっと回ってきただけにもかかわらず,なぜだか成長した気がする\footnote{回転のある場での周回積分を連想させられる(一周回って集めて元に戻ってきただけなのに,なぜだか手元には数字が残っている,的な)が,これは飛躍かもしれない.}し,実際に読者は心を動かされる.なんらかの脆弱性を突かれているという可能性も指摘される.

	\newpage

	\begin{thebibliography}{99}

		\bibitem{moonshine} 円城塔, 『ムーンシャイン』, 創元日本SF叢書, 東京創元社, 2024

		\bibitem{tsk} 田崎晴明, 『統計力学Ⅰ』, 培風館, 2008

		\bibitem{bzr} Eric S. Raymond, 『伽藍とバザール』, 2000, \url{https://cruel.org/freeware/cathedral.pdf}

		\bibitem{tds} 村田奈生, テイヤール・ド・シャルダン研究史素描, \textit{東京大学宗教学年報}, (36), 195-214, 2019, \url{https://doi.org/10.15083/00078541}

		%\bibitem{kns} 小西広志, 回勅『ラウダート・シ』に見られるテイヤール・ド・シャルダンの影響, \textit{日本カトリック神学会誌}, (28), 175-191, 2017

		\bibitem{noo} Eric S. Raymond, 『ノウアスフィアの開墾』, 2005, \url{https://cruel.org/freeware/noosphere.pdf}

		\bibitem{zen} Tim Peters, The Zen of Python, \textit{Python Enhancement Proposals}, 2004, \url{https://peps.python.org/pep-0020/}

		\bibitem{shr} エルヴィン・シュレーディンガー, 『生命とは何か』, 岩波文庫, 岩波書店, 2008

		\bibitem{knk} Kenji Itao, Kunihiko Kaneko, Emergence of economic and social disparities through competitive gift-giving, \textit{PLOS Complex System}, \textbf{1}(1), 2024, \url{https://doi.org/10.1371/journal.pcsy.0000001}

		\bibitem{phil} 『新しく学ぶ西洋哲学史』, ミネルヴァ書房, 2022

		\bibitem{tki} 円城塔, 得意技を自覚する(2023), 2023, \url{https://scrapbox.io/enjoetoh/%E5%BE%97%E6%84%8F%E6%8A%80%E3%82%92%E8%87%AA%E8%A6%9A%E3%81%99%E3%82%8B(2023)}

	\end{thebibliography}

\end{document}