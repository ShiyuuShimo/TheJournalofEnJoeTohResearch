\documentclass[10pt, a5paper, twoside]{jsarticle}

\usepackage{okumacro}
\usepackage{enumitem}
\usepackage{amssymb}
\usepackage{amsmath}{}
\usepackage{amsfonts}
\usepackage{amsthm}
\usepackage{url}
\usepackage{here}
\usepackage[dvipdfmx]{graphicx}
\usepackage{wrapfig}
\usepackage{makeidx}
\usepackage{braket}
\usepackage{fancyhdr}
\usepackage[top=20truemm,bottom=20truemm,left=15truemm,right=15truemm]{geometry}

\pagenumbering{roman}

\pagestyle{fancy}
	\fancyhead{}
	\fancyhead[RE]{円城塔研究}
	%\fancyhead[LO]{目次}
	\fancyhead[LE, RO]{\thepage}
	\fancyfoot{}
	\fancyfoot[LE, RO]{\footnotesize{The Journal of EnJoeToh Research, Vol.2, No.2, 2024} }

\theoremstyle{definition}
	\newtheorem{dfn}{定義}
	\newtheorem{thm}{定理}

\begin{document}

	\begin{center}

	~ %強制改行

		\Large{目次}

		\vspace{5mm}

		\normalsize{特集 短編集『ムーンシャイン』刊行記念 中編}

	\end{center}

	\begin{itemize}

		\item 目次 \dotfill i

			\vspace{3mm}

		\item 【資料】コンメンタール「遍歴」 \dotfill 1

			\vspace{3mm}

		\item 【資料】コンメンタール「ローラのオリジナル」 \dotfill 13

			\vspace{3mm}

		\item 次号予告 \dotfill 21

			\vspace{3mm}

		\item 奥付 \dotfill 22

	\end{itemize}

	\vfill

	\flushleft{ 表紙の画像は国立国会図書館デジタルコレクションの田口友樹編『八丘椿』\footnote{\url{https://dl.ndl.go.jp/pid/2538818/1/25}}を加工して作成した.}

	\newpage

	\begin{center}

		\Large{\textit{memorandum}}

	\end{center}

\end{document}