\documentclass[10pt, a5paper, twoside]{jsarticle}

\usepackage{okumacro}
\usepackage{enumitem}
\usepackage{amssymb}
\usepackage{amsmath}{}
\usepackage{amsfonts}
\usepackage{amsthm}
\usepackage{url}
\usepackage{here}
\usepackage[dvipdfmx]{graphicx}
\usepackage{wrapfig}
\usepackage{makeidx}
\usepackage{braket}
\usepackage{fancyhdr}
\usepackage[top=20truemm,bottom=20truemm,left=15truemm,right=15truemm]{geometry}

\pagenumbering{roman}

\pagestyle{fancy}
	\fancyhead{}
	\fancyhead[RE]{円城塔研究}
	%\fancyhead[LO]{目次}
	\fancyhead[LE, RO]{\thepage}
	\fancyfoot{}
	\fancyfoot[LE, RO]{\footnotesize{The Journal of EnJoeToh Research, Vol.1, No.2, 2023} }

\theoremstyle{definition}
	\newtheorem{dfn}{定義}
	\newtheorem{thm}{定理}

\begin{document}

	\begin{center}

	~

		\Large{目次}

	\end{center}

	\begin{itemize}

		\item 目次 \dotfill i

			\vspace{3mm}

		\item 【資料】円城塔資料集 \dotfill 1

			\vspace{3mm}

		\item 【評論】“ハメ手を持つこと”実例解説1 : ムーンシャイン \dotfill 7

			\vspace{3mm}

		\item 【評論】“ハメ手を持つこと”実例解説2 : 良い夜を持っている \dotfill 10

			\vspace{3mm}

		\item 【資料】京フェス2022円城塔部屋資料 \dotfill 13

			\vspace{3mm}

		\item 【資料】京フェス2023円城塔部屋資料 \dotfill 22

			\vspace{3mm}

		\item 次号予告 \dotfill 29

			\vspace{3mm}

		\item 奥付 \dotfill 32

	\end{itemize}

	\vfill

	\flushleft{ 表紙の画像は国立国会図書館デジタルコレクションの『新版引札見本帖』\footnote{\url{https://dl.ndl.go.jp/pid/1682622}}を加工して作成した。}

	\newpage

	\begin{center}

		\Large{\textit{memorandum}}

	\end{center}

\end{document}