\documentclass[10pt, a5paper, twoside]{jsarticle}

\usepackage{okumacro}
\usepackage{enumitem}
\usepackage{amssymb}
\usepackage{amsmath}{}
\usepackage{amsfonts}
\usepackage{amsthm}
\usepackage{bm}
\usepackage{xurl}
\usepackage{here}
\usepackage[dvipdfmx]{graphicx}
\usepackage{wrapfig}
\usepackage{makeidx}
\usepackage{braket}
\usepackage{ascmac}
\usepackage{fancyhdr}
\usepackage[top=20truemm,bottom=20truemm,left=15truemm,right=15truemm]{geometry}

\pagestyle{fancy}
	\fancyhead{}
	\fancyhead[RE]{円城塔研究}
	\fancyhead[LO]{円城塔資料集}
	\fancyhead[LE, RO]{\thepage}
	\fancyfoot{}
	\fancyfoot[LE, RO]{\footnotesize{The Journal of EnJoeToh Research, Vol.1, No.2, 2023} }

\theoremstyle{definition}
	\newtheorem{dfn}{定義}
	\newtheorem{thm}{定理}

\setcounter{page}{1}

\begin{document}

	~ %強制改行

	\begin{center}

		\Large{円城塔資料集}

		\vspace{3mm}
		
		\large{下村思游}

	\end{center}

	\vspace{3mm}

	\section{1次資料(WEB)}

		円城塔の1次資料(本人の手による文章等)のうち、ウェブ上で容易に参照可能なものを集積する。

		\subsection{円城塔賞本人インタビュー(カクヨム)}

			\url{https://kakuyomu.jp/info/entry/webcon9_enjoetoh}

		\subsection{X}

			@EnJoeToh

			\url{https://twitter.com/EnJoeToh}

			2023年4月以降極端に更新頻度が落ち、宣伝・広報が主となっていることに注意。

		\subsection{mastodon}

			@enjoetoh@mstdn.jp
	
			\url{https://mstdn.jp/@enjoetoh}

			2023年10月現在はこちらがメインのSNS。

		\subsection{カクヨム}

			えんしろ/@enjoetoh

			\url{https://kakuyomu.jp/users/enjoetoh}

		\subsection{GitHub}

			EnJoeToh

			\url{https://github.com/EnJoeToh}

			コード置き場。『プロローグ』のほか、カクヨム掲載の10000字小説$\times 4$『花とスキャナー』、2000字小説$\times 100$『通信記録保管所』のプレーンテキストが置かれている。『花とスキャナー』『通信記録保管所』はCC BY-NC 4.0なので、機械学習や形態素解析に用いることが出来る。

		\subsection{Scrapbox}

			\url{https://scrapbox.io/enjoetoh/}

			ゲンロンSF講座のためのメモなどが散乱している。小説の具体的な技術や手法が細かく記されているほか、ゲンロンSF講座での講評が掲載されているので、円城塔賞を狙うならば要確認。

			~

			\url{https://scrapbox.io/mojika/}

			『文字渦』自作解説。量こそ少ないが、作者本人による自作解説は最重要資料である。

		\subsection{ブログ}

			Self-Reference ENGINE

			\url{http://self-reference.engine.sub.jp/}

			円城塔のブログ。2012年11月で更新は終了している。

		\subsection{Tumblr}

			EnJoeToh

			\url{https://enjoetoh.tumblr.com/}

			更新は2020年10月で終了している。

		\subsection{読書メーター}

			EnJoeToh

			\url{https://bookmeter.com/users/6997}

			円城塔の読んだ本が(記録される限りで)全て確認出来る。円城塔の好みを把握し、好みを的確に攻略するべき。


		\subsection{円城塔の帯文(Togetter)}

			\url{https://togetter.com/li/1245748}

			6dts氏(twi: @6dts)による円城塔による帯文のまとめ。	

	\section{2次資料(円城塔研究:質的アプローチ)}

		円城塔の2次資料(本人以外の手による文章等)のうち、質的アプローチを行なっているものを集積する。

		\subsection{円城塔全作リスト}
		
			\url{https://t.co/UT2dSJjiNT}

			6dts氏(twi: @6dts)による円城塔全作品リスト。小説だけでなく、書評やインタビューも網羅。円城塔作品への書評も把握可能な範囲で網羅。もうこれだけでいいんじゃないかな。リンク先に遷移すると即座にファイルがダウンロードされることに注意。


		\subsection{円城塔研究}
		
			\url{https://sites.google.com/view/kashiwaya/%E6%9F%8F%E5%B1%8B%E6%9B%B8%E5%BA%97%E9%9B%BB%E5%AD%90%E9%A0%92%E5%B8%83#h.fo94szfdwx3f}

			円城塔に関する研究を掲載した下村による同人誌。PDF形式で無料頒布中。

		\subsection{All-Reference ENGINE}

			\url{https://scrapbox.io/allreferenceengine/}

			Scrapboxで公開されている下村による研究ノート。内容は未整理。

		\subsection{加藤夢三『合理的なものの詩学 : 近現代日本文学と理論物理学の邂逅』(ひつじ書房, 2019)}

			「補遺ii」において、円城塔『Self-Reference ENGINE』を多元世界の中での“私”性に注目して論じる。(なお、私は自己言及そのものを厳密に考察すべきであると考えている)

		\subsection{加藤夢三『並行世界の存在論 : 現代日本文学への招待』(ひつじ書房, 2022)}

			“並行世界もの”の可能性を論じる一冊。第八章は丸々『Self-Reference ENGINE』の検討に割かれている。

	\section{2次資料(円城塔研究:量的アプローチ)}

		円城塔の2次資料(本人以外の手による文章等)のうち、量的アプローチを行なっているものを集積する。

		\subsection{円城塔『通信記録保管所』および『花とスキャナー』の文字頻度解析}

			\url{https://shiyuu-sf.hatenablog.com/entry/2023/10/15/000706}

			下村による、円城塔の短篇小説集『通信記録保管所』と『花とスキャナー』の文字頻度解析の結果を荒くまとめたもの。

		\subsection{夏目漱石『こころ』の文字頻度解析}

			\url{https://shiyuu-sf.hatenablog.com/entry/2023/10/17/005624}

			下村による、夏目漱石の長篇小説『こころ』の文字頻度解析の結果を荒くまとめたもの。

		\subsection{円城塔『通信記録保管所』からの円城塔的文字列生成}

			\url{https://x.com/Akimasa_K/status/1720758900394442879}

			Akimasa Kitajima氏による、円城塔の生成した文字列から円城塔的な文字列を再構成する試み。円城塔本人曰く、「デフォルメ的強調が見えます」\footnote{\url{https://x.com/EnJoeToh/status/1720766477702430885}}とのこと。

		\subsection{円城塔を近似する}

			\url{https://colab.research.google.com/drive/1oXxBIYJvvUYsVZP6WYAUCb3QK09zTJtO}

			Akimasa Kitajima氏による、高精度円城塔シミュレータ。円城塔のテクストを力学系として解析する試みについても触れられている。

	\section{研究用ツール等}

		円城塔研究に有用なツール等を集積する。

		\subsection{NDLOCR}

			\url{https://github.com/ndl-lab/ndlocr_cli}

			国立国会図書館がCC BY 4.0で公開しているOCRソフトウェア。実際に動かそうとすると大変ハードルが高いので、次に示すツールを利用するのがおすすめ。

		\subsection{NDLOCR\_v2の実行例.ipynb}

			\url{https://colab.research.google.com/github/nakamura196/000_tools/blob/main/NDLOCR_v2%E3%81%AE%E5%AE%9F%E8%A1%8C%E4%BE%8B.ipynb}

			東京大学史料編纂所の中村覚氏(twi: @satoru196)による、上記のNDLOCRをgoogle colabで実行させるお試しツール。計算資源が限られているため、実行にやや時間はかかる(夏目漱石『こころ』の全文OCRに15分弱要した)ものの、無料かつ手軽に扱えるのは大変大きな魅力。

\end{document}