\documentclass[10pt, a5paper, twoside]{jsarticle}

\usepackage{okumacro}
\usepackage{enumitem}
\usepackage{amssymb}
\usepackage{amsmath}{}
\usepackage{amsfonts}
\usepackage{amsthm}
\usepackage{bm}
\usepackage{url}
\usepackage{here}
\usepackage[dvipdfmx]{graphicx}
\usepackage{wrapfig}
\usepackage{makeidx}
\usepackage{braket}
\usepackage{ascmac}
\usepackage{fancyhdr}
\usepackage[top=20truemm,bottom=20truemm,left=15truemm,right=15truemm]{geometry}

\pagestyle{fancy}
	\fancyhead{}
	\fancyhead[RE]{円城塔研究}
	\fancyhead[LO]{“ハメ手を持つこと”実例解説1 : ムーンシャイン}
	\fancyhead[LE, RO]{\thepage}
	\fancyfoot{}
	\fancyfoot[LE, RO]{\footnotesize{The Journal of EnJoeToh Research, Vol.1, No.2, 2023} }

\theoremstyle{definition}
	\newtheorem{dfn}{定義}
	\newtheorem{thm}{定理}

\setcounter{page}{7}

\begin{document}

	~ %強制改行

	\begin{center}

		\Large{“ハメ手を持つこと”実例解説1 : ムーンシャイン}

		\vspace{3mm}

		\large{Technique in Moonshine}

		\vspace{3mm}
		
		\large{下村思游}

	\end{center}

	\vspace{3mm}

		円城塔は、自身のScrapboxにおいて、“ハメ手を持つこと”について紹介している\footnote{\url{https://scrapbox.io/enjoetoh/%E3%83%8F%E3%83%A1%E6%89%8B%E3%82%92%E6%8C%81%E3%81%A4%E3%81%93%E3%81%A8}}。

		本論では、円城塔作品において、実際に“ハメ手”が機能している例を示す。これによって、円城塔作品が実際に“ハメ手”を用いて創作されたことを確認し、この手法が有効であることを示す。

		“ハメ手”として、例1「2人の登場人物が、時空的に離れた場所で、それぞれモノローグする」、例3「理詰めで押し続けるように見せて、限界に達したところで破綻させる」が挙げられている。実例として、短篇「ムーンシャイン」(創元SF文庫『超弦領域』(東京創元社, 2009, 絶版)またはハヤカワ文庫JA『日本SFの臨界点〔恋愛篇〕』(早川書房, 2020)収録)においてこれらが用いられていることを確認する。以下、『日本SFの臨界点〔恋愛篇〕』紙版を用いる。

		「ムーンシャイン」は、語り手である“僕”と人智を越えた超常的計算能力をもつ“私”の2人の一人称視点からの物語が交互に繰り返される構造となっている。“僕”は常識的な範疇からやや外れた優れた数学的能力を有している\footnote{冒頭の1729に関するエピソードは“インドの魔術師”ラマヌジャンのエピソードである。}ものの、その語りは一応なりとも理解可能である。一方、“私”の超常的計算能力およびそれを支える認知能力は真に異質であり、その語りで描かれる世界の描像は明らかに通常の世界認識とは異なる(\cite{moon} p.302)。この差異は、2人の数学的能力が“僕”は抽象的なものであり、“私”は具体的なものであることによってもたらされている、という解釈が説得的である。実際、“私”の数学的能力が極端に巨大な数学的対象の具体的な把握能力によることが文中の描写から示唆される(\cite{moon} p.302)。

		現代自然科学は、具体的対象の観察から離れ、抽象的対象への操作によって一般的な理論を導出し、近代化を達成した。数学自身もその例外ではなく、ラッセルやヒルベルトに代表される徹底的な抽象化とゲーデルやフォン・ノイマンによる論理的地盤整備によって近代化が行われた。一方で、“私”は異常に強大な認知能力によって具体的対象を具体的なままに認識することが可能である。本作はそのような具体的な数学能力に優れた“私”が数学を押し進め、限界に達したところで抽象へと上がっていく様を描く(例3の構成的実装)。

		また、本作は、“私”、の上で走る17と19というインターフェース、と交流する“僕”、というように真には理解不可能であろう異質な巨大知性を多重の仲介を通じて読み解く構造が採用されている。根気強く解釈を積み重ねていけば“私”の語りも十分に理解可能であるが、その過程では数学の専門的な知見が要求されるため容易ではない。すなわち、読者は、“僕”から17・19を経て“私”を理解しようとする過程で、ほとんど必ず理解の限界に至る。

		読者は小説の内容が理解出来ないことにより抑圧される。この抑圧が張り詰めていった最後に、“私”の語りは急激に視覚的になり、美しい情景が描写される。この視覚的描写は読者にとって非常に理解しやすいものであり、それまでの数理科学的に難解な描写による抑圧は解放される(例3の演出的実装)。同時に、“私”はそれまでの具体的な把握能力を一切捨て去り、抽象的な数学へと踏み出していく。物語終盤の2人のモノローグは、抽象的な数学の世界に留まらざるを得ない“僕”と、抽象的な数学を捨てた具体的な数学をも捨て去って、さらに1階上の抽象的な数学の世界へと歩み出す“私”との間で交わされている(例1の実装)\footnote{ここで、“僕”のいる抽象的な数学の世界と、“私”のいる抽象的な数学の世界は一致しないことに注意。}。

		円城塔が「ムーンシャイン」において、構成面および演出面の両面で例3を実装するとともに、例1をも実装する様子を確認した。「ムーンシャイン」は極めて難解な数理科学的知見を含む小説であるが、それが大変感動的な作品として受容されているのは、これらの手法が効果的に実装されていることによると考えられる。		

	\begin{thebibliography}{99}

		\bibitem{hamete} 円城塔, ハメ手を持つこと, \url{https://scrapbox.io/enjoetoh/%E3%83%8F%E3%83%A1%E6%89%8B%E3%82%92%E6%8C%81%E3%81%A4%E3%81%93%E3%81%A8}, 2018

		\bibitem{moon} 円城塔, 「ムーンシャイン」, 『日本SFの臨界点〔恋愛篇〕』, ハヤカワ文庫JA, 早川書房, 287-336, 2020

		\bibitem{moon_s_are} 下村思游, 「ムーンシャイン」, All-Reference ENGINE, \url{https://scrapbox.io/allreferenceengine/%E3%80%8C%E3%83%A0%E3%83%BC%E3%83%B3%E3%82%B7%E3%83%A3%E3%82%A4%E3%83%B3%E3%80%8D}, 2020

		\bibitem{moon_s_blog} 下村思游, 「ムーンシャイン」について, SF游歩道, \url{https://shiyuu-sf.hatenablog.com/entry/moonshine}, 2020

	\end{thebibliography}

\end{document}