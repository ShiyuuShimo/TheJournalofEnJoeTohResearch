\documentclass[10pt, a5paper, twoside]{jsarticle}

\usepackage{okumacro}
\usepackage{enumitem}
\usepackage{amssymb}
\usepackage{amsmath}{}
\usepackage{amsfonts}
\usepackage{amsthm}
\usepackage{bm}
\usepackage{url}
\usepackage{here}
\usepackage[dvipdfmx]{graphicx}
\usepackage{wrapfig}
\usepackage{makeidx}
\usepackage{braket}
\usepackage{ascmac}
\usepackage{fancyhdr}
\usepackage[top=20truemm,bottom=20truemm,left=15truemm,right=15truemm]{geometry}

\pagestyle{fancy}
	\fancyhead{}
	\fancyhead[RE]{円城塔研究}
	\fancyhead[LO]{“ハメ手を持つこと”実例解説2 : 良い夜を持っている}
	\fancyhead[LE, RO]{\thepage}
	\fancyfoot{}
	\fancyfoot[LE, RO]{\footnotesize{The Journal of EnJoeToh Research, Vol.1, No.2, 2023} }

\theoremstyle{definition}
	\newtheorem{dfn}{定義}
	\newtheorem{thm}{定理}

\setcounter{page}{10}

\begin{document}

	~ %強制改行

	\begin{center}

		\Large{“ハメ手を持つこと”実例解説2 : 良い夜を持っている}

		\vspace{3mm}

		\large{Technique in \textit{Have a good night}}

		\vspace{3mm}
		
		\large{下村思游}

	\end{center}

	\vspace{3mm}

		円城塔は、自身のScrapboxにおいて、“ハメ手を持つこと”について紹介している\footnote{\url{https://scrapbox.io/enjoetoh/%E3%83%8F%E3%83%A1%E6%89%8B%E3%82%92%E6%8C%81%E3%81%A4%E3%81%93%E3%81%A8}}。

		本論では、円城塔作品において、実際に“ハメ手”が機能している例を示す。これによって、円城塔作品が実際に“ハメ手”を用いて創作されたことを確認し、この手法が有効であることを示す。

		“ハメ手”として、例2「特定のジャンルものとしてはじめ、ある地点でジャンル自体をひっくり返す」が挙げられている。実例として、短篇「良い夜を持っている」(新潮文庫『これはペンです』(新潮社, 2014)収録)においてこれが用いられていることを確認する。以下、『これはペンです』新潮文庫紙版を用いる。

		「良い夜を持っている」は、人智を超えた完全記憶を持つ“父”を、その足跡を辿ることで理解しようとする息子\footnote{「これはペンです」の描写から、男と断定していいだろう。}の“わたし”の視点から描く物語である。“父と子”というテーマは、志賀直哉『暗夜行路』や菊池寛『父帰る』、あるいは直裁にイワン・ツルゲーネフ『父と子』、そして近年では小川哲の諸作品が挙げられるように、古今東西の文学で扱われ続けてきたものである。本作は、正体の掴めない父の姿を、その死後に足跡を辿ることで理解していこうという古典的かつ文学的な題材を、完全記憶と物理学的発想によってSFとしてひっくり返し、さらに最終盤でSFから文学へひっくり返す。これを実際に見てみる。

		“父”は、その強力すぎる記憶力によって、物事を完全に記憶している。“父”はその完全記憶によって、もう一つの街を記憶の中に構成することが可能である。この“記憶の街”は十分複雑なので、“記憶の街”自体が自分自身を構成可能であり、しかもその“記憶の街”が街として完全に機能するが故に、“記憶の街”が街として活動する(野良猫が去る、烏が電線に停まるなど)と、元の記憶は乱れてしまう\footnote{記憶と“記憶の街”が1対1に対応するため。}(文学からSFへの飛躍)。

		完全記憶というアイデアはホルヘ・ルイス・ボルヘスの短篇「記憶の人、フネス」でも登場しているが、本作では完全記憶を物理学的に考察することで、十分強力な記憶は現実と区別出来ないことを導く。現実において“父”は死亡したが、“父”の構成した“記憶の街”は、そこに一度作り出されてしまったが故にそこにあり続ける\footnote{『SRE』におけるSREの存在理由に酷似している。}。そして“記憶の街”で、“父”はもう一度“母”と出会う(SFから文学への収束)。ここでは現実と記憶は既に区別がつかず、その出会いは紛れもなく現実である。ここで、現実と記憶が渾然一体となっていることは、作中の描写\footnote{“姪”の握っていた赤いビー玉は、“父”が記憶対象を示すために用いていた赤いポインタである, \cite{hgn} p.216}から明らか。

		そもそも、本作は、以下のような叙情的な書き出しから始まっていた。

		“目覚めると、今日もわたしだ。/そんな当然すぎる事柄が本当は何を意味しているのか、最近ようやく少し理解できるようになったと感じる。”

		これは文学的な表現でありながら、実はグレッグ・イーガンのSF短篇「貸金庫」のパロディである。文学をやるようでいて、最初からSFになることは予定調和だったとも言える。文学的テーマがSFで膨らまされていくこと、そしてSF的発想が文学的大団円で収束すること、この両面でジャンルをひっくり返すことに成功していると言える。掴みどころのない(しかも所々に数理的に難解な部分のある)法螺話が、最後に突如として家族の再会のシーンという(日常的な感覚では)分かりやすいシーンに収束する落差が、感動的なシーンとして作用するのだろう\footnote{これは前回見た例3「理詰めで押し続けるように見せて、限界に達したところで破綻させる」にも通じる。}。

		ここで、生まれながらに人の感情に対して鈍感であったり、記憶が上手く順序立てされなかったりする認知特性をもつ者\cite{smz}や、訓練によって数理的な思考を日常の中で暴走させ続けている者\cite{kmr}がこの社会に少数ながらも実在することについて、十分に注意したい。

		異常な世界を構成する、あるいは鑑賞することで楽しむのは大変結構なことである。しかし、物語を扱う者として以前に、他者と共に生きなければならない者として、その“異常な世界”なるものの中で平常に生きている人間が存在するかもしれないことを十分理解しなければならない\cite{nkm}。

	\begin{thebibliography}{99}

		\bibitem{hmt} 円城塔, ハメ手を持つこと, \url{https://scrapbox.io/enjoetoh/%E3%83%8F%E3%83%A1%E6%89%8B%E3%82%92%E6%8C%81%E3%81%A4%E3%81%93%E3%81%A8}, 2018

		\bibitem{hgn} 円城塔, 良い夜を持っている, これはペンです, 新潮文庫, 新潮社, 119-216, 2014

		\bibitem{hgn_s_are} 下村思游, 「良い夜を持っている」, All-Reference ENGINE, \url{https://scrapbox.io/allreferenceengine/%E3%80%8C%E8%89%AF%E3%81%84%E5%A4%9C%E3%82%92%E6%8C%81%E3%81%A3%E3%81%A6%E3%81%84%E3%82%8B%E3%80%8D}, 2019

		\bibitem{smz} 清水光恵, スクラップ置き場と私, 発達障害の精神病理Ⅱ, 星和書店, 108-124, 2020

		\bibitem{kmr} 木村達雄, 数学は体力だ!, 筑波フォーラム, 45, 104-107, 1996

		\bibitem{nkm} 円城塔, 中身について考える2(2023), \url{https://scrapbox.io/enjoetoh/%E4%B8%AD%E8%BA%AB%E3%81%AB%E3%81%A4%E3%81%84%E3%81%A6%E8%80%83%E3%81%88%E3%82%8B%EF%BC%92(2023)}, 2023

	\end{thebibliography}

\end{document}