\documentclass[10pt, a5paper, twoside]{jsarticle}

\usepackage{okumacro}
\usepackage{enumitem}
\usepackage{amssymb}
\usepackage{amsmath}{}
\usepackage{amsthm}
\usepackage{url}
\usepackage{here}
\usepackage[dvipdfmx]{graphicx}
\usepackage{wrapfig}
\usepackage{makeidx}
\usepackage{braket}
\usepackage{fancyhdr}
\usepackage[top=20truemm,bottom=20truemm,left=15truemm,right=15truemm]{geometry}

\pagestyle{fancy}
	\fancyhead{}
	\fancyhead[RE]{円城塔研究}
	\fancyhead[LO]{京フェス2022円城塔部屋資料}
	\fancyhead[LE, RO]{\thepage}
	\fancyfoot{}
	\fancyfoot[LE, RO]{\footnotesize{The Journal of EnJorToh Research, Vol.1, No.2, 2023} }

\theoremstyle{definition}
\newtheorem{dfn}{定義}
\newtheorem{thm}{定理}

\setcounter{page}{13}

\begin{document}

	{\Large  } %強制改行

	\begin{center}

		\Large{京フェス2022円城塔部屋資料}

		\vspace{3mm}
		
		\large{下村思游}

	\end{center}

	\vspace{3mm}

	\section{はじめに}

		本稿は京都SFフェスティバル2022合宿企画の「円城塔作品を語る部屋ふたたび」の資料である。

		今年は、円城塔作品の数理科学的概念の紹介に焦点を当てていた去年の京フェス2021円城塔部屋資料とは毛色を変えてみた。

	\section{この企画の目的}

		\begin{itemize}

			\item 円城塔が取り組む文学テクストへの数理的アプローチ(数理文学)について紹介する。

			\item 円城塔の数理文学の基礎となる統計力学・熱力学などについて平易に解説する。

			\item 実際に簡単な数理モデルを扱い、円城塔の関心についてちょっと理解する。

			\item 深夜の円城塔トークを十二分に楽しむ。

		\end{itemize}

	\section{数理文学}

		\begin{dfn}

			数理文学とは、数理的に文学テクストを構成する、文学テクストを数理的に解析するなど、文学テクストを数理的に扱う営みである。

		\end{dfn}

		数理文学の具体的手法

		\begin{itemize}
			
			\item 既存のテクストを数理的に解析する
			
			\item 数理的アプローチによってテクストを構成する
		
		\end{itemize}

		去年の京フェスの話は前者で、今年は後者に取り組んでみようというわけ。

		なお、一般の自然言語からなるテクストを数理的に解析するのは通常困難なため、数理文学とは数理的に構成されたテクストを数理的に解析する営みのみに限られてしまう。扱う範囲が極端に狭くなってしまったとはいえ、数理文学は無力ではない。数理文学がもたらす文学の新たな可能性については「十二面体関係」の解説\footnote{\url{https://shiyuu-sf.hatenablog.com/entry/dodecahedron} \\ 要約:下村による「十二面体関係」の読解は、代数的構造を文芸的に素朴に解釈して得られるものである。しかし、そのような最も単純な数理文学的行為によって得られた読解は、新鮮で興味深いものに思える。このようにして、我々は、数理文学によって、創作・批評の双方で新たな文学を得ることが出来る。}に書いた通り。

	\section{円城塔による数理的アプローチ}

		《エクリヲ》8号収録の「言葉と小説の果て、あるいは始まりはどこか」\cite{ecrito}(円城塔へのメールインタビュー)において、円城塔自身が自らの取り組みについて分析的に言及しているので、非常に参考になる。

		\begin{itemize}

			\item 文系的な言語と理系的な言語の共参照

			\item 文章という系と脳神経系の相互作用

		\end{itemize}

		これらの代表として、それぞれ「ムーンシャイン」と「ドルトンの印象法則」が挙げられる。

		今回は、テクストと読者から構成される読書という系について、相互作用としての読書体験を考えていきたい。

		\subsection{相互作用}

			\begin{dfn}

				相互作用とは、2つ以上の物体が互いに作用し、互いの状態を変化させることである。
			
			\end{dfn}

			読書という経験はテクストという系と読者という系の相互作用として記述出来て、そのような相互作用を記述する体系として最適なのは物理である。

			\subsubsection{「ドルトンの印象法則」}

				柏書房『夕暮れの草の冠』収録。ドルトンの法則(分圧の法則)の文章に関するアナロジである“ドルトンの印象法則”なる法則についての話。

				要約すると、マクロな系である文学テクストの操作によってミクロな系である読者の変容が実現されることを、熱力学的に示した作品。

				\begin{itemize}

					\item 文学テクスト:マクロな熱力学系

					\item 読者(脳神経系):ミクロな系

				\end{itemize}

				テクストの変化によって、変化しているのは自分自身であるということを確認することの浮遊感、センス・オブ・ワンダー。

			\subsubsection*{分圧則の導出}

				理想気体の状態方程式$ pV = nRT $を認めて議論する。

				全体積$ V $の容器を$ V_A , V_B $に分割し、それぞれに圧力$ p $の理想気体A, Bを封入する。また、系の温度は$ T $で常に一定である。

				このとき、気体A, Bそれぞれの状態方程式は、$ \begin{cases} p V_A = n_A R T \\ p V_B = n_B R T \end{cases} $

				容器のしきりを外して各気体を自然に混合させると、状態方程式は、$ p (V_A + V_B) = (n_A + n_B) RT$

				$ (V_A + V_B) = V $として、$ p = \frac{n_A R T}{V} + \frac{n_B R T}{V} $

				右辺の項は、それぞれ気体A, Bが全体積Vを占めたときの圧力に等しい。この圧力を、気体A, Bの分圧という。

				各気体の分圧の和が全圧$ p $に等しいことも分かる。これを、ドルトンの分圧法則(以下、分圧則)という。

			\begin{thm}

				分圧則:混合気体の圧力は、各純粋気体の分圧の和に等しい。

			\end{thm}

			\subsubsection{「Your Heads Only」}

				ハヤカワ文庫JA『Boy's Surface』収録。「ドルトン」と実は似てる。

			\subsubsection{「梅枝」}

				新潮文庫『文字渦』収録。

				装丁、書体などを含めた“テクスト”の変容によって、文章の印象が変化することを指摘。

				「ドルトン」的に言えば、テクストを構成する文字や“体積”が変化しているのだから、そりゃ当然“印象”も変わるでしょ、という感じ。

		\subsection{共参照(今回は扱いません)}

			\subsubsection*{前提として}

				十分な強さをもつ任意の形式的体系において、その体系では証明出来ない真である算術の文が存在する(第一不完全性定理)。証明は「for Smullyan」の項にあり。

				平たく言うと、任意の記述体系には、真ではあるのだが、まるで盲点のように言及出来ない要素が存在する、ということ。ここからのアナロジで、ひとつの表現領域(例えば小説)では表現できないものがあるのでは、ということ。他者同士で行われる相互作用を、同一の系の中で作用させると共参照(あるいは自己言及)と呼ばれるような状況になってくる。

			\subsubsection{「for Smullyan」}

				「存在するが、提示出来ない物語が存在する」という主張をする作品。

				まあ、アナロジとしても合っているし、自然言語は一般に十分な強さをもつので、間違ってはいない。全ての可能な小説を提示することは出来ない、じゃあどうやって小説の表現可能な領域を増やしていきましょうか、という話。

				要するに、現状把握のための文章。具体的な解決策を考えるのはまた別の話。

			\subsubsection{「ムーンシャイン」}

				ハヤカワ文庫JA『日本SFの臨界点 恋愛篇』収録。

				膨大な計算力を背景に数学を直観的に把握するに至った少女が、さらに一歩先の抽象的な数学的宇宙に踏み出していく様を、(比較的)読者に近い視点をもつ少年の視点から描く話。

				少女視点だけ、少年視点だけだとそれぞれ意味不明だが、異質な両者が互いに理解し合おうとする働き(共参照)によって物語が語られていく。

		\subsection{翻訳(今回は扱いません)}

			最初期の「松ノ枝の記」以来、円城塔作品に頻出するモチーフである。

			研究者を経験しているので当然英語は出来るだろうし、異なる言語間での情報伝達というもの関心があったとしても当然のことだと思う。

			\subsubsection{「松ノ枝の記」}

				講談社文庫『道化師の蝶』収録。

				原本の翻訳の翻訳の翻訳、翻訳が先行して原本が後から出る、そもそも存在しない原本の翻訳など、空中に架橋するような奇妙な翻訳が多数登場する。

				この本のメインは芥川賞受賞作「道化師の蝶」だが、これはかなり読みづらい作品なので、こちらの「松ノ枝の記」の方が楽しく読めると思う。

			\subsubsection{「この小説の誕生」}
				
				《群像》2020年8月号収録。

				翻訳のブラックボックス性と脳のブラックボックス性を対比させる。いまそこで物語が立ち上がる様を記述している。伊藤計劃との合作「解説」(ウィリアム・ギブスン&ブルース・スターリング『ディファレンス・エンジン』の解説)も同様の形態をとる。

				意味を保ちつつ、円城塔本人の意図を超える文章が生成されることを期待し、それは成就される。

			\subsubsection{「梅枝」}

				新潮文庫『文字渦』収録。冒頭に紫式部『源氏物語』「梅枝」の直訳の一部が引かれている。

			\subsubsection{『怪談』}

				KADOKAWA。ラフカディオ・ハーン(小泉八雲)による『KWAIDAN』を“直訳”したもの。

				日本の怪談を英語に訳したとき、英語圏読者からするとその物語はわれわれの知る怪談とは異なる奇妙な物語として受け取られたことが想像される。逆に、本書は、異様な物語としての怪談を演出するため、わざと直訳風の文体で再構成されている。

		\subsection{自動生成(今回は扱いません)}

			\subsubsection{「Japanese」}

				ハヤカワ文庫JA『Self-Reference ENGINE』収録。平仮名、片仮名、平片仮名、片平仮名、平片片仮名、etc. という感じで高次の仮名が無秩序に出土する話。

			\subsubsection{「天書」}

				新潮文庫『文字渦』収録。漢字を射出する“門構え”が登場し、門構えを射出する門構えも登場する。

	\clearpage

	\section{付録:ゴジラS.Pについて}

		\begin{quote}

			「答えはいつも目の前にあり、

			答えは世界そのものであり、
			
			世界は問いの中にあり」\hspace{\fill}『ゴジラS.P』第1話OP
		
		\end{quote}

		円城塔が脚本・シリーズ構成・SF考証を務めたアニメ。ウェブではNetflix配信。

		円城塔本人によるノベライズでは、単純なノベライズではなく、作中に登場するAI“ナラタケ”のある分岐体(ブランチ)による視点で語り直される。

		アニメ版はともかく、小説版には(物理学的・数学的に)難しい内容はほとんど含まれておらず、他の円城塔作品と比較すると論理が極めて明瞭であり、学術的に難解な概念も存在しないので分かりやすい。円城塔作品への入門には、結構向いている気がする。毛色が違う感じもするので、これが通常営業だと思って他の作品を読むと面食らうだろうけど

		『ゴジラS.P』を一言で表すと、“終わりは始まる前からわかっていたのに、それが終わりであることが分からなったお話”。あるいは、“終わりが終わりであることが、終わってみてはじめて気づくお話”。

		厳密には、時間閉曲面(Closed Timelike Curve, CTC)(特にGödel解\footnote{円筒状の時空の側面を一周して同一世界点に帰着するような世界線を自然に構築すること(=時間方向に閉じた時空閉曲面の存在)を許すEinstein方程式の厳密解。})における物語の自然な構成による物語。

		Gödel解の時空において、時間順序は自明ではなく、定義不可能($ \because $各世界点に時間推進演算子を適当な回数作用させることで元の世界点に帰着するため)。図示すると、ぐるっと一周回って元に戻る矢印。『ドラえもん のび太の大魔境』に近いという説はある。


	\section{おわりに(+宣伝)}

		来年から東京か京都のどちらかで勤務することになりました。これまでよりSFイベントに参加しやすくなったので、その際はぜひよろしくお願いします。

		今年2月に刊行された早川書房『ハヤカワ文庫JA総解説1500』に、円城塔『Self-Reference ENGINE』『後藤さんのこと』をはじめ、いくつか作品解説を寄稿しました。お手元にぜひ一冊どうぞ。

		2021年12月25日に発売されたSFマガジン2022年2月号(特集:未来の文芸)に、大学SF研を中心としたファンダムについてのエッセイ『未来のSFを担う人に:SFへの参加のすすめ』を寄稿しました。若い世代のSF活動について関心のある方や、SFファンダムに興味はあるけど躊躇のある方、そもそもファンダムというものがよくわからない方など、様々な方に読んでいただけるように日本におけるファンダムの概略から最新の動向まで丁寧に書いています。ちょっと古めですが、よろしくお願いします。

		今月25日発売のSFマガジン2022年12月号(ヴォネガット生誕100周年記念号)に、『タイタンの妖女』『スラップスティック』『バゴンボの嗅ぎタバコ入れ』『はい、チーズ』『これで駄目なら』の解説5点を寄稿したので、こちらもよろしくお願いします。

	\begin{thebibliography}{99}

		\bibitem{ecrito} 円城塔, 言葉と小説の果て、あるいは始まりはどこか, エクリヲ, \textbf{8}, 8-19, 2018

	\end{thebibliography}

	\clearpage

	\section*{注記}

		本資料は、2022年に開催された京都SFフェスティバル2022で行われた「円城塔作品を語る部屋ふたたび」のために制作された2022年10月頃成立の原資料\footnote{\url{https://scrapbox.io/allreferenceengine/%E4%BA%AC%E3%83%95%E3%82%A7%E3%82%B92022%E5%86%86%E5%9F%8E%E5%A1%94%E9%83%A8%E5%B1%8B%E8%B3%87%E6%96%99}}を2023年6月にPDFとして再編集したものである。

		参考文献や出典については後日補った情報が多いが、全て成立当時には利用可能であったことに注意されたい。

		それ以外の箇所については、特に断りのない限り、欠損部分も含め、全て原資料が成立した2021年10月当時の記述をそのまま転記している。

\end{document}