\documentclass[10pt, a5paper, twoside]{jsarticle}

\usepackage{okumacro}
\usepackage{enumitem}
\usepackage{amssymb}
\usepackage{amsmath}{}
\usepackage{amsfonts}
\usepackage{amsthm}
\usepackage{url}
\usepackage{here}
\usepackage[dvipdfmx]{graphicx}
\usepackage{wrapfig}
\usepackage{makeidx}
\usepackage{braket}
\usepackage{ascmac}
\usepackage{fancyhdr}
\usepackage[top=20truemm,bottom=20truemm,left=15truemm,right=15truemm]{geometry}

\pagestyle{fancy}
	\fancyhead{}
	\fancyhead[RE]{円城塔研究}
	%\fancyhead[LO]{TITLE OF ARTICLE}
	\fancyhead[LE, RO]{\thepage}
	\fancyfoot{}
	\fancyfoot[LE, RO]{\footnotesize{The Journal of EnJoeToh Research, Vol.2, No.3, 2024} }

\theoremstyle{definition}
	\newtheorem{dfn}{定義}
	\newtheorem{thm}{定理}

\setcounter{page}{21}

\begin{document}

	\begin{center}

		\Large{\textit{memorandum}}

	\end{center}

	\newpage

	{\large 執筆者紹介}

	\vspace{3mm}

	下村思游

	 司書,SF研究者,SFレビュアー.専門は素粒子物理学,加速器科学.

	 著書に『ハヤカワ文庫JA1500総解説』(共著,早川書房,2022).

	 直近の仕事は『SFマガジン』2024年10月号円城塔『ムーンシャイン』書評.

	 Bluesky: @ss-scifi.bsky.social

	 mastodon: @ss\_scifi@mstdn.jp

	 mail: shiyuu.shimomura\_at\_gmail.com : \_at\_ $\rightarrow$ @

	\vfill

	\hrulefill

	\center

	{\Large 円城塔研究 2巻3号}

	2024年12月31日 発行

	\flushleft{編集 数理文学研究会\\    京都, 日本\\発行 柏屋\\    京都, 日本, \url{https://sites.google.com/view/kashiwaya}\\ISSN 2759-0178(print)\\ISSN 2758-9358(online)}

	\hrulefill

	内容に関する質問・意見等は shiyuu.shimomura\_at\_gmail.com : \_at\_ $\rightarrow$ @




\end{document}