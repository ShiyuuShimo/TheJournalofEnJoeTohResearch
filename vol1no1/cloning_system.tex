\documentclass[10pt, a5paper, twoside]{jsarticle}

\usepackage{okumacro}
\usepackage{enumitem}
\usepackage{amssymb}
\usepackage{amsmath}{}
\usepackage{amsfonts}
\usepackage{amsthm}
\usepackage{url}
\usepackage{here}
\usepackage[dvipdfmx]{graphicx}
\usepackage{wrapfig}
\usepackage{makeidx}
\usepackage{braket}
\usepackage{fancyhdr}
\usepackage[top=20truemm,bottom=20truemm,left=15truemm,right=15truemm]{geometry}

\pagestyle{fancy}
	\fancyhead{}
	\fancyhead[RE]{円城塔研究}
	\fancyhead[LO]{系の複製に関する古典的考察}
	\fancyhead[LE, RO]{\thepage}
	\fancyfoot{}
	\fancyfoot[LE, RO]{\footnotesize{The Journal of EnJoeToh Research, Vol.1, No.1, 2023} }


\theoremstyle{definition}
	\newtheorem{dfn}{定義}
	\newtheorem{thm}{定理}

\setcounter{page}{15}

\begin{document}

	{\Large  } %強制改行

	\begin{center}

		\Large{系の複製に関する古典的考察}

		\vspace{3mm}

		\large{A classical discussion about cloning a system}

		\vspace{3mm}
		
		\large{下村思游}

	\end{center}

	\vspace{3mm}

	\begin{abstract}

		円城塔作品において,ある系を別の系に転写しようという試みがしばしば登場する.本研究では,最も単純なモデルとして単一の粒子からなる系を考察した.この系は,系の濃度によって,濃度が0,自然数$n$,$\aleph_0$のように場合分けされる.このうち,濃度0または$\aleph_0$のとき転写の前後で系の濃度は不変であり,$n$のときのみ転写先の系の濃度は増大する.

		%概要の完成

		\vspace{3mm}

		In the EnJoeToh's works, it is often appeared that trying to clone a system to other systems. We discuss the simplest model, the systems consisting of same particles. These systems are classified into 0, natural number $n$, and $\aleph_0$ by the cardinality of the systems. When the cardinality of the system is 0 or $\aleph_0$, it is invariant to transcription. On the other hand, when it is $n$, it increases after transcription.

	\end{abstract}

	\section{導入}

		円城塔作品において,「Boy's Surface」におけるモルフィズムや,「文字渦」における始皇帝の問題,「$\varnothing$」における全宇宙の記述,「ドルトンの印象法則」における読書という系に関する考察など,知りたい系の情報を保存したまま変形・転写しようとする試みがしばしば登場する.ここでは特に,ある系を情報を保存したまま別の系に転写することについて数学的・物理学的に考察する.

		この考察は円城塔作品に頻出する問題の1つを個別に切り出して理解しようとする試みである.これによって,今後の研究の効率化が期待される.

	\section{考察}

		\subsection{物理学的前提条件}

			時空は空間3次元$+$時間1次元の4次元時空であるとする.扱う系は古典的であり,系を構成する粒子は内部構造をもたない単一の粒子であり,物理量として運動量のみをもつとする\footnote{この物理量の候補としては,運動量以外にも,ベクトル量である物理量,あるいは適当な操作によりベクトル量を構成出来るスカラー量(例えば温度)が考えられる.}.各粒子間の相互作用は考慮しない.

			系の転写は古典的な手法によって行われる.ここで,転写とは,以下のように定義される.

			\begin{dfn}
			
				転写とは,元の系の情報を保存したまま,別の系で記述することである.
			
			\end{dfn}

			転写において,元の情報を圧縮することは認められない.転写においては,元の情報と1対1に対応する形で複製されなければならない\footnote{要するに,転写とは,プログラミングにおけるコンパイルと同義である.つまり,本考察は,“古典的計算機”の構成方法の考察と言って差し支えない.}.また,ある系が転写され別の系でも同様に表現されるならば,その構成は無数に存在する\footnote{ひとつでも等価な表現を与える構成が存在するならば,その構成と等価な表現を与える冗長な構成が無数に構成出来るため.}.この無数の構成のうち,最も少ない要素からなる構成を基本構成という.

			\begin{dfn}

				基本構成とは,同じ情報を記述する系のうち,最も少ない要素からなる構成のことである.
			
			\end{dfn}

		\subsection{数学的定義}

			本論文で考察する系は,粒子がいくつか集まって構成される.このような“ものの集まり”を集合という.

			\begin{dfn}
				
				集合とは,いくつかのものをひとまとめにして考えた‘ものの集まり’のことである\cite{matsu}.

			\end{dfn}

			このとき,集合に含まれる“もの”の1つ1つを要素または元という.

			%同等の定義

			集合同士を比較するとき,最も基本的なものとして集合の“大きさ”を比較することが挙げられる.ある集合の“大きさ”と,別の集合の“大きさ”が同じとき,それらの集合は同等である.

			\begin{dfn}
				
				2つの集合$A$, $B$の間に全単射(1対1対応)が存在するとき,$A$と$B$は同等である,と言い,記号で$A \sim B$と書く\cite{hara}.

			\end{dfn}

			%基数の定義

			集合の“大きさ”を表す尺度として,基数と濃度が用いられる.

			\begin{dfn}
				
				有限集合の基数は自然数である.また,可算無限集合の基数は$\aleph_0$である.

			\end{dfn}

			\begin{dfn}

				濃度は,以下のすべてを満たす\cite{nlab}.
				
				\begin{enumerate}
					\item すべての集合はただ1つの濃度をもつ.
					\item すべての基数はなんらかの集合の濃度である.
					\item 2つの集合が同じ濃度をもつとき,かつそのときに限り,両者の間には全単射が存在する.
				\end{enumerate}

			\end{dfn}

			なお,基数は可算無限集合より大きな集合に対しても考えることが出来るが,定義や議論が過度に複雑になる\footnote{例えば,連続体仮説など.}上,本考察では可算無限集合より大きな集合を考慮しないため,可算無限集合までの定義に留めた.

			ここで,濃度の定義より,“集合の集合”は集合ではない\footnote{“集合の集合”は集合ではないとする集合論を公理集合論と言い,そうでない集合論を素朴集合論という.素朴集合論では,ラッセルのパラドックスをはじめとする深刻な矛盾が引き起こされることが知られている.}.また,有限集合の場合,その濃度はその元の個数に一致する.

		\subsection{系に関する考察}

			以下,系の複雑さによって分類し,考察する.

			%上記の仮定より,系はもはや自身を構成する粒子の数だけでしか区別できない.

			最初に,対象となる系$S$が0個の粒子から構成されるとき,すなわち空集合であるとき.このとき,空集合である系$S$の転写の基本構成は空集合である系$S'$である.系$S$と系$S'$はどちらも等価な空集合$\varnothing$であるから,系$S$の転写は自分自身である\footnote{空集合$\varnothing$は内部構造をもたないため.また,空集合$\varnothing$は自明なクワイン$^{*6\text{-}1}$.\\$^{*6\text{-}1}$自分自身と全く同じ文字列を出力する文字列のことを,クワインという\cite{geb}.}.

			次に,対象となる系$S$が$n$個の有限個の粒子から構成されるとき.このとき,自由度3の$n$粒子からなる系$S$は,自由度$3n$の1粒子からなる系$S_{3n}$と等価である.自由度1を記述するためには少なくとも1つの粒子が必要となるため,系$S_{3n}$は少なくとも$3n$個の粒子からなる系$S'$に転写される\footnote{系$S$の転写先の濃度が$3n$以上であることは,系$S'$の基本構成が必ず$3n$になることを意味しないことに注意.}.

			最後に,対象となる系$S$が可算無限\footnote{自然数全体$\mathbb{N}$と同等であること.}個の粒子から構成されるとき.このとき,系$S$の濃度は$\aleph_0$であり,濃度$\aleph_0$の集合の要素を自然数倍しても濃度は不変であるから,系$S$から濃度$\aleph_0$である任意の系$S'$への全単射が存在する.つまり,系$S$は同じ濃度をもつ系$S'$へ転写される.

			ここで,転写元の系$S$と転写先の系$S'$の濃度に注目すると,系$S$の濃度が0あるいは$\aleph_0$の場合はそれぞれ同じ濃度の系$S'$に転写され,系$S$の濃度が自然数$n$の場合は濃度$3n$がより大きい系$S'$に転写される.すなわち,転写元の系の濃度が0か$\aleph_0$の両極端であるときのみ転写の前後で濃度は一致し,その中間では転写先の系の濃度は必ず転写元の系の濃度より大きくなる.

	\section{結論}

		系の転写について,転写元の系の濃度によって場合分けをし,その振る舞いを見た.転写元の系の濃度が0または可算無限のとき,系は同じ濃度の系に転写される.転写元の系の濃度が自然数のとき,系はより濃度の大きい系に転写される.

		本考察で扱った,系から別の系への転写という構造は,“対象”から別の“対象”へと伸びる“矢印”として解釈出来る\footnote{この解釈自体が“系から別の系への転写”という構造から“対象から別の対象へと伸びる矢印”という構造への高階の転写になっている$^{*9\text{-}1}$ことに注意.\\$^{*9\text{-}1}$そもそも,本考察自身,“円城塔作品に頻出する転写という試み”の高階の転写である.}.したがって,本考察は圏を用いて等価な表現を構成出来る可能性が示唆されており\footnote{実際,本考察における系と転写はそれぞれ圏論における対象と射に相当し,かつ合成則・結合則・単位則を満たすことから,系と転写からなる対は確かに圏の定義を満たす.},今後の研究が期待される.

		また,対象となる系として“この宇宙全体”を考えると,この宇宙全体に含まれる素粒子の数は有限個に過ぎないため,転写先の系は“この宇宙全体”よりも大きくなければならない.したがって,この宇宙全体の厳密な転写をこの宇宙の内部で構成することは不可能\footnote{なお,厳密な構成でなく,総体としてそれらしく振る舞うように転写することは可能.実際,チューリング完全であるライフゲームは再帰的構成が可能なことが知られている.実装例は\cite{shr}を参照されたい.}.

	\section{謝辞}

		本考察において,議論に多大な助言を与えてくださった名古屋大学大学院多元数理科学研究科のI氏に深く感謝申し上げる.

	\begin{thebibliography}{99}

		\bibitem{matsu} 松坂和夫, 『集合・位相入門』, 岩波書店, 1968

		\bibitem{hara} 原啓介, 『集合・位相・圏』, 講談社, 2020

		\bibitem{nlab} \textit{cardinal number}, nLab, \url{https://ncatlab.org/nlab/show/cardinal+number}, 2023

		\bibitem{geb} ダグラス・R・ホフスタッター, 『ゲーデル,エッシャー,バッハ : あるいは不思議の扉』, 白揚社, 1985

		\bibitem{shr} さはら, \textit{Life Universe}, oimo.io, \url{https://oimo.io/works/life/}, 2022

	\end{thebibliography}

\end{document}