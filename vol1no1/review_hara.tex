\documentclass[10pt, a5paper, twoside]{jsarticle}

\usepackage{okumacro}
\usepackage{enumitem}
\usepackage{amssymb}
\usepackage{amsmath}{}
\usepackage{amsfonts}
\usepackage{amsthm}
\usepackage{url}
\usepackage{here}
\usepackage[dvipdfmx]{graphicx}
\usepackage{wrapfig}
\usepackage{makeidx}
\usepackage{braket}
\usepackage{fancyhdr}
\usepackage[top=20truemm,bottom=20truemm,left=15truemm,right=15truemm]{geometry}

\pagestyle{fancy}
	\fancyhead{}
	\fancyhead[RE]{円城塔研究}
	\fancyhead[LO]{書評:原啓介『眠れぬ夜の確率論』}
	\fancyhead[LE, RO]{\thepage}
	\fancyfoot{}
	\fancyfoot[LE, RO]{\footnotesize{The Journal of EnJoeToh Research, Vol.1, No.1, 2023} }

\theoremstyle{definition}
	\newtheorem{dfn}{定義}
	\newtheorem{thm}{定理}

\setcounter{page}{25}

\begin{document}

	{\Large  } %強制改行

	\begin{center}

		\Large{書評:原啓介『眠れぬ夜の確率論』}

		\vspace{3mm}
		
		\large{下村思游}

	\end{center}

	\section{書誌情報}

		\begin{itemize}
			
			\item 題名 : 眠れぬ夜の確率論

			\item 著者 : 原啓介

			\item 出版社 : 日本評論社

			\item 出版日 : 2020-07-25

			\item ISBN : 9784535789173
		
		\end{itemize}

	\section{書評}

		本書は、立命館大学元教授で確率論を専門とする数学者原啓介による、確率にまつわる奇妙で不思議な話題を紹介する本である。本書は雑誌『数学セミナー』2018年4月号から1年にわたって連載された12の文章を元に加筆・整理されたもので、単行本化にあたって全5部15章立てとなっている。この5つの部にはそれぞれ原理、意味、数理、推理、そして人間という題が与えられており、確率論の基礎に近いところからこの宇宙が存在する確率、あるいは私たち人間という知的生命が存在可能な確率といった突飛な確率についての話まで広がっていく。

		本書は数学雑誌の連載が元になっているだけあって、数学に関する一定程度の訓練を経験していることを前提とする部分もあるのだが、全体としては不思議な確率論の世界を巡る読み物として楽しめる。そんな本書を書評の第一弾として取り上げたのは、円城塔本人が、本書を円城塔作品評論に取り組みたい人に勧めているからである\footnote{「僕の本の評論をしたいとかいう(奇特な)人にもお勧めしたい。」 \\ \url{https://twitter.com/EnJoeToh/status/1291302813880868864?s=20}}。

		円城塔は、なぜ、このようなことを言ったのだろうか。私は、以下のようなことを望んでいたのではないかと考えている:身近な現実的問題を解決するために導入された確率論が、その期待に反して極めて複雑かつ魅力的な体系であると紹介すること。確率論におけるコルモゴロフの偉大な足跡を足がかりに、コルモゴロフ複雑性\footnote{本書では「コルモゴロフ複雑度」として登場するが、どちらも等価な表現である。}から停止性問題や第一不完全性定理に至る数学の深淵の畔を案内すること。そして、それらの概念の難解さが自分自身を含む系を扱うことの困難さに起因すると紹介すること。

		そもそも、確率論とは、博打うちたちが自分の勝ちを最大にするために賭博を研究する過程で成立したものだった。本書では愛すべき博打うちの例として『源氏物語』に登場する近江の君がまず挙げられ、実在の人物では数学者フェルマーとパスカルの名も挙げられる。このフェルマーとパスカルによる素朴な議論から、確率論を取り巻く様々な話題が次から次へと提示されていく。ゲーム理論におけるナッシュ均衡の初歩的な例示ではホームズとモリアーティー教授の対決が取り上げられ、ルイス・キャロルの確率論に関する考察を土台に長さや面積といった幾何学的対象が確率論と密接に結びつくことが示されるように、説明の随所で文学的なエッセンスが散りばめられ、数理科学に馴染みの薄い読者でも楽しく読み進められるような配慮がなされている。

		そして話はコルモゴロフによる確率論の現代化に移る。コルモゴロフはルベーグによる測度論を経て確率の現代的な定義を与えた。この確率の現代的定義を元に、コルモゴロフはランダムな現象に見出せる“ランダムさ”をコルモゴロフ複雑性という概念で表した。それはやがて(チューリング機械と同様な)任意の計算機について、その計算が計算可能かどうかを判定するアルゴリズムが存在するかどうか、という停止性問題\footnote{停止性問題とは、任意の計算が計算機で計算可能かどうかを確かめる試みである。当然計算可能であることは素朴に期待されていた(そうでなければ、この世には解くことの出来ない計算が存在することになる)が、停止性問題はチューリングによって否定的に証明された。つまり、計算機には原理的に計算不可能な計算が必ず存在する。}に発展していくことになる。

		停止性問題と第一不完全性定理の難解さは、自分自身が扱いたい系の内部に存在していることに起因する。すなわち、停止性問題では停止性問題自身が解くべき問題に含まれることが、第一不完全性定理では無矛盾であることを示したい数学的体系に第一不完全性定理自身が含まれることが、その取り扱いを極度に困難にしてしまう。これらの問題は天才たちによる巧みな構成によって困難を回避し、それぞれ驚くべきだが自明な命題を主張した。この自己言及構造の困難さについて、第5部「人間」では人間原理をはじめとした興味深い例題を扱うことで理解を深めていく。自己言及の問題は現代数理科学の様々な箇所で直面するもので、人間原理で取り上げられている天文学・宇宙論の分野においては、自分自身が含まれている宇宙について、その宇宙の内部から観測を行うことの困難さとして表面化する\footnote{一方で、実のところ、ガウスの驚異の定理などによって、それほど困難でないことが知られている。}。

		現代科学は、20世紀前半に次々に発見された自己言及構造に起因する困難さに直面しつつ、巧みな操作によってその困難を解消して今日まで発展してきた。本質的な困難さを扱う円城塔作品を理解するために、また相互に連絡し合う極めて複雑なネットワークとしての現代科学を理解するために、ぜひ本書を読むことを勧めたい。

		\begin{thebibliography}{9}

			\bibitem{hara1} 原啓介, 『測度・確率・ルベーグ積分』, 講談社, 2017

			\bibitem{hara2} 原啓介, 『線形性・固有値・テンソル』, 講談社, 2019

			\bibitem{hara3} 原啓介, 『集合・位相・圏』, 講談社, 2020

			いずれも同著者による速習本。どれも記述がよくまとまっており、過不足がなく、初学者が初めて手にとるコンパクトなテキストとして非常に使い勝手がいい。

		\end{thebibliography}

\end{document}