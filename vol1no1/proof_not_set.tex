\documentclass[10pt, a5paper, twoside]{jsarticle}

\usepackage{okumacro}
\usepackage{enumitem}
\usepackage{amssymb}
\usepackage{amsmath}{}
\usepackage{amsfonts}
\usepackage{amsthm}
\usepackage{bm}
\usepackage{url}
\usepackage{here}
\usepackage[dvipdfmx]{graphicx}
\usepackage{wrapfig}
\usepackage{makeidx}
\usepackage{braket}
\usepackage{ascmac}
\usepackage{fancyhdr}
\usepackage[top=20truemm,bottom=20truemm,left=15truemm,right=15truemm]{geometry}

\pagestyle{fancy}
	\fancyhead{}
	\fancyhead[RE]{円城塔研究}
	\fancyhead[LO]{“集合の集合”が集合でないことの証明}
	\fancyhead[LE, RO]{\thepage}
	\fancyfoot{}
	\fancyfoot[LE, RO]{\footnotesize{The Journal of EnJoeToh, Vol.1, No.1, 2023} }

\theoremstyle{definition}
	\newtheorem{dfn}{定義}
	\newtheorem{thm}{定理}
	\newtheorem{clm}{主張}
	\newtheorem{axm}{公理}
	\newtheorem{prf}{証明}

\setcounter{page}{11}

\begin{document}

	~ %強制改行

	\begin{center}

		\Large{“集合の集合”が集合でないことの証明}

		\vspace{3mm}

		\large{The proof that a collection of sets is not a set}

		\vspace{3mm}
		
		\large{下村思游}

	\end{center}

	\vspace{3mm}

	\begin{abstract}

		本論文では,ラッセルのパラドックスの解決を通じて,公理的集合論において“集合の集合”が集合でないことを示した.自己言及構造をもつラッセルのパラドックスの解決策を確認することは,自己言及的な構造が頻出する円城塔作品の理解に役立つ.

		\vspace{3mm}

		In this paper, we introduce that a collection of sets is not a set in Zermelo-Fraenkel set theory (ZF set theory) by solving Russell's paradox. It is useful that checking the resolution of Russell's paradox contains self-referencial structure for understanding the works, written by EnJoeToh, self-referencial structure often appeared in.

	\end{abstract}

	\section{導入}

		本論文では,“集合の集合”が集合ではないことを示す.これは歴史的にはラッセルのパラドックスとして知られ,自己言及(self-reference)構造をもつパラドックスの代表的なものの1つである.ラッセルのパラドックスは,その登場時(1900頃)\cite{fre,dis}に多大なセンセーションを巻き起こし,人間の合理的理性の根本を揺るがす大発見として受け止められた\cite{noe}.

		しかし,その当事者である数学では,ラッセルによる型理論の発表(1903)\cite{rus},ツェルメロによる公理的集合論の提唱(1908)\cite{zer}という複数の手法により,ラッセルのパラドックスは早期に解決を見た.

		円城塔作品では,自己言及的な構造が装いを変えて何度も登場する.そこで,自己言及構造をもつパラドックスの代表であるラッセルのパラドックスがいかにして解決されたのかを見ることで,円城塔作品の理解の礎としたい.

	\section{本論}

		\subsection{ラッセルのパラドックス}

			ラッセルのパラドックス以前に数学で用いられていた(素朴)集合論においては,以下に示す内包公理を素朴に認めていた.

				\begin{axm}
					
					内包公理

					任意の述語$P(x)$に対して,$P(x)$を満たす元$x$の集合$ \{ x \ |\  P(x) \} $が存在する.

				\end{axm}

			ここで,述語とは,日常語ではなく,数学の厳密な定義語である.ここでは,受け取った値に対して真偽値を返す関数であると考えていただければよい.

			この内包公理を用いると,ラッセルのパラドックスが導かれる.

				\begin{clm}

					ラッセルのパラドックス

					自分自身を元としてもたない集合全体から成る集合$X$は$X = \{ x \ |\  x \notin x\} $で表される.

					いま,$X \in X$と仮定すると,$X$の構成方法より$X \notin X$が導かれるが,矛盾.

					一方,$X \notin X$と仮定すると,$X$の構成方法より$X \in X$が導かれるが,これもまた矛盾.
				
				\end{clm}

		\subsection{公理的集合論による解消}

			ツェルメロは内包公理を以下に示す分出公理\footnote{なお,分出公理は置換公理と空集合の公理から導出されるため,歴史的経緯を考慮しない場合はZF公理系から外されることが多い.}に制限することによって,矛盾の発生を回避した.

				\begin{axm}
					
					分出公理

					任意の述語$P(x)$と集合$A$に対して,$P(x)$を満たす$A$の元$x$の集合$ \{ x \in A \ |\  P(x) \} $が存在する.

				\end{axm}

			以下,実際に矛盾が生じないことを証明する.

			\begin{prf}\label{sol}
				
				ラッセルのパラドックスの解決

				分出公理にしたがって,任意の集合$A$に対して自分自身を元としてもたない集合全体から成る集合$R_A$を$ R_A = \{ x \in A \ |\  x \notin x \} $と表す.$R_A \in A$と仮定する.

				このとき,背理法によって$R_A$が$A$の要素ではないことを示し(*),さらに背理法によって“集合の集合”が集合でないことを示す(**).

				いま,$R_A \in R_A$と仮定すると,$R_A$の構成方法より$R_A \notin R_A$が導かれるが,矛盾.一方,$R_A \notin R_A$と仮定すると,$R_A$の構成方法より$R_A \in R_A$が導かれるが,これもまた矛盾.

				したがって,$R_A \notin A$である(*).

				次に,集合全ての集まりが集合であると仮定し,これを$U$と表す.

				$A$に$U$を代入すると,(*)より$R_U \notin U$である.これは$U$の定義$R_U \in U$に矛盾する.

				したがって,“集合の集合”$U$は集合ではない(**).

			\end{prf}

			分出公理(あるいは分出公理と等価な公理)を採用した公理的集合論において,ラッセルのパラドックスは,任意の集合に含まれない集合が存在することを主張する定理として振る舞う.したがって,公理的集合論においては,ラッセルのパラドックスはラッセルの定理あるいはラッセルの補題として言及されるべきである\footnote{ここで,証明\ref{sol}で用いた論法は,カントールが非加算濃度をもつ集合の存在を証明する際に用いた対角線論法と酷似していることに注意.}.

			\begin{thm}
				
				ラッセルの定理

				任意の集合$A$に対して,$R_A \notin A$なる集合$R_A$が存在する.

			\end{thm}

	\section{結論}

		このように,ラッセルのパラドックスを克服するために提唱された,ツェルメロおよびフレンケルによるZF公理系(Zermelo-Fraenkel set theory)は,自己言及的な集合の構成を禁止することで,自己言及構文のもたらす矛盾を回避した.しかしながら,自己言及構文を扱うことを禁止したために,ZF公理系およびそれを拡張したZFC公理系は自己言及構造を扱うことが出来ない\cite{ytb}.

		おそらく無矛盾であるZFC公理系\footnote{第一不完全性定理より,ZFC公理系の内部でZFC公理系が無矛盾であることを証明することは不可能.}は,古典数学や現代数学のほとんどの土台となってきた.しかし,先に述べた通り,ZFC公理系では扱うことの出来ない領域があることから,ZFC公理系を離れた議論も盛んに行われている.

		ラッセルのパラドックスは一見極めて難解に見えるが,その解決を通じて人類は豊かな数学的手法を手に入れることが出来た.難解な対象に行き当たっても,諦めず粘り強く取り組み,新境地を開拓するに至った知の営みの片鱗を知ることで,円城塔作品の理解の礎としてただければ幸いである.

	\begin{thebibliography}{99}

		\bibitem{fre} ゴットロープ・フレーゲ, 『フレーゲ著作集6 書簡集付日記』, 勁草書房, 2002

		\bibitem{dis} B. Rang, W. Thomas, Zermelo's discovery of the "Russell paradox", \textit{Historia Mathematica}, \textbf{8}(1), 15-22, 1981, \url{https://doi.org/10.1016/0315-0860(81)90002-1}

		\bibitem{noe} 野家啓一, 『科学哲学への招待』, ちくま学芸文庫, 筑摩書房, 2015

		\bibitem{rus} Bertrand Russell, \textit{The principles of Mathematics}\footnote{ホワイトヘッドとの共著\textit{Principia Mathematica}\textbf{とは異なる}ことに注意.}, Cambridge University Press, 1903

		\bibitem{zer} Ernst Zermelo, Neuer Beweis für die Möglichkeit einer Wohlordnung, \textit{Mathematische Annalen}, \textbf{65}, 107-128, 1908, \url{https://doi.org/10.1007/BF01450054}

		\bibitem{ytb} 矢田部俊介, ラッセルのパラドックス : 傾向と対策(1), あいまいな本日の私blog, \url{https://ytb.hatenablog.com/entry/20070912/p2}, 2007

	\end{thebibliography}

\end{document}