\documentclass[10pt, a5paper, twoside]{jsarticle}

\usepackage{okumacro}
\usepackage{enumitem}
\usepackage{amssymb}
\usepackage{amsmath}{}
\usepackage{amsfonts}
\usepackage{amsthm}
\usepackage{url}
\usepackage{here}
\usepackage[dvipdfmx]{graphicx}
\usepackage{wrapfig}
\usepackage{makeidx}
\usepackage{braket}
\usepackage{fancyhdr}
\usepackage[top=20truemm,bottom=20truemm,left=15truemm,right=15truemm]{geometry}

\pagestyle{fancy}
	\fancyhead{}
	\fancyhead[RE]{円城塔研究}
	\fancyhead[LO]{「文字渦」の難解性}
	\fancyhead[LE, RO]{\thepage}
	\fancyfoot{}
	\fancyfoot[LE, RO]{\footnotesize{The Journal of EnJoeToh Research, Vol.1, No.1, 2023} }

\theoremstyle{definition}
	\newtheorem{dfn}{定義}
	\newtheorem{thm}{定理}

\setcounter{page}{20}

\begin{document}

	~ %強制改行

	\begin{center}

		\Large{「文字渦」の難解性: \\ 内部観測と量子複製禁止定理}

		\vspace{3mm}

		\large{Difficulty of the short story \textit{Mojika}: \\ Internal measurement and quantum no-cloning theory}

		\vspace{3mm}
		
		\large{下村思游}

	\end{center}

	\vspace{3mm}

	\begin{abstract}

		短篇「文字渦」における始皇帝と俑の問答は,連立方程式の解を求めることに帰着される.文献\cite{sshimo}における考察および量子複製禁止定理を考慮すれば,解は自明なもののみ許される.また,本作に対して厳密な数理的読解を行うべきでないとの批判は,作中の描写より否定される.

		\vspace{3mm}

		The argument between the First Emperor and You in the short story \textit{Mojika} is equivalent to solving simultaneous equations. Considering the discuss on reference \cite{sshimo} and quantum no-cloning theory, the solution is limited only trivial one. The criticism that a strict mathematical reading for \textit{Mojika} should not be performed is denied by the content of \textit{Mojika}.

	\end{abstract}

	\section{導入}

		「文字渦」において,始皇帝は,時間と共に移ろう自身の姿を全くそのまま写しとった像を作成せよと命じた.これに対して,俑は,始皇帝の像を作らないことで応えた.これは作中で試行錯誤を経て提示されたアイデアであり,SF的にいかにも面白い解答であるが,実は自明な解答である.これからそれを実際に示す.

		本論文では,「文字渦」について数学的・物理学的に厳密な考察を行う.これについて,小説に対して厳密な数理的読解を行うべきではないとの批判が存在する.しかし,作中において,始皇帝が俑に対して物理学的に厳密な解答を求めていることから,この批判は直ちに否定される.

	\section{考察}

		\subsection{数理的再構成}

		まず,始皇帝の要求を数学的・物理学的に整理したい.

		始皇帝の要求は,“真人”としての始皇帝の像を作れ,というものだった\footnote{「『そなたに命ずるのは,真人の像である』」「あくまでもこのわたしの像をつくるのだ.」\cite{mojika} p.25}.ここで,真人とは,時間発展\footnote{時間が進むにつれて物理量が変化すること.逆に,時間発展しないとは,時間が進んでも一切変化しないことを指す.}しない人間である.このような真人の像を時間$t$を用いて$ S_{figure}(t) $と表す.一方,始皇帝は時間発展する\footnote{「俑の前に現れる嬴は見れば見るほど,とらえどころのない人物である.くるくると印象が変転してとどまらない.」\cite{mojika} p.27}.時間発展する始皇帝を同様に時間$t$を用いて$ S_{Emperor}(t) $と表す.

		$S_{figure}(t)$は$S_{Emperor}(t)$を写し取ったものであるから,等価な観測に対して同一の観測結果を与えることが期待される.つまり,始皇帝の要求は「時間発展する始皇帝$ S_{Emperor}(t) $について,観測を表す0でない演算子$ O $\footnote{$ O = 0 $とすると任意の対$ \{ S_{Emperor}(t)$, $ S_{figure}(t) \} $が方程式を満足するため,$ O \neq 0 $}を作用させたとき,常に始皇帝と同じ観測結果を与える俑$ S_{figure}(t) $を構成せよ」と表される.

		すなわち,始皇帝の要求は連立方程式

		\begin{equation*}
			\begin{cases}
				O S_{Emperor}(t) = A \\ O S_{figure}(t) = A
			\end{cases}
		\end{equation*}

		を満足する$ S_{figure}(t) $を求めることに帰着される.

		\subsection{特殊解}\label{spe}

			$ O \neq 0 $であるから$ S_{Emperor}(t) = S_{figure}(t) = 0 $が方程式を満足することは自明.この自明な解は“始皇帝の俑を作らない”と解釈される.実際,$ S_{figure} = 0 $は時間$t$に依らない\footnote{ある物理量が時間に依存しないとき,かつそのときに限り,その物理量の時間微分は0である.}.つまり,はじめから俑を作らなければ滅びない.

			この特殊解は,\cite{sshimo}において転写元の系の濃度が0の場合に相当する.確かに,転写元の系の構成要素の集合$\{ S_{Emperor}(t) \} = \{ \varnothing \}$は転写先の系の構成要素の集合$\{ S_{figure}(t) \} = \{ \varnothing \}$に転写され,両者は$\varnothing$に等しい.

		\subsection{一般解}\label{gen}

			$ S_{Emperor}(t) = S_{figure}(t) $を厳密に満たすことを要請する.始皇帝は高々有限個の粒子によって構成される系である.始皇帝が0個の粒子で構成される場合は既に論じたので,ここでは有限個の粒子の場合を論じる.

			\cite{sshimo}における議論によって,転写したい系が有限個の粒子から構成される場合,その粒子が内部構造や相互作用をもたないならば,転写元の系$S_{Emperor}(t)$から転写先の系$S_{figure}(t)$への濃度を保存した転写が禁止されることが示されている.このとき,内部構造や相互作用をもたない粒子は議論の簡単のために仮定されたものであった.しかし,本作において始皇帝を構成する粒子は内部構造や相互作用をもつ実在粒子であるから,この議論はそのままでは適用されない\footnote{元の粒子の情報を,内部構造や相互作用による空間的配位を用いて巧妙に転写することが考えられる.}.

			しかし,始皇帝が実在粒子から構成されるならば,その厳密な転写にあたっては,量子情報物理学的な効果も考慮しなければならない.ここで,量子複製禁止定理\footnote{純粋状態と呼ばれる量子状態はその複製が一般に禁止される.これを量子複製禁止定理という.量子計算機は極めて安全であることが知られているが,その安全性を保証するのがこの量子複製禁止定理である.詳細については文献\cite{Hotta1,Hotta2}を参照されたい.}より,未知の系$S_{Emperor}(t)$の厳密な複製は禁止される.

		\section{検討}

			\ref{spe}および\ref{gen}で示した数学的・物理学的考察より,始皇帝の問題の解は自明な特殊解$ S_{Emperor}(t) = S_{figure}(t) = 0 $に限られ,それは“始皇帝の俑を作らない”と解釈されることがわかった.これは作中で俑が示した「皇帝なき秦の姿こそが真人の像である」という解答に一致し,始皇帝はこの解答を容れた.

			ここで,そもそも始皇帝の問題を数学的・物理学的に厳密に考察すべきではないとする批判が可能である.しかし,これは作中の描写より,否定される.

			始皇帝は「時間に侵された像は必要ない」「あらゆる事物は時の流れから逃れられない.そういうことのわからぬ者は埋めるしかない」\footnote{\cite{mojika}p.25}とし,“時間に束縛されない観測”を厳密に行うよう俑に要請している\cite{com}.ここで,“時間に束縛されない観測”が物理学において非相対論的観測に関する理論として厳密に体系化されている\footnote{物理学以外にも“時間に束縛されない観測”に関する議論があるかもしれないが,実際の物理現象に対する検証可能性を考慮すれば,実際に像を作成して提示しなければならない始皇帝の問題を論じる際は,物理学における議論を参照するのが最適であると考えられる,}ことを考えれば,始皇帝の問題およびそれへの俑の解答に関する考察は物理学的に厳密に行われるべきである.

			なお,時間に束縛されない観測である非相対論的観測と対を成す時間に束縛された観測として相対論的観測がある.近代までの素朴な古典的観測理論である外部観測は,20世紀初頭において相対論・量子論に起因する困難を経験し,現代的な観測理論である内部観測(internal measurement)へと適切に拡張された.先に挙げた非相対論的観測は外部観測の一種であり,一方相対論的観測は内部観測の一種である.現実で行われる観測は全て内部観測であり,内部観測についてよく理解することは様々な物理現象を考える上で極めて有用であるが,それには著しく多くの準備を必要とするため,ここでは内部観測という概念が存在することを示すに留める.

			また,内部観測に酷似した体系として内在物理学(endophysics)がある.円城塔作品には「内在天文学」(英題:\textit{Endoastronomy})があること,また円城塔本人が内在物理学の存在を示唆したツイート\footnote{「\url{https://en.m.wikipedia.org/wiki/Endophysics}」$^{*10\text{-}1}$\url{https://twitter.com/EnJoeToh/status/1320283140871446533}\\$^{*10\text{-}1}$URLのみ記載されたツイートであることに注意.}\footnote{ここで,このツイートは下村による「内在天文学」解説記事\cite{endo}の公開から約20分後に前後の文脈から独立して唐突に投稿されたものであることに注意したい.}が存在することからも,内部観測・内在物理学が円城塔作品の理解に有用であることがわかる.一方で,内部観測はその成り立ちから相対論・量子論の知見を前提としている.つまり,円城塔作品の難解さの一部は,その根幹に(ほとんど断りなくしかし自明に)内部観測が導入されていることに起因すると考えられる.しかし,自然現象が内部観測によってしか観測出来ないことを考えれば,円城塔作品に関しても自然現象と同様に当然内部観測を考慮した観測を行うべきであるという主張も出来る.

			本来,未知の現象に対する観測は,その観測方法の特性を十分に検証した上で行われるべきものである.円城塔作品の難解さなるものは,本来行われなければならない基本的な検討を行わないまま作品に取り組んだ結果,当然生じてしまった困難を作品に押し付けたことによる誤解ではないだろうか.その解消のために,その1作1作について検討と考察を慎重かつ適切に行なっていくべきである.


	\section{結論}

		「文字渦」における俑の解答が物理学的に自明な解答であることを示し,また本作について数理的な読解を行うべきではないとの批判は,作中の描写から否定されることを示した.

		本作で見られた,対象となる系を$\varnothing$まで削減することで対象となる系に自分自身を記述させようというアイデアは,のちの「$\varnothing$」にも見られる.これは濃度が0である系を転写する場合に相当する.逆に,系の濃度が$\aleph_0$である場合は,転写元の世界と転写先の世界の区別がつかなくなる,というアイデアとして「良い夜を持っている」に見出される.本論文における議論は,これらの作品の解釈への応用が示唆されており,今後の研究が期待される.

	\begin{thebibliography}{99}

		\bibitem{mojika} 円城塔, 『文字渦』, 新潮文庫, 新潮社, 2021

		\bibitem{sshimo} 下村思游, 系の複製に関する古典的考察, 円城塔研究, \textbf{1}(1), 15-19, 2023

		\bibitem{com} 下村思游, コンメンタール「文字渦」, 円城塔研究, \textbf{1}(1), 1-10, 2023

		\bibitem{endo} 下村思游, 円城塔「内在天文学」について, SF游歩道, \url{https://shiyuu-sf.hatenablog.com/entry/2020/10/25/164339}, 2020

		\bibitem{Hotta1} 堀田昌寛, 『入門 現代の量子力学:量子情報・量子測定を中心として』, 講談社, 2021

		\bibitem{Hotta2} 堀田昌寛, 『量子情報と時空の物理』, 第2版, サイエンス社, 2019

	\end{thebibliography}

\end{document}