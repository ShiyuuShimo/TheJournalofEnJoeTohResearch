\documentclass[10pt, a5paper, twoside]{jsarticle}

\usepackage{okumacro}
\usepackage{enumitem}
\usepackage{amssymb}
\usepackage{amsmath}{}
\usepackage{amsfonts}
\usepackage{amsthm}
\usepackage{url}
\usepackage{here}
\usepackage[dvipdfmx]{graphicx}
\usepackage{wrapfig}
\usepackage{makeidx}
\usepackage{braket}
\usepackage{fancyhdr}
\usepackage[top=20truemm,bottom=20truemm,left=15truemm,right=15truemm]{geometry}

\pagestyle{fancy}
	\fancyhead{}
	\fancyhead[RE]{円城塔研究}
	\fancyhead[LO]{資料:京フェス2021円城塔部屋資料}
	\fancyhead[LE, RO]{\thepage}
	\fancyfoot{}
	\fancyfoot[LE, RO]{\footnotesize{The Journal of EnJoeToh Research, Vol.1, No.1, 2023} }

\setcounter{page}{28}

\begin{document}

	{\Large  } %強制改行

	\begin{center}

		\Large{京都SFフェスティバル2021円城塔部屋資料}

		\vspace{3mm}
		
		\large{下村思游}

	\end{center}

	\vspace{3mm}

	\section{はじめに}

		本稿は京都SFフェスティバル2021 \footnote{ \url{https://kyofes.kusfa.jp/cgi-bin/Kyo_fes/wiki.cgi} }合宿企画「円城塔作品を語る部屋」の資料である。

		この通り企画を進めるとは限らないものの、とりあえずここに書けるだけ書いていこうと思う。

		企画内容自体は円城塔作品を読んだことのない人には厳しいものになるかもしれないが、これから円城塔作品を読み始めようという方向けに、初心者向けの作品リストを載せておく。

		\subsubsection*{京フェスについて}

			京都SFフェスティバル、通称京フェスは、京大SF研主催のSFイベントである。例年は京都の旅館を借り切り、夜を徹して行われているのだが、今年はコロナの関係でDiscordとZoomを併用してウェブ上での遠隔開催となっている。参加費無料、聞きたい企画だけ聞いて退散もOKなので、気になる企画があったら気軽に参加することをおすすめする。参加登録は公式ページの参加フォームから。

	\section{この企画の目的}

		\begin{itemize}

			\item 数学・物理学・論理学・計算機科学など数理科学の概念を適切に援用することで、円城塔作品の見通しが良くなることを指摘する。

			\item 特に有用な概念についていくつか紹介し、それらが作中でどのように用いられているかを確認する。

			\item よくわからないながら楽しんでいる円城塔作品の面白さの源泉を、ちょっとわかるようになる。

			\item 深夜の円城塔トークを十二分に楽しむ。

		\end{itemize}

	\section{初心者向け円城塔作品}

		\subsection{『文字渦』}
			
			新潮文庫。まずはこれ.

			いつもの円城塔らしさはそのままに、物語の主題はすべて一貫して文字であるので、主題を見失ったり、見当すらつけられなかったりという事態に陥る可能性が非常に小さい。視覚的にインパクトのある仕掛けが多数存在し、SFに慣れていない人にもおすすめ。

			\subsubsection*{『文字渦』ワンポイントアドバイス}

				よくわからなくなったら、文字だけを拾って読む。

				読み進めていくうちに悩むことがあったら、「梅枝」を読み返すと良い。

			\subsubsection*{『文字渦』注意点}

				「梅枝」中程で登場する「二次元五近傍オートマトン」の話は難解。『Boy's Surface』収録の「Your Heads Only」に記載されているオートマトンの説明を前提としており、「梅枝」だけでの理解は困難。

		\subsection{『バナナ剝きには最適の日々』}
		
			ハヤカワ文庫JA。議論がそこまで複雑ではないこと、ギャグがふんだんに散りばめられていること、かつ円城塔作品を貫く概念である内部観測や補集合的言及といった概念が明確に登場することから、雰囲気に慣れるのに最適。

			数学や物理を専門とする人には『文字渦』よりとっつきやすいかも。

			\subsubsection*{『バナナ剝きには最適の日々』ワンポイントアドバイス}

				定義っぽいことを言っていたら、とりあえずそれを信じて読み進めてみる。

				ギャグまみれなので基本笑って読み流す。

			\subsubsection*{『バナナ剝きには最適の日々』注意点}

				「エデン逆行」は相対論に依拠した作品であり、物理学的に難易度が高い。

				幻想文学として読む分には問題ないが、内容を物理学的に正しく理解するのは難しい。

		\subsection{『これはペンです』}

			新潮文庫。『文字渦』と並んで非常に“文芸的”な雰囲気に見える短篇集。元々《新潮》初出ということもあり、数理ネタについては冗長とも思えるほどの丁寧な説明がつけられている。

			幻想文学や一般文芸が好きな人はこれが一番親しみやすいかも。

			\subsubsection*{『これはペンです』ワンポイントアドバイス]}

				特になし。思うままに楽しもう。

			\subsubsection*{『これはペンです』注意点]}

				特になし。

				ただし、きちんと理解しようとすると途端に凶悪化する(詳しくは後述)。

		\subsection{『Self-Reference ENGINE』}

				ハヤカワ文庫JA。原液の円城塔に挑戦したいという方は、迷わずこの作品からどうぞ。

				圧倒的な速度と濃度で押しつぶされ、話の筋が見えないままにわからされてしまう感覚は唯一無二。
			
			\subsubsection*{『Self-Reference ENGINE』ワンポイントアドバイス}

				普通に読み進めるのが厳しかったら、気楽に読める部分まで流し読みするのが吉。

			\subsubsection*{『Self-Reference ENGINE』注意点}

				冒頭の文章、「全ての可能な文字列。全ての本はその中に含まれている。」という文章は、明らかに偽。

				要するに“『虐殺器官』の大嘘”と同じですよ、ということ。

	\section{円城塔作品における重要概念}

			正直なところ、これらのすべてが複雑に関連し合っているので、厳密には不可分であるような気がする。

			それでも説明の簡単のため、杓子定規ながらいちいち項目を立てて説明を試みる。

		\subsection{内部観測}

			\begin{quote}

				「あくまでもこのわたしの像をつくるのだ。無論、時間の中で変化を続けるこのわたしの像ではない。当然、時間を超えた像となる。頭の先が一瞬前、足の先が一瞬後の姿を写すというような、時間に侵された像は必要ない。ほんの一瞬、そこにあったがゆえに、永遠に存在せざるをえなくなるようなものが望みだ」\hspace{\fill}円城塔「文字渦」

			\end{quote}

			わかりにくいけど最重要。

			内部観測とは、相互作用によって実現される観測と定義される。定義より、内部観測において、観測者は観測の対象と不可分である。相対論的観測・量子力学的観測は内部観測的。

			逆に、外部観測は、観測者と対象が完全に分離しており、観測者は対象の全情報を直ちに観測可能で、観測の前後で対象の系は完全に保存される。いわば、理想的な観測であり、実在しない観測である。宇宙で実現される観測は厳密には内部観測的ではあるが、外部観測に近似される。

			内部観測は、その定義から自己言及的な観測であることが明らかである。

			\subsubsection*{参考作品}
			
			\begin{itemize}
				
				\item「バナナ剝きには最適の日々」

					破壊的観測(観測の前後で対象が不可逆に変化してしまう観測)がモロに登場する作品。基本的にはシュールギャグ作品なのだが、そのギャグ成分はなにか本質的なものを揺さぶっていることへのどうしようもなさ、対抗手段のなさへの諦念からくる笑いかもしれない。

			\end{itemize}

		\subsection{自己言及}

			\begin{quote}

				「私の名は Self-Reference ENGINE。」\hspace{\fill}円城塔「Self-Reference ENGINE」

			\end{quote}

			これも最重要。

			定義は読んで字のごとく、自分自身に言及すること。円城塔のデビュー作のひとつが[『Self-Reference ENGINE』]という題であることからも、その重要性は明らかだろう。
			
			自己言及が様々なバグを引き起こしやすいことは古来よく知られており、古くは“嘘つきのパラドックス”、近年(?)で重要なところだと“Russellのパラドックス”、そして第一不完全性定理の根幹部分(ミニ・ゲーデルの定理)としてその矛盾の実例が知られている。

			自己言及は矛盾を生みやすい“危険”な構造だと判断するのは拙速。プログラミングでは自己言及的(再起的)な定義が多用されているし、自己言及だからといって論理体系が即座に崩壊するわけでもない。知っていれば便利に使えるし、よく研究されているぶんよほど信頼出来る。

			円城塔作品の読解のコツとして、自己言及要素が見えるころには話の大筋が掴めているという経験則がある。内部観測・決定不能性といった要素は基本的に自己言及構造をとることが多く、自己言及構造を手がかりに作品の解析を進めることが出来る。

			\subsubsection*{自己言及に関連するパラドックス}

				\begin{itemize}

					\item 嘘つきのパラドックス

						「私は嘘つきだ」

						厳密には、“嘘つき”という語の厳密な定義がないので、矛盾以前の問題であるとして退けることが可能。

					\item 自己言及のパラドックス

						「この文は偽である」

						この文は厳密に定義されているので、真偽値を決定不可能な命題である。

					\item Russellのパラドックス

						「素朴集合論において、「$ \{ X | X \notin X \} $なる集合$X$をおくと、$ \{ X | X \notin X \} \Leftrightarrow \{ X | X \in X \}$」

					\item Curryのパラドックス

						「この文が真であるなら、Aである」が任意のAについて“証明”可能。

					\item ミニ・ゲーデルの定理

						「真であるのだが、印刷出来ない文章が存在する。」

				\end{itemize}

			\subsubsection*{参考作品}

				\begin{itemize}
		
					\item 「for Smullyan」
					
						ミニ・ゲーデルの定理を直接扱った作品。内容を正しく理解するには、数理論理学の知識が必要かも。

				\end{itemize}

		\subsection{補集合的記述}

			\begin{quote}

				「わたしは一つの包絡線だ。コンパスで描かれた円ではない。中心点を欠いたまま、傾いていく直線たちの集合体だ。何故だか自分は円だと信じる、奇妙な直線たちの集まりだ。」

				\hspace{\fill}円城塔「墓石に、と彼女は言う」

			\end{quote}
			
			ある事物Aについて、直接Aに言及するのではなく、Aの余事象である非Aについて言及し尽くすことで間接的にAを記述すること。

			話はずれるが、SCP-055\footnote{\url{http://scp-jp.wikidot.com/scp-055}}(通称“丸くないやつ”)に円城塔風味を感じるのは多分これのせい。

			\subsubsection*{参考作品}

				\begin{itemize}

					\item 「パラダイス行」

						右回りに一周の長さを測ったときと、左回りに測ったときで長さが変わる不思議な洞窟をめぐる話\footnote{【後日追記】周回積分では? という指摘をいただきました。確かに、これは周回積分以外の何物でもない。}。

					\item 「烏有此譚」

						主題は圏論における「米田の補題」に基づいているらしいのだが、それはそれとして至る所でこの補集合的記述が見られる(ポンペイの下りとか)。

				\end{itemize}

		\subsection{決定不能性}

			\begin{quote}

				「何かが理解可能であるかどうかは、科学の範疇には属さない。理解できるものを理解するのが、科学の役割であるからだ。この言が不適切だとされるなら、理解できたものしか語らないのが科学であるから。原理的に理解のできないものに対しては、せいぜい理解不可能であると理解するのが科学の上げることのできる得点だ。科学は理解不可能なものを理解可能な対象へと切り替える機能を持っておらず、これはあまりにも当然すぎる。」\hspace{\fill}円城塔「ガベージコレクション」

			\end{quote}

			どのような問題を与えられても、問題を解くことができないこと。例えば、先述の自己言及に伴う各種のパラドックスは(素朴には)決定不能命題。

			直近では、熱平衡するかどうかという問題と計算機が停止するかどうかという問題(停止性問題)が等価であるという証明がなされ、一般の物質が熱平衡するかどうかという問題が決定不能であるということが証明された。(停止性問題が決定不能であることはTuringによって既に証明されている)

			\subsubsection*{参考作品}

				\begin{itemize}
				
					\item 「内在天文学」

						観測するたびに姿を変える対象について、私たちはその真の姿を知ることは出来るだろうか? というお話に、ボーイミーツガールと謎の叙情性をトッピングしたエモエモ短篇。

					\item 「墓の書」

						創作中の人物の死と、その墓に関する考察。いわゆる後期クイーン的問題の問題意識に近いか。生きている作中人物はいずれ死ぬはずであり、あるいは生まれてきたはずである。テクストには生きている時点での情報しかないにもかかわらず生誕と死亡が考えられるのは不思議なことだ、という話を延々とまぜっかえす。

					\item 「Japanese」

						\footnote{原資料において記述が欠落している。}

				\end{itemize}

		\subsection{双対性}

		\begin{quote}

			「右があるって信じるならば、左もあると信じるべきだ。」\hspace{\fill}円城塔「パラダイス行」

		\end{quote}

			2つのものが対の性質を持っていること。

			\subsubsection{参考作品}


				\begin{itemize}

				\item 「ガベージコレクション」

					男$ \leftrightarrow $女、時間順行$ \leftrightarrow $時間逆行、情報$ \leftrightarrow $ガベージデータという双対関係が頻出する作品。本作では、情報論的エントロピーに関する情報熱力学的考察から、時間の進行と“過去”の忘却が等価であることを導く。要するに、時間の進行と完全記憶はトレードオフということ。たとえ忘れたことすら忘れてしまったとしても、不可逆に記憶を失い続けながら、私たちは時間を生き続けなければならない。寂寥と郷愁の入り混じるエモエモ短篇。

				\item 「十二面体関係」

					\footnote{原資料において記述が欠落している。}

				\end{itemize}

	\section{作品解説その他の文章}

		\subsection{「$\varnothing$」と「文字渦」:内部観測を軸に}
			
			\subsubsection{「$\varnothing$」簡易解説}

				この作品は、作品宇宙の終焉を、その宇宙内部から観測した作品として解釈可能。

				物理学の最終目標は、この宇宙全てを記述する法則を得ることである\cite{Hawking}。

				作中では、宇宙が$ \varnothing$に落ち込んでしまうことで、宇宙の完全な記述が達成されてしまった($ \because \varnothing$はクワイン\footnote{その文字列を実行したとき、自分自身を出力する文字列のこと。一般にクワインを構成することは困難だが、最も単純な自明なクワインとして$ \varnothing $(空集合)が知られている。})。

				しかしながら、作中における物理学の完成を、作中の人物たちは知りようがない。完全な物理学の完成という至上の瞬間を、宇宙自身すら知る術はない。これがとにかく悲痛であり、エモエモ。

			\subsubsection{「文字渦」簡易解説}

				始皇帝が甬に出題した問題は「始皇帝の姿を真人として構成せよ」であった。それに対する甬の解答は、「“始皇帝の存在しない秦”の俑の作成」であった。

				これは作中で試行錯誤を経て提示されたアイデアであり、SF的にいかにも面白い解答であるが、実は自明な解答である。これからそれを実際に示す。

				始皇帝の問題を数学的・物理学的に整理すると、「時間発展する始皇帝$ S_{Emperor} $について、観測を表す0でない演算子$ O $を作用させたとき、常に始皇帝と同じ観測結果となるような俑$ S_{figure} $を構成せよ」となる。

				つまり、連立方程式$ \begin{cases} O S_{Emperor} = A \\ O S_{figure} = A \end{cases} $を満足する$ S_{figure} $を求めることに帰着される。

				一般に、この方程式の一般解を求めることは困難だが、$ O \neq 0 $であるから$ S_{Emperor} = S_{figure} = 0 $が方程式を満足することは自明。この自明な解は“始皇帝の俑を作らない”と解釈される。実際、$ S_{figure} = 0 $は時間tに依らない。つまり、はじめから俑を作らなければ滅びない。

				このとき、$ S_{figure} $が出力するのは“なにもない”であるので、これもまたクワインである$ \varnothing $と解釈出来る。

				ということで、「$ \varnothing $」と「文字渦」が、ともに内部観測とクワインを軸に構成された作品であることが理解される。

			\subsubsection{「$ \varnothing $」「文字渦」の統一的解釈}

				要するに、両作が扱うのは、原理的に記述しえないもの(無限、時間発展)を記述するにはどうすればいいですか、ということ。

				ひとつは、語りたい対象と語り手を同じサイズにすること。観測者と観測の対象が同じサイズになれば、自然と内部観測的になる。『Self-Reference ENGINE』の超知性体たちが、自然現象を解析しようとして自然現象そのものになってしまったのは、これが原因。

				もうひとつは、語りたい対象を自明なレベルまで削減してしまうこと。時間発展する系として自明なクワインを採用すれば、内部観測的にもなる。無限を語りたいなら、無限を$ \varnothing $まで削減するか、自分自身が無限になるか。

				ぱっと見では題材も時代も規模もまったく異なるのだが、実はその根幹をなす論理は全く同一の構造に依っていることがわかる。

		\subsection{『Self-Reference ENGINE』と「$ \varnothing $」、そして『攻殻機動隊』の草薙少佐:SRE=草薙素子説}

			士郎版・押井版双方の最初のシーン、少佐が暗殺後にビルから飛び降り、光学迷彩で夜に溶け込んでいくシーン。これのイメージ。
			
			存在を探知されてしまった少佐は、光学迷彩で夜に見えるようにふるまい、夜景と同化してしまう。これがSREが見えなくなってしまう(いなくなってしまう)構造と全く同じ。

			\begin{itemize}

				\item 攻殻:荒巻$ \rightarrow $光学迷彩(夜景にしか見えない)/少佐、背景(夜景)

				\item SRE:読者$ \rightarrow \varnothing$(無にしか見えない)/SRE、背景(無)

			\end{itemize}

			問題なのは、『SRE』の場合、$ \varnothing $とSREが全く同じ存在であり(なぜならクワインなので)、かつ背景の無と論理学的に区別がつかないというところにある。そもそも、少佐は自分の意思で光学迷彩を使っているが、SREは機械論的に、完全に決定論的に作動してしまうという違いもある。

			ちなみに、真の扱いたいものの手前に原理的制約による“壁”がある、という構造を持つ作品は非常に多いあるいは、なにかを求めるとなにかを失うというトレードオフ関係をモチーフにした作品であるとも言える。

		\subsection{外部観測から内部観測への転回としての「良い夜を持っている」}

			はじめ、語り手は“父”という対象の観測者として振る舞う。超記憶をもつ“父”は“記憶の街”を構成し、街中にオブジェクトを配置することで記憶を維持していた。

			物語の最終行において、“姪”はいつしか手にしていた赤いビー玉を手放す。この描写によって、作中世界が“父”の“記憶の街”と同一であることが明示される。赤いビー玉は、“父”が“記憶の街”で用いていた、記憶対象を示すポインタにほかならない。ついでに、「良い夜を持っている」作中世界が「これはペンです」と同一であることも導かれる。

			完全な外部観測者として“父”を記述していたはずの語り手は、ここで自身も“父”の“記憶の街”の住人であり、観測される対象であったということに気づく。つまり、“父”について言及することは、“わたし”についての自己言及にほかならない。自己言及であり、内部観測である。

			この世の観測は、厳密にはすべて内部観測である。にもかかわらず、物語として提示されると驚くってのは不思議なことのように思う。一言で言えば、ディック的。現実が妄想と同一だったことを知る恐怖。それと同時に発生する文芸的大団円。この不協和音が魅力の原点にあるのではないか。

			内部観測云々はともかく、“父”が“記憶の街”で“母”と再会する光景を“わたし”が見届ける、という情景はエモエモだし、非常に文芸的でわかりやすい形式。この作品が円城塔作品の中でも上位人気なのは、これが原因なのではないか。ずっと難しくてわからないことを並べて読者を抑圧して、最後のエモエモフレーズでオトす、そういうギャップ萌え要素が多大にあると思う\footnote{円城塔本人のコメントあり \\ \url{https://scrapbox.io/enjoetoh/%E3%83%8F%E3%83%A1%E6%89%8B%E3%82%92%E6%8C%81%E3%81%A4%E3%81%93%E3%81%A8}}。

		\subsection{内部観測と外部観測の入り混じる「ムーンシャイン」}

			比較的こちら側の語り手である“僕”の語りと、完全にあちら側の“私”の語りが概ね交互に繰り返される形で物語が展開される。ある種の数学的超能力を持つ少女と、その少女を悪の組織から守ろうとする少年のボーイミーツガール。

			確固たる自我があり、かつ旧来的な語りを用いる少年は外部観測。観測者と観測の対象の区別が不明瞭で、多重共感覚の十分複雑なネットワークの発火として実現される少女は内部観測(厳密には、おそらく少女はネットワークそのもので、発火は少女を走るインターフェース)。

			数のみによって構成された“私”の視点だけではまともな小説の構成が困難。読者へのインターフェースを噛ませてやらねばならず、本作ではそれが“僕”の視点。“私”$ \rightarrow $“17”と“19”$ \rightarrow $“僕”$ \rightarrow $読者、という多重インターフェース。

			問題は、“僕”から読者への間でほとんど連絡がないというところか。群論を学部の専門科目として受講したことがないと少女のなにがヤバいのかを理解出来ないと思う。

			少女がヤバいのは、巨大数を直感的に扱えるところ。本来ならば人間の扱えるオーダーを超えた数を、片手で数える感覚で扱えるのは異次元にも程がある。作品冒頭で、“僕”が「1911なんて数はよくわかんねぇ、1729なら多少語れるけど」的なことを言っているのは、これの暗示。無論、「お前も十分すごくない?」というギャグであるとも思う。

	\section{まとめ}

		自ら変化することの出来ないような、例えば紙に印刷されたテクストは、死んだテクストだ。テクストの記述する対象は時間発展するが、一般の固定的なテクストは時間発展しない。

		これを超えるためには2つの方法がある。

		ひとつは、時間発展するテクストの記述方法を見つけること。インタラクティブな記述の開発。これは自己言及的な解決方法だと言える。

		もうひとつは、時間発展する読者も含めて、読書体験と定義すること。時間発展する読者がテクストに作用するという系を考えれば、テクストから得られる読書体験は時間発展する。(量子力学がわかる人向けには)Schrödinger描像とHeisenberg描像の文学へのアナロジを考えてほしい。テクストと読者のどちらが時間発展項を持つべきかという議論は本質的ではない(どっちだろうがどうとでも解釈出来る)。これは内部観測的な解決方法だと言える。

		\begin{itemize}
		
			\item 内部観測:私の語るものは、私。
		
			\item 自己言及:私を語ることで、私を超えていく。
		
			\item 補集合的記述:私以外をすべて書き尽くすことで、私を得る。
		
			\item 決定不能性:私にだって、わからないことはある。

		\end{itemize}

	\section{質疑応答}

		\begin{itemize}

			\item Self-Reference はプログラミングの自己参照型構造体からなのでは?

			\vspace{1mm}

			 もちろんそれもありますが、より本質的には自己言及という算術構文そのものから来ていると思います。円城塔の専門分野では特に頻出する概念です(内部観測、Curry化、自己参照関数、etc.)。

			\vspace{2mm}

			\item ミニ・ゲーデルの定理について教えてください

			\vspace{1mm}

			 私のブログ\footnote{\url{https://shiyuu-sf.hatenablog.com/entry/smullyan}}で一応証明しています。

			 より正確で厳密な記述をお求めの方は、『スマリヤン数理論理学講義 上』(日本評論社)の223頁問題1を参照してください。本書は学部専門レベルの教科書なので、事前に学部教養レベルの集合論や数理論理学を学習することをお勧めします。

			\vspace{2mm}

			\item ミニ・ゲーデルの定理は集合論に持ち込めば解決可能なのでは?

			\vspace{1mm}
			
			 ミニ・ゲーデルの定理を(素朴)集合論に置き換えると、Russellのパラドックスに一致します。つまり、公理集合論では定義不能です。

			\vspace{2mm}

			\item 外部観測は理想測定と同じですか?

			\vspace{1mm}

			 同じです。内部観測では、観測者と観測の対象が不可分である、という特徴が本質的です。

			\vspace{2mm}

			\item 量子力学と外部観測・内部観測は関係がありますか?

			\vspace{1mm}

			 あるにはありますが、ほとんどの場合、多世界解釈を扱う文脈で登場するので、距離を置くのが正しいかと思います。多世界解釈は今日では量子情報論によって否定的に捉えられています。反証可能性もなく、科学理論としては信頼性の低い議論です。
			
			 というか、そもそも外部観測・内部観測を扱う議論が物理学的にかなりインチキくさいので、下手に手を出すと火傷するかも……。

			\vspace{1mm}

			\item (「$ \varnothing$」について)自明な解が意味を持つのは隠れた変数がないときですか?

			\vspace{2mm}

			 そうだと思います。とはいえ、隠れた変数の存在を確認する実験について、私はなにも思いつかないのですが……。

			 Bell不等式は理解していますが、「$ \varnothing$」の世界で実現可能な実験を思いつきません。

			\vspace{2mm}

			\item 「良い夜を持っている」の数理的な部分はどこですか?

			\vspace{1mm}

			 明確に数理的だと断言出来る部分はないかと思います。数理的なモチーフを直接利用することが多い円城塔が、高度な数理的価値転換を行いつつ文芸的作品に仕立てている、というのが重要だと思います。

			 あるいは、数理的モチーフを自然に文芸作品に織り込むことに成功したことで、この系統(『Boy's Surface』〜「ムーンシャイン」〜「良い夜を持っている」)は完成されたのかもしれません。

			\vspace{2mm}

			\item 自己言及で(粗雑に)まとめている部分が多くないですか?

			\vspace{1mm}

			 ご指摘の通りです。しかしながら、自己言及としてまず粗く括ってみることで、そもそもある程度体系的に分類出来ることを示すことが重要だと考えています。

			 自己言及が円城塔作品に頻出するモチーフであることは事実であり、また従来自己言及が登場しないと考えられていた作品群(「$ \varnothing $」、「文字渦」など)に自己言及のバリエーションが見えると指摘出来たことは今回の成果であると考えています。

			 粗く括ってしまえば、不完全性定理も内部観測も自分語りも、自己言及と言い切ってしまうことが出来、これがあまり行儀のよくない読みであることも理解しています。それでも、とりあえず括ってみられることを示すことが、なにより重要であると思います。

			 一方で、研究が進めば、より的確な表現によって分類が可能になるとも考えています。私が今回お話ししたことは、あくまで発展途上の分野の中間報告であって、完成された体系ではありません。迷走しつつも真理の一端をなんとか解釈しようともがいていた前期量子論を思い浮かべていただければと思います。

			 とにかく、円城塔作品に対する数理的読解は、これまで試みられたことが少ない営みであり、今まさに発展しつつある不完全な取り組みです。機会を見つけて成果を公開しつつ、様々な議論を通じて発展させていきたいという次第です。

			 今はまだ円城塔作品にしか試みていませんが、他の作者の作品に同様のアプローチを試みること、そして数理的読解を想定した創作の模索など、この試みから派生する文芸的フロンティアはまさに無辺であると思います。

		\end{itemize}

	\section{【後日追記】コメント返し}

		\begin{itemize}

			\item 内部観測という言葉が、科学用語としての適用範囲を越え、比喩として濫用されているのではないか。

			\vspace{1mm}

			 おっしゃる通りです。

			 ひとつ言い訳をするとすれば、このような読みを行なっているのは観測範囲では私1人だけ(言い過ぎか?)であり、新しいおもちゃを与えられてとりあえず遊びまくっているという感覚があります。私が孤独に行なっている活動を、広く知ってもらおうというのが今回の企画の目的でもあったので、協力していただける方が増えることが私の望みです。私個人では気づかないこと、個人であるが故の議論不足、読みすぎなどを、人数を増やすことで少しずつ解消して行けたらと思います。個人の力だけでは、到底限界がありますので。

			\vspace{2mm}

			\item 面白かったです(多数)

			\vspace{1mm}

			 ありがとうございます。今後も継続的に取り組んで、また面白い発見があったら企画を持ち込もうと思います。

			 ブログなどでも研究成果の発表を進めていこうと思いますが、いずれきちんとした論文や批評として発表したいと思っています。いつになるかは不透明ですが。

			\vspace{2mm}

			\item これを本にまとめてSF大賞を狙いましょう。

			\vspace{1mm}

			 ありがとうございます。このコメント、個人的に一番嬉しかったです。

			 今私が行なっていることは、一般的な研究活動で言えば研究ノートレベルに過ぎません。先行研究もありませんし、研究の質を担保するためには、より数理的に厳密な定義・議論とピアレビューが必要不可欠です。それには時間も人間も足りませんし、なにより力も足りません。少しずつでも地道に実績を積み上げていって、本なりなんなりにまとめてパブリッシュ出来ればな、と思っています。

		\end{itemize}


	\begin{thebibliography}{99}

		\bibitem{Hawking} スティーヴン・ホーキング, 『ホーキング、宇宙を語る』, ハヤカワ文庫NF, 早川書房, 1995
		
		\bibitem{Smullyan} レイモンド・M・スマリヤン, 『スマリヤン数理論理学講義』, 日本評論社, 2017-2018

	\end{thebibliography}


	\section*{注記}

		本資料は、2021年に開催された京都SFフェスティバル2021で行われた「円城塔作品を語る部屋」のために制作された2021年10月頃成立の原資料\footnote{\url{https://scrapbox.io/allreferenceengine/%E4%BA%AC%E3%83%95%E3%82%A7%E3%82%B92021%E5%86%86%E5%9F%8E%E5%A1%94%E9%83%A8%E5%B1%8B%E8%B3%87%E6%96%99}}を2023年6月にPDFとして再編集したものである。

		参考文献や出典については後日補った情報が多いが、全て成立当時には利用可能であったことに注意されたい。

		それ以外の箇所については、特に断りのない限り、欠損部分も含め、全て原資料が成立した2021年10月当時の記述をそのまま転記している。

\end{document}