\documentclass[10pt, a5paper, twoside]{jsarticle}

\usepackage{okumacro}
\usepackage{enumitem}
\usepackage{amssymb}
\usepackage{amsmath}{}
\usepackage{amsfonts}
\usepackage{amsthm}
\usepackage{bm}
\usepackage{url}
\usepackage{here}
\usepackage[dvipdfmx]{graphicx}
\usepackage{wrapfig}
\usepackage{makeidx}
\usepackage{braket}
\usepackage{fancyhdr}
\usepackage[top=20truemm,bottom=20truemm,left=15truemm,right=15truemm]{geometry}

\pagestyle{fancy}
	\fancyhead{}
	\fancyhead[RE]{円城塔研究}
	\fancyhead[LO]{資料:コンメンタール「文字渦」}
	\fancyhead[LE, RO]{\thepage}
	\fancyfoot{}
	\fancyfoot[LE, RO]{\footnotesize{The Journal of EnJoeToh Research, Vol.1, No.1, 2023} }

\theoremstyle{definition}
	\newtheorem{dfn}{定義}
	\newtheorem{thm}{定理}

\begin{document}

	~ %強制改行

	\begin{center}

		\Large{コンメンタール「文字渦」}

		\vspace{3mm}

		\large{Commentary of short story \textit{Mojika}}

		\vspace{3mm}
		
		\large{下村思游}

	\end{center}

	\vspace{3mm}

	\section{解説}

		頁数・行数は『文字渦』文庫版・紙版\cite{mojika}に準拠した.

		\subsection{題名}

		無論,中島敦の短篇「文字禍」が元ネタ.

		\subsection{p.11, l.12-13}

		\begin{quote}
			
			「引き延ばしては折りたたみ,折り畳んでは引き延ばし,〜」
		
		\end{quote}
		
		パイこね変換が思い出される.(古典)パイこね変換とは,以下のように定義される\cite{pie}.

		\begin{dfn}

			古典パイこね変換
			
			古典パイこね変換は,単位平面\footnote{1辺の長さが1である正方形.} $0 \leq q, p \leq 1$を単位平面自身に変換するもので,以下の写像で定義される.
			
			\begin{equation*}
				(q, p) \rightarrow \begin{cases} \Big{(}2q, \displaystyle \frac{p}{2} \Big{)} & \Big{(} 0 \leq q \leq \displaystyle \frac{1}{2} \Big{)} \\ \Big{(} 2q-1, \displaystyle \frac{p+1}{2} \Big{)} & \Big{(} \displaystyle \frac{1}{2} < q \leq 1 \Big{)} \end{cases}
			\end{equation*}

		\end{dfn}

		この変換は,$p$方向に面積を保存したまま単位平面を押しつぶし,$q$方向にはみ出た分を切り取って残りの部分の上に載せる操作に相当する\cite{pie}.このパイこね変換の定義は単純であり,かつ変換自体も容易だが,複数回繰り返すことで極めて複雑な結果が得られることでも知られている\cite{yama}.

		このパイこね変換はその名の通りパイ生地をこねるときの操作に酷似している.単純なパイこね変換から複雑な結果が得られることは,実際にパイ生地をこねれば均一の生地が得られることからも想像出来るだろう\footnote{厳密には,因果律は逆で,パイ生地をこねると均一になる現象の数理モデルとしてパイこね変換が考えられた.}.

		\subsection{p.12, l.13-14}

		\begin{quote}
			
			「裏表のない輪であるとか,中空の真球だとか,〜」

		\end{quote}

		前者はメビウスの帯,後者は球殻を指す.

		帯状の長方形をおく.この片方の端を180度反転させて反対側の端に接続させて得られる曲面をメビウスの帯(Möbius band),あるいはメビウスの輪(Möbius loop)という.このメビウスの帯は,表裏を決定出来ない\footnote{曲面上のある地点から一方向になぞっていくと,“2周”回って元の位置に戻ってしまう.すなわち,“表”を進んでいたはずがいつの間にか“裏”に回っており,再び“表”に戻ってくる.このことを数学では「向きづけ不可能」と言い,メビウスの帯は向きづけ不可能曲面である.}ことが知られている.

		メビウスの帯,クラインの壷,ボーイ曲面(Boy's surface\footnote{少年の表面,ではなくこの図形の発見者である数学者ヴェルナー・ボイ(Werner Boy)に由来する.無論,「Boy's Surface」の元ネタである.})は互いに深く関係しあっている.先ほど,メビウスの帯の紹介で“接続”という言葉を用いたが,これは数学においては“同値類で割る”という語が当てられている.メビウスの帯は長方形の2対の辺の組のうち1対のみについて“割った”が,2対とも“割る”とクラインの壺が得られる.こうして同値関係で割ることによって得られるメビウスの帯(に“蓋”をして閉じたもの)とクラインの壷は,数学的には2次元実射影空間$\bm{RP}^2$と位相同形である.この$\bm{RP}^2$は3次元空間$\bm{R}^3$へはめ込み\footnote{「はめ込み」は厳密な数学的定義語なのだが,その定義は極めて誤解を招きやすいと判断したためここでは紹介しない.現時点では,メビウスの帯,クラインの壷,ボーイ曲面が互いに深く関係していることを知っていただければそれでよい.厳密な定義は文献\cite{mat}を参照されたい.}可能なことが知られており,その具体的な構成がボーイ球面である.

		球殻(sphere)とは,球の表面(球面)のことである.球殻には厚さはなく,自分自身をすり抜けることを許すならば,連続的な操作の下でその表面と裏面を入れ替えること(球面を裏返すこと)が可能であることが知られている\cite{smale}\footnote{余談だが,スメイルによる証明\cite{smale}は高度に抽象的であり,球面に対する実際の操作はほとんど論じられていなかったところ,実際の操作の視覚的な構成が盲目の数学者モラン(Morin)らによって与えられたことは大変興味深い.なお,モランらによる操作では,その操作の途中でボーイ曲面を経由する.}.

		また,選択公理\footnote{(空集合でない)集合を集めてきた(空集合ではない)集合族$ { \{ A_{\lambda} \} }_{\lambda \in \Lambda}$に対して,その集合族の各要素である各集合からそれぞれその集合の元$ a \in A_\lambda $を1つずつ選んできた集合が存在することを認める公理\cite{hara}.これが認められることは至極当然のようであり,実際そうなのだが,直感に反する数学的事実を次々と導出することが知られている.1から2を生成するかのようなバナッハ--タルスキの定理はその最たる例である.}を認めるならば,3次元球殻を有限個の部分に分割し,元の球殻と全く同一の球殻2つを再構成することが可能\footnote{すなわち,球殻1つから同じ球殻をいくらでも作り出すことが出来る.なお,元の論文\cite{bantar}では球体に対しての操作であり,球殻の操作ではないことに注意.}であることが知られている.これをバナッハ--タルスキの定理という.ここで,バナッハ--タルスキの定理に至る直前の定理としてシェルピンスキー--マズルキーウィチの定理\footnote{「エデン逆行」に登場するシェルピンスキー・マズルキーウィチ辞典の元ネタ.この定理は,自分自身をいくつかに分割してその一部を移動させることで,自分自身と同じ分身体を構成することが可能であると主張する.}というものがあり,これに選択公理を導入することによってバナッハ--タルスキの定理が導出される\cite{math}.

		\subsection{p.13, l.9-10}\label{eisei1}

		\begin{quote}

			「俑に可能な反抗は粘土で形づくることができるものに限られており,粘土の形で反抗の意を表す方法はわからなかった.」

		\end{quote}

		重要な伏線.結局のところ,俑は物理学的な考察を経て,粘土によって始皇帝への反抗を高らかに示すことになる.

		\subsection{p.18, l.1-2}

		\begin{quote}

			「漢字の自由すぎる繁殖を恐れた康熙帝〜」

		\end{quote}

		本作のみならず,連作『文字渦』において,文字は当然繁殖する.

		後世に残るものとは,よく保存されるものではなく,よく複製されるものである.よく保存されたとしても,保存されている数が少なければ不意の災害などで簡単に滅んでしまう.一方,よく複製されるものは,多少その細部が変化しようとも,大勢としてはその姿形を留めているだろう.もし一部の集団が大きく変化して原型を留めなくなったとしても,そうでない集団が元の姿形を伝えていくことだろう.このような発想は,後続の「微字」によく現れるほか,円城塔作品の様々なところで暗黙の了解として扱われている\footnote{例えば,複製の容易性については「天書」,複製の可逆性・不可逆性については「ガベージコレクション」.}.

		この発想の源流として,ドーキンスが『利己的な遺伝子』で辿ってみせた遺伝子から模倣子への流れの逆,遺伝子のように自己複製を通じて増殖していく情報というアイデアが挙げられる.ここでは,クワイン\footnote{自分自身と全く同一な文字列を出力する文字列.同様の構文を初めて公に使用したとされる哲学者・論理学者W・V・O・クワインにちなんで,物理学者ダグラス・ホフスタッターによって名付けられた\cite{geb}.}などの自己言及的な構文も重要な役割を果たす.

		\subsection{p.23, l.6}

		\begin{quote}

			「嬴である.」

		\end{quote}

		本作の主要登場人物,始皇帝嬴政の初登場.この“嬴”字は登場ごとに微妙に字面が異なることに注意.

		\subsection{p.25, l.9-12}

		\begin{quote}

			「あくまでもこのわたしの像をつくるのだ.(中略)ほんの一瞬,そこにあったがゆえに,永遠に存在せざるをえなくなるようなものが望みだ.」

		\end{quote}

		ここで,始皇帝は,本作で解決するべきSF的問題を提起する.結論,始皇帝は,外部観測によって常に始皇帝本人と同じ状態を実現する系を厳密に構成せよと主張している.これは一般には原理的に実現不可能な主張である.以下,始皇帝の主張がいかに無理難題であるかを物理学的に解説する.

		まず,外部観測とは,対象となる系の情報について,その全体を瞬間的に全て得ることが出来(=非相対論的),観測の前後で系の情報は破壊されず(=非量子力学的),対象となる系と観測者である自分自身が完全に分離されている(=外在的)観測のことである.平たくいえば,これは極めて不自然かつ都合のいい観測であり,非現実的な観測である.われわれが普段素朴に行う観測は,この外部観測に近似される.日常生活における観測行為が非現実的な観測である外部観測に近似されても特に不都合がないのは,日常生活が非相対論的かつ非量子力学的かつ外在的な状況に限られているからである.

		一方,内部観測とは,対象となる系の情報について,その全体を得るには部分を少しずつ集めていかなければならず\footnote{無論,情報をこつこつと集めている間にも,系は時間と共に変化してしまい,最初の方に集めた情報は既に古びている.}(=相対論的),観測の前後で系の方法は不可逆に破壊されてしまい(=量子力学的\footnote{この量子力学的観測(破壊的観測)が印象的に登場する作品として,「バナナ剝きには最適の日々」が挙げられる.}),対象となる系と観測者である自分自身が不可分である(=内在的)観測のことである.内部観測による観測では,知りたい系の情報は相対論的・量子力学的制約の下でしか得られず,また観測によって知りたかった系の情報が不可逆に破壊されてしまう.観測によってその系の情報を知ってしまったがために,その系が観測の直後にどのような情報をもっているのか,あるいはその情報がどのように変化していくのかを知ることが出来なくなってしまうのである.

		ここで,始皇帝の主張は2段階に分割出来る.すなわち,“始皇帝を外部観測によって観測せよ”と“外部観測によって得られる時時刻刻と変化する情報を全くそのまま反映する像を作成せよ”.

		前者について考えると,そもそも外部観測とは理想的な観測であって現実では実現不可能なものであるから,初手から実現不可能である.これに加えて,後者では常に変化し続ける情報を全く転写するような像を構成せよという.例え始皇帝の完全な情報が得られたとしても,コンピュータのように通信が光速によって制限される相対論的な装置\footnote{例えこのコンピュータが一般的なフォン・ノイマン型計算機ではなく量子計算機であったとしても,その通信速度は光速に束縛されているので状況は変わらない.}では始皇帝の望む像は構成出来ない.

		このように,始皇帝は物理学的に一般には実現不可能な要求を多重に行っている.しかし,本作において,俑は物理学的考察を通じて特殊な例として始皇帝の要求を満たす像を構成してみせる\footnote{詳細は「「文字渦」の難解性:内部観測と量子複製禁止定理」\cite{sshimo1}を参照されたい.}.始皇帝の主張が物理学的・数学的に実現不可能なものであること,そしてその解法が数学的・物理学的に正しく鮮やかなものであることを確認することは,極めて重要な取り組みである.

		\subsection{p.26, l.4-5}

		\begin{quote}

			「しかし,時の中に不滅のものは御座いません」
		
			「不滅を滅ぼす矛を用いて,滅を先に滅するのだ」

		\end{quote}

		始皇帝は,自身の主張が実現不可能な矛盾であることを知りながら,それを強いている.

		\subsection{p.27, l.9-12}

		\begin{quote}

			「俑の前に現れる嬴は見れば見るほど,とらえどころのない人物である.くるくると印象が変転してとどまらない.おおよその形をとって目を上げると,そこにいるのはもう別人である.」

		\end{quote}

		\ref{eisei1}で論じた通り,始皇帝を示す“嬴”字は登場ごとに変化している.これは,本作において,漢字とその表現対象が1対1に厳密に対応していることによる\footnote{詳細は次号掲載予定の「「文字渦」における漢字とその表現対象の1対1対応」で論じる.}.すなわち,始皇帝が変化すれば始皇帝を表す“嬴”字は変化し,逆に始皇帝を表す“嬴”字が変化すれば始皇帝がつられて変化してしまう.

		白川静によれば,漢字とは「実在の世界と不可分にして対応する」ものであり,「一語を一字をもってあらわす」原則がある\cite{kanji}.短篇「文字渦」および連作『文字渦』全体を貫く公理である漢字とその表現対象との1対1対応は,この白川の主張のSF的な誇張であると推察される.また,塚本邦雄\footnote{正しくは,「塚」「邦」は旧字体を用いる.使用している環境ではうまく表示できなかったため,ここでは新字体で代用した.}が「正字」(旧字体)を重視し,正字と新字体の差異に注意を払うよう警告していた\cite{tsuka}ことも連想される.

		\subsection{p.29, l.15}

		\begin{quote}
			
			「零だ」
		
		\end{quote}

		この始皇帝のセリフは,自らが出題した問題が自明な解しかもたないことの示唆としても読める.

		\subsection{p.30 l.2}

		\begin{quote}
			
			「阿語生物群」

		\end{quote}

		古生物学におけるエディアカラ生物群,バージェス頁岩動物群,\ruby{澄江}{チェンジャン}生物群などに由来する言い回しか.

		エディアカラ生物群はエディアカラ紀\footnote{今から約6億3500
		年前〜約5億4100万年前\cite{ken}.},バージェス頁岩動物群と澄江生物群はカンブリア紀\footnote{今から約5億4100万年前〜約4億8500
		万年前\cite{ken}.}の化石から発見された生物の総称で,いずれも極めて多様な姿形をもつ生物種が爆発的に登場し,突如として消滅したことを示している.

		\subsection{p.34, l.13-15}

		\begin{quote}

			「エイ\footnote{“エイ”は正しくは“嬴”字が微小に変化した字.他の“エイ”についても同様.}といい,エイといった./エイといい,エイという./エイといい,エイという.」

		\end{quote}

		本作におけるクライマックスだろう.ここで明らかに“嬴”字が変化していることを読者に示し,文字に対する注意を促す.以降,連作『文字渦』では文字を用いた文体芸が多用される.

		\subsection{p.35, l.13}

		\begin{quote}
			
			「真人の姿を一体の像へ移すことは叶いません」

		\end{quote}

		俑が最終的に示す解答は既にここに現れている.

		\subsection{p.37, l.4}

		\begin{quote}

			「姿形はそっくりなのだが,〜」

		\end{quote}

		“作られた”人物たちの過去や未来に関する考察は,のちの短篇「墓の書」\footnote{『新潮』2021年9月号掲載.後期クイーン的問題に近い立場から,フィクションに登場させられてしまった登場人物たちの誕生や死,そしてその死後に建てられるであろう墓について考察する作品.}における創作中の人物の死とその墓に関する考察への関連が指摘される.また,本作や「墓の書」に見える生死や性別といった生物学的な属性はのちに削ぎ落とされ,「ローラのオリジナル」\footnote{『紙魚の手帖』Vol.12掲載.画像生成AIに生成された“わたしのローラ”なる画像について,その来歴や行末を考察する作品.}に再び姿を表す.

		対象の間に矢印があるとき,それらの記述や解析には圏論を用いることが出来る.人形たちの“血縁関係”についても,圏を用いて記述を試みる価値がある.なお,圏論は『烏有此譚』において米田の補題が象徴的に用いられた実績がある.

		\subsection{p.37, l.8-9}

		\begin{quote}

			「起こりうることはあらかじめ知られ,しかもいつか必ず起こる.」

		\end{quote}

		物理学的に可能な現象は必ず実現される\footnote{似たような言葉がヴェルヌ,アインシュタイン,ディズニーなどに仮託されているのをよく見かけるが,その典拠は相当怪しい.},という物理学における暗黙知を示唆している.同様の記述が「バナナ剝きには最適の日々」\footnote{ハヤカワ文庫JA『バナナ剝きには最適の日々』収録.発狂した計算機の独白を通じて,内部観測的考察のエッセンスを示す.「想像が可能なことは,いつか実現されるんじゃなかったろうか.」p.47}や「ガベージコレクション」\footnote{ハヤカワ文庫JA『後藤さんのこと』収録.可逆計算と情報熱力学を手がかりに,時間の流れが存在することは忘却と等価であると主張.「思考可能なものは,実現される.」p.175}にも見られることから,円城塔作品においても暗黙知として機能していることが示唆される.

		\subsection{p.37, l.14-16}

		\begin{quote}
			
			「皇帝なき秦の姿こそが真人の像であるというのが俑の答えで,存在しないものは滅びない.嬴はこの解答を受け入れた.」

		\end{quote}

		俑は物理学的・数学的考察に基づき真人の像を構成し,始皇帝も俑の解答を受け入れた.ここから,始皇帝が天子たる大器の持ち主であることや,自身の問題への数学的・物理学的に正しい解答を正しいと判断出来る素養のある人物であることが読み取れる.

		なお,現実の世界を何かに転写した結果,元の世界と転写先の模倣世界が区別がつかなくなる,というアイデアは過去に「良い夜を持っている」\footnote{新潮文庫『これはペンです』収録.完全記憶を持つ父の足跡を辿ることで,父を理解しようとする.}で用いられている.「良い夜を持っている」において,物理学的・数学的に正しくかつ興味深い議論が行なわれている上,それが文芸的な表現に昇華されていることは大変興味深い.

		\subsection{p.38, l.1-2}

		\begin{quote}

			「阿房宮が炎上し,その炎によってようやく真人の像が完成するのは,まだもう少し先の出来事である.」

		\end{quote}

		焼物を焼き上げるイメージとともに,炎によって洗い清められることで完全になる,あるいは秦は滅ぼされることによって永遠となる,という主張が読み取れることにも注意したい.

	\begin{thebibliography}{99}

		\bibitem{mojika} 円城塔, 『文字渦』, 新潮文庫, 新潮社, 2021

		\bibitem{pie} 井上啓, 大谷雅則, I. V. Volovich, パイこね変換の量子--古典対応について, \textit{数理解析研究所講究録}, \textbf{1266}, 125-132, 2002, \url{http://hdl.handle.net/2433/42103}

		\bibitem{yama} 山口昌哉, 『カオスとフラクタル』, ちくま学芸文庫, 筑摩書房, 2010

		\bibitem{mat} 松本幸夫, 『多様体の基礎』, 東京大学出版会, 1988

		\bibitem{smale} Stephen Smale, A clssification of immersions of the two-sphere, \textit{Transactions of the American Mathematical Society}, \textbf{90}(2), 281-290, 1959, \url{https://doi.org/10.2307/1993205}

		\bibitem{hara} 原啓介, 『集合・位相・圏』, 講談社, 2020

		\bibitem{bantar} Stefan Banach, Alfred Tarski, Sur la décomposition des ensembles de points en parties respectivement congruentes, \textit{Fundamenta Mathematicae}, \textbf{6}(1), 244-277, 1924, \url{http://eudml.org/doc/214280}

		\bibitem{math} 難波博之, シェルピンスキー・マズルキーウィチのパラドックス, \textit{高校数学の美しい物語}, 2021, \url{https://manabitimes.jp/math/1295}

		\bibitem{geb} ダグラス・R・ホフスタッター, 『ゲーデル,エッシャー,バッハ : あるいは不思議の扉』, 白揚社, 1985

		\bibitem{sshimo1} 下村思游, 「文字渦」における難解性:内部観測と量子複製禁止定理, 円城塔研究, \textbf{1}(1), 20-24, 2023

		%\bibitem{sshimo2} 下村思游, 「文字渦」における漢字とその表現対象の1対1対応, 円城塔研究, \textbf{1}(1), 2023

		\bibitem{kanji} 白川静, 『漢字百話』, 中公文庫, 中央公論新社, 2002

		\bibitem{tsuka} 島内景二, 『塚本邦雄』, 笠間書院, 2011

		\bibitem{ken} 土屋健, 『エディアカラ紀・カンブリア紀の生物』, 技術評論社, 2013

	\end{thebibliography}

\end{document}