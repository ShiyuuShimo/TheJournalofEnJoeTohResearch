\documentclass[10pt, a5paper, twoside]{jsarticle}

\usepackage{okumacro}
\usepackage{enumitem}
\usepackage{amssymb}
\usepackage{amsmath}{}
\usepackage{amsfonts}
\usepackage{amsthm}
\usepackage{url}
\usepackage{here}
\usepackage[dvipdfmx]{graphicx}
\usepackage{wrapfig}
\usepackage{makeidx}
\usepackage{braket}
\usepackage{ascmac}
\usepackage{fancyhdr}
\usepackage[top=20truemm,bottom=20truemm,left=15truemm,right=15truemm]{geometry}

\pagestyle{fancy}
	\fancyhead{}
	\fancyhead[RE]{円城塔研究}
	%\fancyhead[LO]{TITLE OF ARTICLE}
	\fancyhead[LE, RO]{\thepage}
	\fancyfoot{}
	\fancyfoot[LE, RO]{\footnotesize{The Journal of EnJoeToh Research, Vol.1, No.1, 2023} }

\theoremstyle{definition}
	\newtheorem{dfn}{定義}
	\newtheorem{thm}{定理}

\setcounter{page}{45}

\begin{document}

	~ %強制改行

	\begin{center}

		\Large{\textit{memorandum}}

	\end{center}

	\newpage

		~ %強制改行

	\begin{center}

		\Large{\textit{memorandum}}

	\end{center}

	\newpage

		~ %強制改行

	\begin{center}

		{\Large 次号予告}

			\vspace{10mm}

		\begin{itemize}

			\item 【評論】円城塔序論:難解さについて

				\vspace{3mm}

			\item 【資料】コンメンタール「これはペンです」

				\vspace{3mm}

			\item 【資料】コンメンタール「良い夜を持っている」

				\vspace{3mm}

			\item 【論文】「良い夜を持っている」の統一的解釈のために:心理学・医学・物理学・数学・情報理論の観点からの多面的検討(仮)

				\vspace{3mm}

			\item 【論文】「「文字渦」における漢字とその表現対象の1対1対応」

				\vspace{3mm}

			\item 【資料】京フェス2022円城塔部屋資料 
		
		\end{itemize}

		\flushright{ほか,書評1本を掲載予定.\\内容は予告なく変更される場合があります.}

	\end{center}

	\vspace{15mm}

	\begin{screen}
		
		次号には,今号に間に合わなかった論文・評論各1本を追加で掲載予定.

		コンメンタール「これはペンです」,コンメンタール「良い夜を持っている」を一挙掲載し『これはペンです』を徹底解釈.さらに「良い夜を持っている」総説論文で同作の統一的解釈を与え,「文字渦」「ムーンシャイン」「$\varnothing$」への接続を示す.

	\end{screen}

	\newpage

	{\large 執筆者紹介}

	\vspace{3mm}

	下村思游

	 司書,SF研究者,SFレビュアー.専門は素粒子物理学,加速器科学.

	 著書に『ハヤカワ文庫JA1500総解説』(共著,早川書房,2022).

	 直近の仕事は『SFマガジン』2023年10月号特集内ブックガイド.

	 X(Twitter): @ss\_scifi

	 mastodon: @ss\_scifi@mstdn.jp

	 mail: shiyuu.shimomura\_at\_gmail.com : \_at\_ $\rightarrow$ @

	\vfill

	\hrulefill

	\center

	{\Large 円城塔研究 1巻1号}

	2023年9月10日 発行

	\flushleft{編集 数理文学研究会\\    京都, 日本\\発行 柏屋\\    京都, 日本, \url{https://sites.google.com/view/kashiwaya}\\ISSN 2759-0178(print)\\ISSN 2758-9358(online)}

	\hrulefill

	内容に関する質問・意見等は shiyuu.shimomura\_at\_gmail.com : \_at\_ $\rightarrow$ @




\end{document}