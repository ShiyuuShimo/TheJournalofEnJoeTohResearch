\documentclass[10pt, a5paper, twoside]{jsarticle}

\usepackage{okumacro}
\usepackage{enumitem}
\usepackage{amssymb}
\usepackage{amsmath}{}
\usepackage{amsfonts}
\usepackage{amsthm}
\usepackage{url}
\usepackage{here}
\usepackage[dvipdfmx]{graphicx}
\usepackage{wrapfig}
\usepackage{makeidx}
\usepackage{braket}
\usepackage{fancyhdr}
\usepackage[top=20truemm,bottom=20truemm,left=15truemm,right=15truemm]{geometry}

\pagenumbering{roman}

\pagestyle{fancy}
	\fancyhead{}
	\fancyhead[RE]{円城塔研究}
	%\fancyhead[LO]{目次}
	\fancyhead[LE, RO]{\thepage}
	\fancyfoot{}
	\fancyfoot[LE, RO]{\footnotesize{The Journal of EnJoeToh Research, Vol.1, No.1, 2023} }

\theoremstyle{definition}
	\newtheorem{dfn}{定義}
	\newtheorem{thm}{定理}

\begin{document}

	{\Large  } %強制改行

	\begin{center}

		\Large{創刊の辞}

	\end{center}

	数理文学研究会は,このたび『円城塔研究』を刊行することとした.『円城塔研究』とは,その題名の通り,円城塔に関する研究成果を発表する雑誌である.

	円城塔作品はその評価に比較すると驚くほど研究や解釈が少ない.ならば,と思って刊行を試みた次第ではあったのだが,実際に取り組んでみると,本当に論じたいことの前提をまず論じなければならず,その前提にも前提が必要であり,それにもまた,というように無限後退が発生し作業は難航した.そのため,創刊号である1巻1号にはとても文学作品に関する研究論文とは思えないような論文が複数収録されることとなった.以下で簡単ではあるが個別に紹介する.

	「コンメンタール「文字渦」」は,短篇「文字渦」の逐条解説ならぬ逐文解説を目指した資料である.もちろん完璧なものではないが,これを基に様々な読みが試みられ,より刷新されていくことを期待している.

	「“集合の集合”が集合でないことの証明」は,ラッセルのパラドックスとして知られる問題を公理的集合論において適切に解決し,ラッセルの定理として整備する論文である.本論文は本号では特に言及されることはない.というのも,本来今号に掲載予定の評論で引用する予定だったのだが,その評論のセルフ校正中に重大な数学的不足が発見され,セルフリジェクトの憂き目に遭うというトラブルが発生したためである.次号以降では頻繁に参照されることとなるため,今しばらくご辛抱いただければ幸いである.

	「系の複製に関する古典的考察」は,系を古典的操作の下で複製することについて考察を行う論文である.本考察は「文字渦」だけでなく複数の作品の解釈において極めて重要であることから,独立の論文として切り出し整理した.

	「「文字渦」の難解性」は,今号の目玉論文と言うべきものであり,短篇「文字渦」におけるSF的解決を数学的・物理学的に考察し,円城塔作品が難解とされる原因について考察する論文である.

	「原啓介『眠れぬ夜の確率論』」は,円城塔が自らの評論を志す人に向けて推薦した書籍の書評である.円城塔がそのような発言をした意図についても考察しているので,ぜひ書評を道標に実際の書籍を読み進めていただきたい.

	巻末には2021年の京都SFフェスティバルで開催した企画「円城塔作品を語る部屋」で配布した資料を掲載した.全体的にややぶつ切りの印象が強いものの,網羅性に優れているため,円城塔作品に頻出するトピックを粗く広く把握しようという用途に適している.

	また,記念すべき創刊号の表紙には,セル・オートマトンのようにも見える画像を見つけたのでそれを使用した.

	様々な困難はあったが,これまでの成果を整理し,綺麗に整頓して発表するのは大変に楽しかった.個人の楽しみに付き合わせるのは非常に恐縮ではあるが,ぜひ末長くお付き合いいただければ幸いである.

	\flushright{数理文学研究会 主宰\\下村思游}

	\clearpage

	\begin{center}

	{\Large  }

		\Large{目次}

	\end{center}

	\begin{itemize}
		\item 創刊の辞 \dotfill i

			\vspace{3mm}

		\item 目次 \dotfill iii

			\vspace{3mm}

		\item 【資料】コンメンタール「文字渦」 \dotfill 1

			\vspace{3mm}

		\item 【資料】“集合の集合”が集合でないことの証明 \dotfill 11

			\vspace{3mm}

		\item 【論文】系の複製に関する古典的考察 \dotfill 15

			\vspace{3mm}

		\item 【論文】「文字渦」の難解性 : 内部観測と量子複製禁止定理 \dotfill 20

			\vspace{3mm}

		\item 【文献紹介】原啓介『眠れぬ夜の確率論』 \dotfill 25

			\vspace{3mm}

		\item 【資料】京フェス2021円城塔部屋資料 \dotfill 28

			\vspace{3mm}

		\item 次号予告 \dotfill 47

			\vspace{3mm}

		\item 奥付 \dotfill 48

	\end{itemize}

	\vfill

	\flushleft{ 表紙の画像は国立国会図書館デジタルコレクションのモトカヅ文様研究部編『古代模様更生帖』(内田美術書肆,1929)\footnote{\url{https://dl.ndl.go.jp/pid/8311032}}を加工して作成した.}

	\newpage

	\begin{center}

		\Large{\textit{memorandum}}

	\end{center}

\end{document}