\documentclass[10pt, a5paper, twoside]{jsarticle}

\usepackage{okumacro}
\usepackage{enumitem}
\usepackage{amssymb}
\usepackage{amsmath}{}
\usepackage{amsfonts}
\usepackage{amsthm}
\usepackage{url}
\usepackage{here}
\usepackage[dvipdfmx]{graphicx}
\usepackage{wrapfig}
\usepackage{makeidx}
\usepackage{braket}
\usepackage{fancyhdr}
\usepackage[top=20truemm,bottom=20truemm,left=15truemm,right=15truemm]{geometry}

\pagestyle{fancy}
	\fancyhead{}
	\fancyhead[RE]{円城塔研究}
	\fancyhead[LO]{「文字渦」における漢字とその表現対象の1対1対応}
	\fancyhead[LE, RO]{\thepage}
	\fancyfoot{}
	\fancyfoot[LE, RO]{\footnotesize{The Journal of EnJoeToh Research, Vol.1, No.1, 2023} }

\theoremstyle{definition}
	\newtheorem{dfn}{定義}
	\newtheorem{thm}{定理}

\setcounter{page}{20}

\begin{document}

	{\Large  } %強制改行

	\begin{center}

		\Large{「文字渦」における漢字とその表現対象の1対1対応}

		\vspace{3mm}

		\large{A one-to-one correspondence \\ between Chinese characters and the objects they signify \\ in the short story \textit{Mojika}}

		\vspace{3mm}
		
		\large{下村思游}

	\end{center}

	\vspace{3mm}

	\begin{abstract}

		連作短篇集『文字渦』を貫く漢字とその表現対象との1対1対応は、冒頭作「文字渦」でその存在を暗示されたのち、明示的に言及されることがない。この1対1対応を認めることで、「文字渦」をはじめとした作品中の描写を自然に解釈可能であることを示した。

		\vspace{3mm}

		A one-to-one correspondence between Chinese characters and the objects they signify in the short stories \textit{Mojika and other stories} is appeared in short story \textit{Mojika} at first, but it is not explained obviously in latter stories. We representate that we can understand the contents of the short story \textit{Mojika} and other stories naturally with such a one-to-one correspondence.

	\end{abstract}

	\section{導入}

		連作短篇集『文字渦』において、漢字とその表現対象との1対1対応は“公理”として振る舞う。短篇「文字渦」における始皇帝の顔の変遷や、「新字」における則天文字の導入による擾乱、「幻字」における佐予、

		そして「かな」においては文字は登場人物として明確に振る舞うに至るまで、これらが

	\section{考察}

		\subsection{「文字渦」と“公理”の導入}

			短篇「文字渦」において、始皇帝嬴はその“嬴”字が揺らいでおり、“嬴”字の揺らぎは作中において嬴の顔が常にうつろうことに対応づけられる。

			また、

			ここで、“公理”として漢字とその表現する対象(嬴の場合では、“嬴”字と嬴の顔)の間に1対1対応が存在するとおく。つまり、漢字が変化すればそれにつれて漢字の表現対象が変化し、逆に漢字の表現対象が変化すれば漢字が変化する。

	\section{検討}

	\section{結論}

	\begin{thebibliography}{99}

		\bibitem{mojika} 円城塔, 『文字渦』, 新潮文庫, 新潮社, 2021

		\bibitem{com} 下村思游, コンメンタール「文字渦」, 円城塔研究, \textbf{1}(1), 1-10, 2023

		\bibitem{kanji} 白川静, 『漢字百話』, 中公文庫, 中央公論新社, 2002

		\bibitem{tsuka} 島内景二, 『塚本邦雄』, 笠間書院, 2011

		\bibitem{saussure} フェルディナン・ド・ソシュール, 『一般言語学講義』, 岩波書店, 1972

	\end{thebibliography}

\end{document}