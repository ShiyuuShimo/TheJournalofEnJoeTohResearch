\documentclass[10pt, a5paper, twoside]{jsarticle}

\usepackage{okumacro}
\usepackage{enumitem}
\usepackage{amssymb}
\usepackage{amsmath}{}
\usepackage{amsfonts}
\usepackage{amsthm}
\usepackage{mathrsfs}
\usepackage{bm}
\usepackage{url}
\usepackage{here}
\usepackage[dvipdfmx]{graphicx}
\usepackage{wrapfig}
\usepackage{makeidx}
\usepackage{braket}
\usepackage{ascmac}
\usepackage{fancyhdr}
\usepackage[top=20truemm,bottom=20truemm,left=15truemm,right=15truemm]{geometry}

\pagestyle{fancy}
	\fancyhead{}
	\fancyhead[RE]{円城塔研究}
	\fancyhead[LO]{資料:コンメンタール「Boy's Surface」}
	\fancyhead[LE, RO]{\thepage}
	\fancyfoot{}
	\fancyfoot[LE, RO]{\footnotesize{The Journal of EnJoeToh Research, Vol.3, No.1, 2025} }

\theoremstyle{definition}
	\newtheorem{axi}{公理}
	\newtheorem{dfn}{定義}
	\newtheorem{thm}{定理}
	\newtheorem{hyp}{予想}
	\newtheorem{ths}{提唱}
	\newtheorem{prn}{原理}
    \newtheorem{clm}{主張}

\begin{document}

	~ %強制改行

	\begin{center}

		\Large{コンメンタール「Boy's Surface」}

		\vspace{3mm}

		\large{Commentary of short story \textit{Boy's Surface}}

		\vspace{3mm}

		\large{下村思游}

	\end{center}

	\vspace{3mm}

	\section{解説}

        頁数・行数は『Boy's Surface』紙版\cite{boys}に準拠した.

        \subsection{題名}

            実在する数学の概念,ボーイ曲面(Boy's Surface)に由来する.

        \subsection{p.12, l.11}

            \begin{quote}

                「数理神学者たち」

            \end{quote}

            聞き慣れない用語だが,現代日本にこれを研究していると主張する研究者が実在し,入門書\cite{ochi}も出版されている\footnote{なお,当該書籍の主張は破綻しており,また研究者本人の主張もしばしば破綻している.}.

            この例は単なる言葉遊びであるが,一方で,数学と神学・哲学の融合を試みた人物は史上多くあった.最も有名なのは,数学によって神の存在証明を試みたスピノザだろう.スピノザの主著『エチカ』は,その正式名称を『幾何学的論証の秩序による倫理』という.

            他にも,スピノザに多大な影響を与えたデカルトは数理的世界観を元に人間を含む自然全てを記述しようと試み,またスピノザの同時代人で同じくデカルトから多大な影響を受けたライプニッツもまた,数学と神学・哲学の統合を試みた人物であった.

            \subsection{p.12, l.12}

                \begin{quote}

                    「カエサルのものはカエサルに.」

                \end{quote}

                新約聖書のイエス・キリストの言葉.元々は,神への信仰という宗教上の問題と,国家への服従という世俗的な問題は別であるという意味だが,ここでは,そこから派生した,すべてのものはあるべきところに戻るべきであるという意味の方だろう.後者に立って,この語を,その階層の現象はその階層の理論によって理解されるべきである(=適用外の理論をむやみに適用してはならない)と解釈すれば,自然の階層性の問題\footnote{物理学において,対象となる系の物理量(大きさ,温度,エネルギー,対象の数など)が変われば,適応可能な支配法則が変わってしまうこと.例として,速度(=運動エネルギー)について,低速度における古典力学と高速度における相対論の対応関係が挙げられる.}が自然に連想される.この解釈は,作中で物理法則の適用範囲に関する議論が扱われることを考慮すれば,十分に支持される.また,直後の「計算のものは計算に.」という言葉もよく説明する.

            \subsection{p.13, l.1}

                \begin{quote}

                    「僕は視線によって生成されて,〜」

                \end{quote}

                「パリンプセストあるいは重ね書きされた八つの物語」に登場する,ホイーラー--ファインマン吸収体理論を強く連想させる.「パリンプセスト〜」および円城塔作品全体を通じたホイーラー--ファインマン吸収体理論の解釈については\cite{pa}を参照されたい.

            \subsection{p.13, l.4}

                \begin{quote}

                    「僕は僕のみに生くるに非ず.」

                \end{quote}

                もちろん,元ネタは“人はパンのみに生くるものにあらず”.旧約聖書のモーセの言葉,そしてそれを引用した新約聖書のイエスの言葉に由来する.神学ネタがからむ作品だからか,聖書ネタが多い.

            \subsection{p.13, 6-}

                \begin{quote}

                    「数学者は僕をモルフィズムと呼ぶ.〜」

                \end{quote}

                このモルフィズムは,数学の一分野である圏論における射(morphism)を指す.

                ここで,圏および射は以下のように定義される\cite{cat}.

                \begin{dfn}

                    圏

                    圏$\mathscr{A}$とは,
                    \begin{itemize}
                        \item 対象(object)の集まり$ \textrm{ob} ( \mathscr{A} ) $
                        \item 各$A, B \in \textrm{ob} ( \mathscr{A} ) $について,$A$から$B$への射(map, morphism)または矢印(arrow)の集まり$ \mathscr{A} ( A, B ) $
                        \item 各$A, B, C \in \textrm{ob} ( \mathscr{A} ) $について,合成(composition)と呼ばれる関数
                            \begin{align*}
                                \mathscr{A} ( B, C ) \times \mathscr{A} ( A, B ) &\to \mathscr{A} ( A, C ) \\
                                (g, f) &\mapsto g \circ f
                            \end{align*}
                        \item 各$A, B \in \textrm{ob} ( \mathscr{A} ) $について,$A$上の恒等射(identity morphism)と呼ばれる$ \textrm{id}_A \in \mathscr{A} ( A, A ) $の元$1_A$
                    \end{itemize}
                    から成り,以下の公理をすべて満たすもののことである.
                    \begin{itemize}
                        \item 任意の$f \in \mathscr{A} ( A, B ), g \in \mathscr{A} (B, C), h \in \mathscr{A} (C, D)$について$(h \circ g) \circ f = h \circ (g \circ f)$が成り立つ.(結合律)
                        \item 任意の$f \in \mathscr{A} ( A, B )$について$f \circ 1_A = f = 1_B \circ f$が成り立つ.(単位律)
                    \end{itemize}
                \end{dfn}

                圏論における射は,感覚的な説明としては,対象と対象の間に立ち,その間の“関係性”を表すものであるという説明がなされる.本作のモルフィズム(=レフラー球)はまさにこの“関係性”の象徴として導入されているように感じられる.詳細は後述するが,本作におけるモルフィズム関連の描写は,モルフィズムが圏論における射に等しいことを強く示唆する.

            \subsection{p.13, l.15-16}

                \begin{quote}

                    「より正確にいえば,そう言ったディテールはあなたと僕たちの間で生成されるものであり,〜」

                \end{quote}

                円城塔作品には,物語とはテクストと読者の相互作用によって生じるものであるという主張が繰り返し見られる.それを支持するのがホイーラー--ファインマン吸収体理論である.

            \subsection{p.14, l.7-10}

                \begin{quote}

                    「僕は今こうして〜」

                \end{quote}

                本作の語り手であり,テクストそのものとみなしてよい\footnote{本作において,観測行為は観測対象と観測者,およびそれらの間に連なるレフラー球によって記述される.レフラー球とレフラー球,レフラー球と観測対象の相互作用はそれぞれ縮約可能なので,最終的には観測者とそれに接するレフラー球しか残らない.}“僕”は,自分自身が何かに書かれたテクストであり,しかも今まさに誰かに読まれているということを自覚している.この自覚は本作を通じて常に保たれる.

                このような,自分自身が何かに書かれたテクストであることを自覚していて,かつ誰かに読まれていることを自覚しているテクストによって成立するフィクションのことを,佐々木敦\cite{ssk}に従ってパラフィクション\footnote{佐々木は,同書において,メタフィクションとパラフィクションの差異について,円城塔は“これは書かれたテクストである”や“私はこのテクストを書いている”というメタフィクションを自明のものとし,“ここにこのテクストがある”というテクストの自己言及に段階を進めたと評価している.私もこれに完全に同意する.佐々木は,これに続けて,円城塔は“あなたは今このテクストを読んでいる”というテクストから読者への他者言及にまで段階を進めたと主張しているが,私はこの解釈に同意しない.なぜなら,円城塔が多用する自己言及構造,特に自己完結したクワインにおいては,その自己言及構造から他者への言及,あるいは他者から自己言及構造への言及が生じてしまうと,元々の自己言及構造は必ず変容するから.円城塔作品における自己言及構造はほとんどすべて自己完結的であり,自分自身が自分自身を指示規律することによって存在している.他者言及であるという解釈は,ほとんどの場合,この自ら指示規律する自己言及構造の破壊を意味する.もちろん,変容してもその構造の一部は自己言及構造を保つかもしれず,また十分要素を集めてくることで元の自己完結なクワインに戻ることも出来るかもしれないが,その“大きさ”は極めて増大していることだろう.円城塔作品において一見他者言及であるかのように思われる箇所は,実は他者言及であるかのように読めるだけの箇所であって,テクストが誰に語るでもなく語った独り言を違法に読んでいるに過ぎないと解釈するべきだろう.宇宙空間から誰に宛てるでもなく発信されたテクストである「バナナ剥きには最適の日々」はこの解釈を支持するものである.この議論は,改めて検討することにしたい.}と呼ぶことにする.

            \subsection{p.14, l.11}

                \begin{quote}

                    「レフラー球.」

                \end{quote}

                本作のキーガジェット.レフラー球はあくまで架空の数学的対象ではあるのだが,それはモルフィズムであり,この作品の語り手自身であり,テクストと読者の間に横たわる“青く澄み透る高次元球体”である.自身から自身への変換を恒等変換といい,これは先掲の通り圏の定義に含まれる.これが可能であることからも,モルフィズムが射であるという説が支持される.

                また,レフラー球が青い理由として2つの要素が考えられる.

                まず,円城塔本人の勘違いである.円城塔は,2007年10月に開催された京都SFフェスティバルでの企画において,『Self-Reference ENGINE』と『Boy's Surface』の自作解題\cite{sri}を行っている.これによれば,円城塔は,宮沢賢治『春と修羅』の序文\footnote{「わたくしといふ現象は/仮定された有機交流電燈の/ひとつの青い照明です」\cite{myz}}にある「青い照明」を「青い証明」と勘違いしていたという.

                次に,ドルトンによる色覚異常の原因に関する仮説が挙げられる.これについては,本作後半でドルトンの色覚異常に関するエピソードが登場するため,そちらで詳述する.

                さらに,レフラー球の発見者アルフレッド・レフラーの由来もまた,『Boy's Surface』文庫版で追加された自作解題「What is the Name of This Rose?」で明かされている.曰く,数学者・哲学者アルフレッド・ノース・ホワイトヘッド,数学者アルフレッド・タルスキ,生理学者・物理学者オットー・レスラー,数学者ミッターク・レフラーを混ぜ合わせたものであるらしい.

                ここで,偶然ここに帰されることになってしまった面白い挿話を紹介したい.実在の神学者,深井智朗の研究不正に関する話である.深井は,“神学者カール・レーフラー”なる人物の著作を紹介するとともに,それらの著作を論拠として研究活動を行っていた.しかし,他の研究者から“神学者カール・レーフラー”は実在しなかったとの指摘があり,深井の大規模な研究不正が発覚するという事件が本作の発表後(2018年)に発生した\footnote{実際のところ,2007年頃から深井にはその研究者としての態度に関する批判があり\cite{yskt},当時の学術誌の書評などで指摘はされているため,円城塔本人が気づく余地はあるのだが,この偶然の一致の因果関係や意図の有無については不明である.}\cite{ntz}.実在しない理論をこねくり回す作品に登場する実在しない数学者の名前が,実際の研究不正に登場する実在しない神学者の名前に酷似していることは,偶然ではあるが非常に興味深い.

            \subsection{p.18, l.8-9}

                \begin{quote}

                    「偉大な数学者として名を上げるにはここが正念場という年齢〜」

                \end{quote}

                24歳という年齢は,現代では修士〜博士にあたる年齢だが,理論物理学・数学などの理論屋にとっては,学者としての最盛期を迎える年齢であるとされている.代表的なところでは,アーベルが病死したのが27歳,リーマンが多様体を導入したのが28歳,ハイゼンベルクが行列力学を完成させたのが24歳,ディラックが量子力学における交換関係と古典力学におけるポアソン括弧の類似性を見出したのが23歳,波動力学と行列力学の等価性を証明したのが24歳,アインシュタインの奇跡の年(特殊相対論,ブラウン運動,光量子仮説)が26歳である.

                ここに挙げたものは人類史上最高峰の上澄みといえる人物たちだが,現代においても,理論屋として大学院に進学する者は,教師と学生ではなく,同じ物理学者として共同研究を行うという形で研究生活を送ることになる.

            \subsection{p.18, 11-}

                \begin{quote}

                    「国際会議の参加のために立ち寄ったパリでの出来事であるというのだが,〜」

                \end{quote}

                これは嘘のようで本当の話.物理学者や数学者の自伝などを読んでいると,裏付けの出来ない記述がやたら出てくる.特に,20世紀前半のヨーロッパを拠点としていた学者だと,そもそも国境を越えるのが割と簡単で多国籍なメンツになることが多く,また戦後処理や亡命でころころ国籍が変わるものがいたりと,判別が難しい情報が多い.例えば,ドイツのロケット工学者として有名なヘルマン・オーベルトは,二重帝国領ルーマニア出身で,学者としてはドイツ,生業の教師としてはルーマニアを拠点としていた.

            \subsection{p.19, l.9-10}

                \begin{quote}

                    「レフラー恋に落つの報に接した知人たちの応答が一様に,〜」

                \end{quote}

                物理学科あるある.(大変失礼な話ではあるのだが)他人に興味がなさそうな教官が妻子持ちだったときの反応がこんな感じ.

            \subsection{p.19, l.12}

                \begin{quote}

                    「バナッハ空間」

                \end{quote}

                実在の数学上の概念.ここで,バナッハ空間は以下の通り定義される\cite{mtz}.

                \begin{dfn}

                    バナッハ空間

                    バナッハ空間とは,完備なノルム空間である.

                \end{dfn}

                ノルム空間は以下の通り定義される\cite{mtz}.

                \begin{dfn}

                    ノルム空間

                    $\mathbb{C}$上ベクトル空間Sがノルム空間(normed space)であるとは,以下のすべての条件を満たすノルム(norm)とよばれる関数$||\cdot||: S \to \mathbb{R}$が存在することである.

                    \begin{itemize}
                        \item 任意の$x \in S$について$||x|| \geq 0$
                        \item $||x|| = 0 \Leftrightarrow x = 0$
                        \item 任意のスカラー$k \in \mathbb{C}$と任意の$x \in S$について$||k x|| = |k| \cdot||x||$
                        \item 任意の$x, y \in S$について$||x + y|| \leq ||x|| + ||y||$(三角不等式)
                    \end{itemize}

                \end{dfn}

                また,完備は以下の通り定義される\cite{mtz}.

            \newpage

                \begin{dfn}

                    完備

                    ノルム空間が完備であるとは,点列$\{x_n\}$が収束するならば,その収束先がそのノルム空間に含まれることである\footnote{未定義語の多用を避けるためやや砕けた言い方となっている.より正確には,任意のコーシー列がそのノルム空間内の収束列になっていることが完備の定義である.詳細は\cite{mtz}を参照されたい.}.

                \end{dfn}

                側溝と並んで“落ちることのできるところ”としてバナッハ空間が挙げられているのは,定義より,バナッハ空間は任意のコーシー列がその内部で収束する,つまり“落ち込んでいける”からだと考えられる.要するに,数理科学系の人間向けのギャグである.

                なお,量子力学で多用されるヒルベルト空間は,ノルムとして内積が入っているバナッハ空間である.

            \subsection{p.20, l.8-11}

                \begin{quote}

                    「恋人と数式,どちらを優先するかという問いが〜」

                \end{quote}

                数理科学系の人間の言い回しそのもの.

            \subsection{p.20, l.16-17}

                \begin{quote}

                    「この光景の裡に〜」

                \end{quote}

                身近な光景から数理科学の理論を思いついたエピソードは多数ある.ニュートンのリンゴと万有引力,湯川秀樹の木漏れ日と中間子,アインシュタインの手鏡と特殊相対論がその代表例である.

            \subsection{p.21, l.5-7}

                \begin{quote}

                    「史上最初に作成された顕微鏡のレンズが〜」

                \end{quote}

                最初の顕微鏡を作った人物には諸説あるものの,レーウェンフックが微生物を見られる精度の顕微鏡を作った最初期の人物であることは確かだろう.

            \subsection{p.23, l.2-}

                \begin{quote}

                    「アルフレッド・レフラー(米,1983-2043)は生来盲目の数学者として〜」

                \end{quote}

                実在した生来盲目の数学者として,ベルナール・モラン\footnote{モランは本作後半で登場するため,詳細は該当箇所で解説する.}が挙げられる.また,後天的に盲目となった数学者としては,オイラーとポントリャーギンが挙げられる.

            \subsection{p.24, l.3-}

                \begin{quote}

                    「レフラー球を紙面に〜」

                \end{quote}

                レフラー球についての復習.ここの描写からも,レフラー球が射であることが支持される.

                %後で支持される理由を補う

            \subsection{p.25, l.1}

                \begin{quote}

                    「ネッカー・キューブ」

                \end{quote}

                ネッカーの立方体とも.立方体の格子を表す線を区別せず全て書いたもの.

            \subsection{p.25, l.2}

                \begin{quote}

                    「ジャストロー図形」

                \end{quote}

                ジャストロー錯視とも.兎にも鴨にも見える錯視図形のこと.同じ名前がつけられた,同じ大きさの扇形図形を並べると大きさに差異があるように見えるものもある.

            \subsection{p.25, l.3}

                \begin{quote}

                    「映像が切り替わる時間間隔は$\Gamma$分布に従い,〜」

                \end{quote}

                $\Gamma$分布は以下のように定義される\cite{tki}.

                \begin{dfn}

                    $\Gamma$分布
                    \begin{align*}
                        f(x) = \begin{cases} \displaystyle \frac{\lambda^\alpha}{\Gamma (\alpha)} x^{\alpha - 1} e^{- \lambda x} \ (0 \leq x) \\ 0 \ (x < 0) \end{cases}
                    \end{align*}

                \end{dfn}

                一定の確率$\lambda$である現象が起こるとき,その現象が起こるまでの時間間隔は$\Gamma$分布に従うことが知られている.錯視の切替が確率的に発生し,またその確率$\lambda$が脳神経の特性によるというのは極めて妥当な推測であり,この記述もまた妥当なものである.

            \subsection{p.25, l.5-7}

                \begin{quote}

                    「もとの図形はただの描かれた線に過ぎず,固定されて動かない.〜」

                \end{quote}

                この記述は正しい.いま,対象$ O_{object}$,観測者$ O_{observer}$,観測者による観測を表す演算子$ L$を導入し,観測行為という系を記述する.
                \begin{align*}
                    S(t) = O_{object} L O_{observer}
                \end{align*}

                ここで,対象である図形は時間発展せず,演算子$L$も時間発展しないので,時間発展項を入れられるのは観測者しか残らない.したがって,時間発展するのは観測者(の脳内の電気信号)である.

            \subsection{p.25, l.9-}

                \begin{quote}

                    「見ているはずのものが実は見えていない〜」

                \end{quote}

                ゲシュタルト心理学などに通じる記述.特に文字に対して生じるゲシュタルト崩壊と呼ばれる現象は,のちの「文字渦」の発想に直接的に結びつく.

            %\subsection{p.25, l.16-17}

                %\begin{quote}

                    %「視覚情報処理系とは,とりあえず進化の荒波を進むことができる程度のものであればよく,〜」

                %\end{quote}

            %後で補う

            \subsection{p.25, l.17-p.26, l.1}

                \begin{quote}

                    「ここにつけこむ余地があり,実際につけこまれており,つけこむことが可能である.」

                \end{quote}

                極めて円城塔らしい一文.最後につけこむことが可能であると改めて言及するのはやや意味不明に見えるが,これは「ここに(原理上)つけこむ余地があり,(自然現象として)実際につけこまれており,(これを利用して意図的に)つけこむこともまた可能である」という意味である.この背後には,物理学的に可能なことは必ず実現されるものであり,実際にそれは自然界で観測されうるし,われわれはそれを利用することが出来るという論理がある.

                これを,「ここに(物理学の机上の空論としては)つけこむ余地があり,(実際の自然現象として)実際につけこまれており,(工学的に)つけこむこともまた可能である」としてもよい.この例としては,核分裂反応が挙げられる.核分裂は物理学的に予言されており,実際にそれはガボンの天然原子炉\footnote{ガボンのオクロで発見された,天然ウラン鉱脈が自然に連続的な核分裂反応を起こしていた痕跡のこと.一定の濃度のウラン鉱脈に地下水が染み込んだことで,軽水炉のような構造が実現され,核分裂反応に至ったと考えられている.}という形で自然界で観測されうるし,われわれはそれを核分裂炉として利用することができる.

                確率的円城塔模倣機関\footnote{下村が円城塔のテクストを学習データとして自作した円城塔LLMのこと.詳細は\cite{llm}を参照されたい.}も似たような文字列を生成するが,それはこの言い回しの言葉遊び的な部分のみを確率的に模倣しただけであり,背後に隠れている数理と論理\footnote{理学的に予言されており,実際に自然現象として確認されるならば,技芸に落とし込める,というのがこの場合での数理と論理.}が欠落している.ここから,言葉遊びの連鎖とその裏にある数理と論理はを示すこの一文は,円城塔の作風をよく表す,重要な用例であると言える.

            \subsection{p.26, l.6-}

                \begin{quote}

                    「レフラー球は,その基盤を顕とすることのない錯覚を引き起こすと知られた,〜」

                \end{quote}

                作中において,われわれは,レフラー球を通じて歪められた観測のみ可能である.これはオットー・レスラーの内在物理学から多大な影響を受けている.

                %内在物理学の概要説明.われわれが観測する“対象”とは,その対象の形而上学的な真の姿そのものではなく,形而下の観測可能な姿に拘束される.

                内在物理学とは,観測者自身が観測したい系や理論の内部にいる場合における物理学および観測理論である.従来の物理学(=外在物理学)が,この世界全体の存在を素朴に仮定するのに対し,内在物理学では,世界全体は観測可能ではなく,観測者が得られるのは,自分自身と,それ以外との間にあるインターフェイスに過ぎないとする\cite{ott}.

                内在物理学は,観測可能なものは世界全体のうち表層だけであるとし,表層の奥に蠢く自然はやはり自然法則に支配されるものの,それを観察することが出来ない以上感知しないとする世界観をおく.これは日常的な感覚\footnote{このような,科学的には間違っているものの,日常生活を過ごす上で大きな支障を来さないよう素朴に獲得された知的体系のことを,素朴理論という.学校教育等において,素朴理論の克服は重要な課題であり,博士号取得者でさえも素朴理論の支配から解放されていないことを示唆する研究\cite{cog}がある.}では非常に特異な世界観であるように思われるが,その日常的な感覚が実は物理学的な事実に反する妄想であり,むしろ内在物理学は相対論・量子論といった現代物理学の基礎が提供する世界観と極めてよく合致する.

                詳細は後述するが,レフラーは数理科学で重視される姿勢を殊更強調し,通常はそのような姿勢を用いなくてもいいような日常生活においてもその姿勢を徹底するなど,いわゆる理系的な特徴が強調されている\footnote{作中で繰り返し描写されるレフラーの杓子定規すぎる言動を,レフラーが自閉スペクトラム症であることの証左とする指摘も可能ではあるが,これについては別途慎重に検討したい.}.本作において内在物理学的な描写が数多く見られるのは,レフラー自身が内在物理学の世界観を貫き通しているからだろう.

                また,レフラー球について気づくのが難しいのは,“真空”や“空気”といったものの実在や科学的議論が遅れた原因に近いと思われる.真空は日常生活にあまり馴染みがなく,また空気は無色透明で目に見えないため,日常生活で深刻な問題をもたらすことが少なかったため,科学的手続きによって検証される器械がなかった.水銀の供給が豊富になりトリチェリの真空が発見されてはじめて,真空に関する科学的な議論が勃発することになった.トリチェリの真空をこぞって議論していた時代がある,というのは現代からすると驚くべきことである.逆に,エーテルが存在しないということが厳密に証明されたのは,改めて考えてみると極めて驚くべきことである.

            \subsection{p.27, 5-8}

                \begin{quote}

                    「あなたは現在,紙面に広がる蜘蛛の巣めいた基盤図形を覗き込んでいて,〜」

                \end{quote}

                「考速」に極めて似た表現\footnote{「あるいは、不意に目の前に投げ出された脳みそを、絡み合う矢印の山と見なして解きほぐすこと。だから順序は逆転される。定義と公理が与えられ、推論規則を用いて定理が導き出されるわけではなく、相互に作用する網目がまずあり、推論規則が見定められ、定理の形が決定されて、定義と公理が抽出される。」\cite{goto}}が登場する.また,この基盤図形が“十分複雑”であると仮定すれば,「ムーンシャイン」も連想される.

            \subsection{p.29, 3-4}

                \begin{quote}

                    「レフラー球とその基盤図形は,存在を証明されてこそいるものの,実現の困難な高次元構造物として知られていた.」

                \end{quote}

                ある対象が存在することを主張する定理のことを存在定理という.このとき,その存在定理はその対象を構成する具体的な方法を与えるとは限らない.このような存在定理のことを構成的でない存在定理といい,作中の描写から,レフラー球の存在定理は構成的でない存在定理の典型的な例であると推察される.

                数学者としては,構成的でない存在定理であってもそこまで不満ではない.なぜなら,実際に構成する方法を議論せずとも数学的な議論を行えるように前もって整備しているから.選択公理から,存在する場合をとってこれるようにすることで,実際の構成方法を回避する.ただし,形式証明の立場(数学を検証する超数学者)からすると,結構嫌いらしい.なぜなら,構成的でない存在定理は選択公理を暗黙理に使用している場合が多いため.

                一方で,物理学者としては,実際にどのように構成すればいいのか,どのような状況であれば実現可能なのかがわからない対象が存在すると言われてもどうしようもなく,構成的な存在定理は意味不明なように感じられる.

            \subsection{p.31, l.1-3}

                \begin{quote}

                    「そうしてみるような人におかれては,〜」

                \end{quote}

                卑近なネタ.初期円城塔には見られるが,のちのちは消失していく.

            \subsection{p.32, l.11-}

                \begin{quote}

                    「フランシーヌ・フランス,この年二十九歳.専門は認知科学.〜」

                \end{quote}

                学者の専門領域とその学者の人間性は,強く結びつくこともあれば,そうでないこともある.

                また,フランシーヌ・フランスという命名からは,「道化師の蝶」に登場する友幸友幸のような対角化された命名が感じられるとともに,富野由悠季っぽさも感じられる.

            \subsection{p.32, l.15-}

                \begin{quote}

                    「盲視として知られる現象は,〜」

                \end{quote}

                盲視は実在する現象.“見えていると意識出来ていないのに見えている”という一見矛盾した現象を指す.

            \subsection{p.33, l.15}

                \begin{quote}

                    「純粋に推論から得られた知識を,まるで視覚から得た知識であるかのように行動する」

                \end{quote}

                これは後に語られる,レフラーの過剰な帰納的・公理的態度と,本来交わるはずのない双方の態度がいとも簡単に混交してしまうという,レフラーの言動の異形さを端的に表している.

            \subsection{p.34, l.4-7}

                \begin{quote}

                    「結局のところ,フランシーヌが注目していたのがその種の二段認識過程であったのに対して,二人の出会いからレフラーが着想した理論はそれが更に暴走したところの多段認識過程〜」

                \end{quote}

                認知機構の多重暴走というモチーフは「ムーンシャイン」とも共通し,ウルフラムの提唱\footnote{十分複雑なものはチューリング完全であり,それは生命である,というもの.円城塔のかつての専門である,複雑系で非常に流行していた,半ば信仰化した科学的直感.詳細は\cite{moon}を参照されたい.}を使い始める気配を漂わせる.また,カントのいう知性の暴走も連想される.

                いわゆる外挿としてのSFの面白おかしさは,知性が暴走する様を楽しむところに近いかもしれない.

            \subsection{p.34, l.12-}

                \begin{quote}

                    「科学的立場としては首肯せざるを得ない見解であり,繰り返して同じ実験結果の得られるものが自然科学の対象である.」

                \end{quote}

                可能なこと(=物理法則が許す現象)はいつか必ず実現され,また一度実現されたことは必ず複数回実現され得る,というのは物理学の大前提\footnote{前者は理論的予言の,後者は再現実験の根拠である.}である.ここでは,これを認めないのであれば,自然科学という体系は成り立たない,ということを言っている.

                自然科学においては,同じ条件で実験を繰り返したとき,同じ結果が得られることを期待する.同じ結果が得られないのであれば,暗黙的な条件を揃えることが出来なかったために同じ条件を設定出来なかったと考えるのが通常である.特に物理学はこれを極めて厳密に要請する\footnote{なお,量子力学は真に同じ条件を設定しても,実験結果がばらつく.これは量子力学においては結果は統計的にしか得られないという量子力学の本質に由来している.しかしながら,そのようなばらついた実験結果を集めていくと,やがて期待される実験結果が統計的に得られる.}.

                このような自然科学における推論は帰納的であり,また自然を理解するための手法は帰納的であるべきだろう.なぜなら,自然こそが真理であり,真理を説明出来ないものは真に真理を記述する体系ではないから.物理学はまさにこの姿勢を明確にしており,いかに物理学的に美しい体系であっても,その体系によって自然現象を説明出来ないのであれば,直ちにその誤った体系を(万物理論としては)棄却する\footnote{実際,相対論以前の解析力学やエーテル理論は真に完成されており,真に美しい完璧な体系であった.しかし,それらは相対論的スケールの物理現象を説明することが出来ず,解析力学は古典力学的スケールにおける相対論の近似理論と解釈されるようになり,またエーテル理論は完全に破棄された.}.

                しかし,帰納的な議論は,自然科学という営みを構成する半分の要素でしかない.帰納的な議論は,自然現象の漸進的な構築とその検証には寄与するが,大きな前進には結びつきにくい.自然法則の体系化やその工学的利用のためには,帰納的に得られた自然法則を公理とし,公理的(演繹的)に議論する必要がある.

                これら帰納的議論と公理的議論は,両輪となってうまくバランスがとられなければならない.過剰に帰納的な議論はいつまで経っても科学理論の完成には至らないだろうし,かといって拙速な帰納的帰結はしばしば誤謬をもたらすだろう.通常,物理学者は,極めて慎重に実験を繰り返すことで,より精度の高い帰納的議論を可能にし,物理理論を丁寧に作り上げていく.

                %一方で,レフラーは過剰に帰納的であり.かつ過剰に公理的である.ここでは,過剰に公理的な態度が,あらゆる命題を逆にして考える癖として描写されている.数学や物理では,“定義よりこれは直ちに従う”というフレーズで強制的に正当化する論理としてこれが観測される.例えば,常圧において,水が沸騰する温度は100${}^\circ$Cであるという定義について,水が常圧において沸騰するのは何度かという問いに対して,100${}^\circ$C,定義よりこれは直ちに従う,というような具合である.ここからしばらく,レフラーの異常なものの見方,より正確には過剰に物理学な姿勢の描写が連続する.

            \subsection{p.35, l.5-12}

                \begin{quote}

                    「レフラーにおいて異なるのは,〜」

                \end{quote}

                ここにある再帰定理の説明は,厳密さを欠くものではあるが,直感的な説明として十分正しい.再帰定理は,(ポアンカレの)回帰定理ともいい,ポアンカレによる天体力学上の三体問題\footnote{三体問題は一部の特殊な条件下を除いて,解析的に解決することが出来ないことが知られている.}の研究の途上で証明された力学系の定理である.この再帰定理は,人類が初めて見たカオスであり,円城塔が専門としていた複雑系という分野の端緒となる出来事であった.ここで,再帰定理の厳密な表現は以下の通り\cite{pla}.

                \begin{thm}
                    ポアンカレの再帰定理

                    $\mathfrak{M}$の任意の可測部分集合$\mathfrak{A}$は,それと同じ濃度をもつ集合$\mathfrak{D}$を含み,その各点$\mathfrak{p} \in \mathfrak{D}$は$\mathfrak{A}$内に無限個の像$\mathfrak{p}_l (l = l_1, l_2, \cdots \ | \ l \to \infty)$をもつ.

                    ただし,$\mathfrak{M}$は,すべての実時間$t$に対して領域$\mathfrak{R}$に留まるような軌道であって,しかも初期値$\xi$の軌道$x(t, \xi)$の作る任意の集合.
                \end{thm}

                レフラーは,この再帰定理を根拠に,一度起こった事象はすなわち無限回起こるというように観測を短絡する.つまり,レフラーにとって,一度でも起こった事象は直ちに公理となる.

                また,直後の“だいたいのことはだいたい同じように起こり続ける”という描写からは,レフラーが統計力学的な認識をしていることが示される.統計力学は,巨視的な集団がもつ特徴的な物理量を少数集めてくることでその集団の振る舞いをおおまかに記述可能であるという理論である.素粒子論があらゆる物理量を厳密に集めてくることを要請する体系であるのに対し,統計力学や熱力学はある程度の観測である程度の精度の予言を可能にする.

                再帰定理はエルゴード理論\footnote{これを直接の発想の基とするのが「遍歴」.}を生み出し,また捩じくれ曲がった末にロシア宇宙主義という思想的潮流を形成することになる.

            \subsection{p.36, l.3}

                \begin{quote}

                    「数学的構造を牽強に付会して妄想を進めるレフラー」

                \end{quote}

                先述の統計力学的認識,ひいては過剰な帰納的・公理的議論は異常であることが明言されている.

            \subsection{p.36, l.10-11}

                \begin{quote}

                    「皿を割った瞬間にかかってきたレフラーからの電話」

                \end{quote}

                フランシーヌが皿を割る可能性がある以上,フランシーヌはいつか必ず皿を割るし,一度でもフランシーヌが皿を割るのであれば,フランシーヌは無限回皿を割る.フランシーヌがいつ皿を割るかという無限の条件分岐によってレフラーは無数に分裂し,その中には,フランシーヌが皿を割った瞬間に偶然レフラーが電話をかけ,偶然フランシーヌが皿を割るのも仕方がないという話題になる分岐も存在する.

                %ここは批判されるべき

            \subsection{p.37, l.11-12}

                \begin{quote}

                    「女性とは〜」

                \end{quote}

                レフラーとフランシーヌの共同生活で同様の事象が多発したのだろう.レフラーからすれば,レフラーを自身の家に入れたフランシーヌが何度も便器に嵌まり込んだという観測事実から,これを帰納して女性とは男性を家に入れては便器に嵌まり込むものであるという一般則が得られ,これの逆をとって,女性とは便器に嵌まり込むために男性を家に入れるものであるという論理が誕生する.

                ここからさらに目的論的になるのは進化論も辿った道である.世間には未だ根強く残っているものの,生物学では目的論的進化論は否定されている.

            \subsection{p.38, l.4-}

                \begin{quote}

                    「その解答の一つを僕は実践させられているわけで,〜」

                \end{quote}

                情報の圧縮,あるいは縮約とそのための縮約記法について書かれている.ある長さの文字列が記述可能な最大の情報量は,コルモゴロフ複雑性に一致する.ここで,コルモゴロフ複雑性は以下の通り定義される\cite{sip}.

                \begin{dfn}

                    コルモゴロフ複雑性

                    $x$を二進文字列とする.$x$の最小記述を$d(x)$と書く.これは,テープ上に$x$を出力して停止するチューリング機械$M$と入力$\omega$の対のうちで,その記述$\langle M, \omega \rangle$の長さが最短であるものを指す.このような記述が複数存在するならば,その中で辞書式順序で最初のものを選ぶ.$x$の記述の複雑さを$K(x)$と書き,$K(x) = | d(x) |$とする.記述の複雑さは,コルモゴロフ複雑性ともいう.

                \end{dfn}

                縮約記法は「良い夜を持っている」で主題として扱われている.このような縮約記法は,アインシュタインの縮約記法やファインマン・ダイアグラムなど物理学で多用される\footnote{余談だが,数理科学と人文科学の最大の差異は,学問の内容を縮約可能か不可能かというところにあると思う.私はニュートンの原著論文を読んだこともないし,コーシーやシュワルツの原著論文を読んだこともないし,無論ブルバキの原著を読んだこともないのだが,当然ニュートン力学をマスターしているし,微積分も出来るし,構造主義的な数学を十分に理解出来る.物理学や数学については,臆することなく論じられる.カントやハイデガー,ポパーの原著を読んでおらず,したがって彼らの哲学を論じることには自信はないが,一方でラッセル,ゲーデルから始まる分析哲学に関しては,原著を読んだことはないが,ある程度自信を持って論じることが出来る.円城塔も,同様のことを呟いている.「「エンジニアの方にとっては、読書とはさしあたりfoldlだと考えて頂いて差し支えありません」みたいな読書論はどうか。」\url{https://x.com/EnJoeToh/status/1939941581450035575}}.

            \subsection{p.38, l.6-7}

                \begin{quote}

                    「一つ埋め込むことができれば、〜」

                \end{quote}

                レフラーの過剰な態度がまたもや登場する.

            \subsection{p.39, l.1-6}

                \begin{quote}

                    「この奇妙な過程が,〜」

                \end{quote}

                今更だが,変換が無数に連続して繋がっていく様からは,カリー化が連想される

                %後で整理されるべき

            \subsection{p.39, 16-17}

                \begin{quote}

                    「こうしてみて,レフラー自身の手になるレフラー論文全集が,いかに超絶技巧を尽くしたものであるか」

                \end{quote}

                “超絶技巧”という言葉は音楽や文学に対する賞賛として使われるが,数学や物理学においても,とんでもなく複雑な計算を天才的な発想で簡単化したことを指してこういうこともある\footnote{私が触れた実例を参考として挙げておきたい.大学院の理論屋向けの場の理論の講義で,経路積分の行間を埋めることが宿題として出された.それは大変に複雑な計算であり,(当時コロナ禍の真っ最中だったので)手元の教科書とネットでアクセス出来る講義ノートをひたすら調べても解決することが出来ず,自身の無能さを前に大学院に進学したことを後悔しながら次の週の講義に出席した.講義が始まり,教官から宿題を板書するよう声がかかったが,誰も名乗り出ない.それもそのはず,宿題となっていたのは,ファインマンがノーベル物理学賞を受賞した最大の貢献の部分であった.教官曰く「こんな超絶技巧が独りで出来るなら,君らはもうノーベル賞を獲ってるからここにいる必要はない」とのこと.}.フランツ・リストがそう言われるように,ラマヌジャンの意味不明な天啓を指して言うことが多い.

            \subsection{p.40, l.7-8}

                \begin{quote}

                    「低次元において面倒でも,むしろ無限次元においての方が定式化の易いものは実は多い.」

                \end{quote}

                前提として,そもそも無限というものは人間にとって非常に難しい.無限にも“大きさ”\footnote{ここでは濃度のこと.}に違いがあると数学者が初めて気づいたのは,実に19世紀の末になってからである.それまでは,哲学者はもちろん\footnote{学部で物理学を学び,のちに哲学に転向した哲学者,大森荘蔵は,数理的な概念を誤解に基づいて濫用した悪い哲学者の実例として有名である.大森によれば,アキレスと亀の逆説は「異常に長い間解決を拒み続けてきた」\cite{omr}というが,これは大森が根本的に現代数学・現代物理学を理解出来ていないから生じてしまったミスである.この逆説は,学部教養程度の物理学や数学を学べば,容易に解決可能なパズルである(と,大森荘蔵の弟子で自身も物理学から哲学に転向した哲学者,野家啓一が講義で言っていた).実際,これは$\varepsilon$--$\delta$論法を用いれば,容易に解決可能.}のこと,数学者も無限という概念をほとんど整理しないまま,混乱した状態で議論を行なっていたし,しかもそれが問題にならない状況にあった.

                一般に,有限は無限よりも扱い易いことが多いが,世の中には性質の極めて悪い有限というものがある.「ムーンシャイン」に登場する散在型有限単純群のひとつ,モンスター群(モンストラスムーンシャイン)は異常に扱いにくいことが知られている.

                物理学で有名なところでは,(無限ではないが)2次元では原子が成立しない一方.1次元と3次元では原子が成立可能なこと\footnote{$\because$ \ 2次元空間では電磁ポテンシャルが全領域で非負となって束縛系を構成出来ないから.}が挙げられる.

            \subsection{p.40, l.9-13}

                \begin{quote}

                    「僕は,無限次元レフラー空間内で,〜」

                \end{quote}

                わからないというのは本当にそう.確かに一義的な良い定義を与えられるにもかかわらず,その実際の振舞いを記述することが難しい,という事態は,まさにカオスを象徴する振舞いである.
                %カオスの簡単な例としては,二重振子が挙げられる.
                %二重振子の実例を書く

            \subsection{p.41, l.7-8}

                \begin{quote}

                    「これがただの連鎖生成される枝分かれではないことが,レフラー空間探索の困難さを生み出している.」

                \end{quote}

                正しい.一方向のみに分岐するのであればまだ解析は楽だったが,ループがあるという条件が事態を極めて困難なものにしている.

            \subsection{p.41, l.16}

                \begin{quote}

                    「循環レフラー群」

                \end{quote}

                レフラー球から生成された連鎖が元のレフラー球に戻り,しかも途中分岐がなく孤立して宙に浮かぶレフラー球の集まりのこと.これは何かから何かの間に矢印が飛んでおり,それが連鎖の果てに元に戻ってくる,つまり自分自身が自分自身を指示規律し孤立して宙に浮かぶ系,という円城塔作品に頻出するモチーフ\footnote{『Self-Reference ENGINE』のSelf-Reference ENGINE,「$\varnothing$」の作品宇宙,「ムーンシャイン」のモンスター群,「考速」のスピノザ『エチカ』の”公理二”など.}の現れ.

                また,直後にある,循環レフラー群をなぜ観測出来るのかという議論は,Self-Reference ENGINEが言及不能であることの議論と等価なはずだが,ここでは“視野の端から見る”ならばセーフという例外規定を設けている.これは相当意味不明で,これこそが本作におけるSF的に最大の嘘かもしれない.

            \subsection{p.42, l.14-15}

                \begin{quote}

                    「無限回のレフラー球覗き込みにおいて,距離無限小まで無限回接近することの知られた,無数のレフラー球系列からなる構造物」

                \end{quote}

                $\varepsilon$--$\delta$論法を思い出す.$\varepsilon$--$\delta$論法とは,大学初年度の解析学で突如出現し,高校数学に慣れきっていた大学新入生の関門として立ちはだかる,数学における常套手法である.歴史的には微分の厳密な定義を与えるときに生み出されたもので,導入したのはコーシーであると考えられている.

                $\varepsilon$--$\delta$論法は慣れてさえしまえばどうということはないのだが,あまりにも犠牲者を出しすぎているので,これに特化した参考書が複数出版されている\footnote{例えば\cite{mtn}や\cite{hsi}など.いずれも出版社は理工系の教科書で定評のある出版社であり,粗悪な本ではない.実際,私も学部生の頃に図書館で借りて読んだことがある.}ほどである.

            \subsection{p.42, l.17}

                \begin{quote}

                    「レフラー予想」

                \end{quote}

                直後にあるレフラー予想の主張を読む限り,要するに,レフラー予想はレフラー球の連鎖(トルネド)は“盲腸”を持たないことを主張している.とはいえ,その否定的証明も肯定的証明のいずれも非常に難しいことが察知される.

                この困難な証明を自動的に証明するためにトルネド内に投下されたエージェントが,本作の語り手であるレフラー球である.

                作中にもある通り,レフラー予想はベンジャミン・ロンドンによって否定的に解決される.すなわち,全てのレフラー球は必ず循環レフラー群の一部であり,“僕”による証明は停止しない.この証明が停止しないということから,停止性問題が直ちに連想される.この停止性問題はコルモゴロフ複雑性と深遠な関係を持ち,さらにチューリング完全とも繋がる.コルモゴロフ複雑性は自己言及構文を前提としていることにも注意したい.

                また,すべてのレフラー球が循環レフラー群の一部である,という作中事実は,本作が内在物理学に従っていることの確固たる証拠である.なぜなら,本作が外在物理学に従うのであれば,ハイゼンベルク切断によって観測者を確立することが出来,循環レフラー群に含まれないレフラー球が可能となるから.

            \subsection{p.43, l.14-}

                \begin{quote}

                    「昔々神様がいて,〜」

                \end{quote}

                このパズルは,数学者・論理学者レイモンド・S・スマリヤンが定式化した数理パズルの言い換え.直後にある,このパズルを解くためのアルゴリズム\footnote{例えば“あなたは正直者ですか,という質問に「はい」と答えますか”}が存在するのも事実であり,スマリヤンはそのアルゴリズムをネルソン・グッドマンの原理\footnote{哲学者ネルソン・グッドマンに由来する.}と呼んでいる\cite{sum}.

                また,分岐ごとに分裂云々というのは,多世界解釈に近いものを感じる.ここで注意したいのは,円城塔は多世界解釈を作品内で多用するが,円城塔は多世界解釈を物理理論として支持しているわけではないということである.円城塔は,多世界解釈は各世界を超越した全世界を総覧できる超越的な観測者を置かなければならないという点で奇妙な理論であるとしている.

                多世界解釈は,適当なことを言っても(物理学的にも,かつSF的にも)許されることが多い.なぜなら,多世界解釈自体が適当で粗雑な理論であるから\footnote{この言は誹謗中傷に片足を突っ込んでいると思われるかもしれないが,もはやここまで言い切ってしまってもいいほど,多世界解釈は物理学的に否定されなければならないと思う.標準的な見解では,コペンハーゲン解釈以外を採用する合理的な理由は物理学的に存在せず\cite{htt, smz},また哲学においても,コペンハーゲン解釈と多世界解釈のどちらが正しいのかという議論はあまり有益であるとはみなされていない\cite{mrt}.}.あの時選ばなかった選択肢,ありえたかもしれない私,というモチーフを多世界解釈は強力に許す.このような後悔は,歳をとればとるほど増える一方であろう.これを自然に導入するのに,多世界解釈は極めて便利である.多世界解釈は本質的に(悪い意味で)SFであるから,導入した瞬間にその作品をSFジャンルに無条件で組み入れることが可能なのである.

            \subsection{p.44, 10-13}

                \begin{quote}

                    「何かが見えるということは,今見えているものを見るものがいるからだとする立場がある.〜」

                \end{quote}

                この表現は,哲学者ダニエル・デネットが心身二元論(特に意識のホムンクルス・モデル)を批判する際に用いた比喩,デカルト劇場(Cartesian theater, カルテジアン劇場とも)に由来している.心身二元論者は,人間の脳が外部信号を感得する機構を説明する際,脳内には意識を司る小人がおり,その小人が脳内に入ってきた外部信号を感得するというモデルをしばしば用いる.デネットは,このモデル(意識のホムンクルス・モデル)は無限後退であるとして批判した\cite{dnt}.

                このように,デカルト劇場は観測者の内部モデルを考察する際の哲学・認知科学における議論である.実は同様の議論が物理学においても行われ,こちらも適切な定式化が試みられている.

                量子力学において,観測行為は相互作用として記述される.これを量子力学的観測といい,相互作用によって初めて状態\footnote{この「状態」という語は日常語ではなく,量子力学における厳密な定義語.}が確定され,相互作用の前後で観測対象と観測者の内部状態は不可逆に破壊される.

                このような観測が量子力学的な階層で完結していればさほど問題は起こらないのだが,量子力学的な観測対象を古典力学的な観測者,すなわちわれわれ人間が観測する場合は,量子力学的効果の及ぶ範囲はどこまでかということを考慮しなければならない.これが特に顕著なのは,観測対象と観測者の間に中間の観測者が存在する場合である.以下,具体的な状況を考えてみる.

                いま,観測者ウィグナーはシュレーディンガーの猫の結果を確認したい.しかし,ウィグナーは忙しくて研究室を離れられないので,友人であるフォン・ノイマンが代わりに猫の生死を確認し,その情報をウィグナーに伝達する.シュレーディンガーの猫は量子力学的な観測対象であり,フォン・ノイマンとウィグナーは古典力学的な観測者であると考えられる.しかしながら,ウィグナーの立場では,猫の生死はフォン・ノイマンから聞かされるまで確定されない.したがって,ウィグナーにとっては,フォン・ノイマンは量子力学的に振る舞うように思われる.シュレーディンガーの猫,フォン・ノイマン,ウィグナーから成る系は,どこまでが量子力学的で,どこからが古典力学的なのだろうか?

                このような思考実験を,その提唱者の名をとってウィグナーの友人という.ウィグナーの友人への回答は,量子力学の解釈とともに様々提案されてきた.ここで紹介したいのは,ハイゼンベルク切断(Heisenberg cut)という概念である.これは系の量子力学的領域と古典力学的領域を区別する境界であり,しかも観測対象と観測者の間の任意の箇所に設定しても等しい分析結果を与えることが知られている.先ほどの事例で言えば,量子力学的なのはシュレーディンガーの猫だけであるとしても,フォン・ノイマンまでであるとしても,ウィグナーが得られる情報は変化しない.

                ハイゼンベルク切断は量子力学における概念であったが,内在物理学においても同様にデカルト切断(Cartesian cut, カルテジアン切断とも)という概念で,観測対象と観測者との区別,特に観測者の境界を定めている\cite{cut}.

                これらハイゼンベルク切断・デカルト切断と作品との対応関係を考えよう.ハイゼンベルク切断およびウィグナーの友人については,ハイゼンベルク切断の有する途中過程の縮約可能性およびウィグナーの友人における途中過程の無限連鎖と無限後退は,作中におけるレフラー球の縮約可能性と連鎖をよく説明する.また,デカルト切断による自己の確立を示唆する箇所が作中に存在することから,本作においてデカルト切断という概念を考慮するべきであると考えられる\footnote{詳細は\ref{russel}で検討する.}.

                %科学哲学かウィグナーの友人のいずれか,あるいはいずれにおいても

            \subsection{p.45, l.7-}

                \begin{quote}

                    「十八世紀,分子論の立役者の一人であるジョン・ドルトンは,〜」

                \end{quote}

                ドルトンは自身の色覚に異常があることに気づいており,自らの色覚異常を告白し,その異常性や原因について考察した論文\cite{dol}を出版している.同論文において,ドルトンは,自身は赤と緑の区別がつかないことを報告しており,ここからドルトンは赤緑色覚異常と呼ばれる症状を有していたことが推察される.また,この原因について,ドルトンは,自身の自身の眼球は青色透明の体液に満たされており,赤と緑の波長の光が体液に吸収されてしまうために区別がつかないのではないかとの仮説を提示している.本作において,観測者と観測対象との間に挟まるレフラー球が青色透明であるのは,このドルトンの仮説に由来していると考えられる.

                ところで,自身の観測が偏向していることに気づくことは非常に難しい.ドルトンは,自身の色覚が多数派ではないと気づき,綿密な観察によってどの色の区別がつかないのかを特定し,(結果として間違ってはいたのだが)仮説の構築にまで辿り着いている,これはドルトンの科学者としての卓越した能力を示している.

                このような無自覚の観測バイアスはあらゆる科学が立ち向かわなければいけないものだが,物理学はその観測理論の特性を極めてよく把握していることが知られている.例えば,もし宇宙自体が曲がっていたとしても,その曲がった宇宙の内部から宇宙が曲がっているという事実は観測によって導出可能である.

                物理学が極めてよく成功しているのは,物理学が観測理論としての量子力学を有していることと,物理法則がなぜか厳密に数学に従うという経験的事実\footnote{物理学者ユージン・ウィグナーは,物理学をはじめとした自然科学において,数学が理不尽なまでの有益であることについて「自然科学における数学の理不尽なまでの有益性」という論文\cite{wig}で考察している,}に立脚しているからのように思われる.例えば,先述の曲がった宇宙の内部からの観測は,ガウスの驚異の定理から導出される結論\footnote{この事実と量子力学,生物物理学を混ぜ合わせたものが「内在天文学」.}である.

            \subsection{p.46, l.11-14}

                \begin{quote}

                    「そうは言ってもわたしたちは違う人間のなのだし,今こうして実感しているものを取り出して並べてみることはできない.〜」

                \end{quote}

                もちろんクオリアの話だが,文脈としてはカントの議論を引いたものであろう.

            \subsection{p.47, l.5-6}

                \begin{quote}

                    「悪しき科学主義」

                \end{quote}

                科学主義とは,科学的方法こそが知識を獲得するために最良な唯一の手段であり,科学的方法で到達不能な知識はそもそもいかなる方法によっても到達不能であるとする立場.偏見だが,ほとんどの物理学者は素朴な科学主義を信奉しているように思う.

                一方で,哲学者・論理学者・数学者バートランド・ラッセルは,哲学者では珍しく科学主義の立場を明らかにしている\footnote{そうでなければ『プリンキピア・マテマティカ』なんて取り組むわけがないだろう.}.

            \subsection{p.47, l.15-p.48, l.5}

                \begin{quote}

                    「独我論と呼ばれる〜」

                \end{quote}

                自分以外の自由意志を認めない,強固な立場の経験論を独我論という.また,直前の客観論と独我論の対立は,自然の階層性に似ている.例えば,古典力学と相対論は相容れないが,その相対論的極限/古典極限で両者は一致する.

                この例からのアナロジを考えれば,レフラー球による観測はレフラーの定理の否定的解決\footnote{以後,これを作中の言葉を用いてロンドンの構造不一致定理という.}より,いかなる経路によってもいずれ任意の地点を経由するので.レフラー球をどのように解釈するか(客観論的か独我論的か)によって到達可能な範囲は変わらないと言うことを指している,というようになるだろうか.

                また,観測対象が同一であれば用いる立場が何であっても結論は一致するであろうという主張は,再帰定理と合わさることによって,レフラー特有の帰納と推論がすぐさま混交する悪癖をよく説明する.

            \subsection{p.49, l.3-4}

                \begin{quote}

                    「つまるところ自意識とは,自意識を担う一群のニューロンの発火パターンであり,その一群から手を伸ばされているニューロンの〜」

                \end{quote}

                “手を伸ばされている”という表現は「パリンプセストあるいは重ね書きされた八つの物語」にも登場している.

            \subsection{p.49, l.8-9} \label{russel}

                \begin{quote}

                    「全てを見張るには全てを見張らねばならず,そうすると全てを見張るものを見張る見張りが必要ということになって,集合論的にも不都合が発生する.」

                \end{quote}

                正しい.この(素朴)集合論における不都合のことを,ラッセルの逆理という.

                \begin{clm}

					ラッセルの逆理

					自分自身を元としてもたない集合全体から成る集合$X$は$X = \{ x \ |\  x \notin x\} $で表される.

					いま,$X \in X$と仮定すると,$X$の構成方法より$X \notin X$が導かれるが,矛盾.

					一方,$X \notin X$と仮定すると,$X$の構成方法より$X \in X$が導かれるが,これもまた矛盾.

				\end{clm}

                ラッセルの逆理の詳細な説明と,公理的集合論における解決については\cite{rus}を参照されたい.

                また,観測者の観測者の観測者の$\cdots$という連鎖の果てに,自分自身を迂遠に観測する系が誕生することが容易に予想される.これが循環レフラー群であり,単調に繋がった円環状の矢印であり,オットー・レスラーの内在物理学における観測である.円城塔の小説の根底には,このような観測行為と,このような観測行為が本質的に逃れられない自己言及がある\footnote{そもそも,自己言及(=恒等変換)を禁止するようなルールは相当窮屈である(少なくとも圏が使えない)のだから,自己言及を許すルールを採用しようというのは至って普通の話であるように思う.}.

                ここで,物理学における観測を考えてみると,観測理論としての量子力学は,ハイゼンベルク切断によって観測者と観測対象を独立したものとして扱っているように思われる.本文の直後の箇所にある「まあこのあたりまでが自分の手に負える自分である」というのは,このハイゼンベルク切断によって観測者である自分が確立されることを示しているように思われる.

            \subsection{p.49, l.13-}

                \begin{quote}

                    「しかしでは何故,一群の自意識ニューロンが,自意識という機能を持っているのかという問いにこの整理は答えていない.〜」

                \end{quote}

                その通り.しかし,これが物理学(複雑系)が答えられる限界であろう.理学全体で考えても,自意識という機能を持っていることが過去のある時点において生存に有利であったという回答にしかならないだろう.

            \subsection{p.50, l.7-8}

                \begin{quote}

                    「そのへんに落ちている平凡なニューロンを適当に自意識様に配置するだけで意識が生まれるとは思えない.」

                \end{quote}

                フランシーヌはウルフラムの提唱の信奉者ではないことが明示されている.

            \subsection{p.50, 14-}

                \begin{quote}

                    「素粒子の究極理論が,〜」

                \end{quote}

                残念ながらその通り.古代ギリシアの昔から,物質の最小単位であると考えられていたものが更に小さな要素から成る内部構造を持つ粒子であると判明してきた例は多々ある.いずれの時代も,最小単位の構成物質の種類が多すぎるということで内部構造が疑われ,実際に内部構造が発見されるという過程を経ている.例えば,(前半は省略するとして)分子から原子,原子から電子・核子,そして標準模型へと至っている.

                現状,標準模型では粒子が足りないので,少なくとももうひとつは対称性(超対称性)があると考えられているが,その超対称性としてどのような群をおくかについては結論が出ていない.これは現状の実験では得られないパラメータ領域で初めて決着がつく話なので,新しい加速器が作られないことには手の出しようがないという状況になっている,一方で,標準模型において素粒子とされているクオークやレプトンに内部構造があるという兆候は見られていない.

            \subsection{p.51, l.3-4}

                \begin{quote}

                    「理論なるものは,理論の中の構成によってその理論の限界を知る構成をとって初めて理論たりうるというのがレフラーの気分である.」

                \end{quote}

                内在物理学の話でもあるし,広く科学理論一般は反証性と検証性を持つべきであり,またその理論を用いる科学者は自らが立脚する理論の限界を把握しておくべきだとの主張でもある.

                理論はその適用限界が存在することによってはじめて科学理論となる,という主張は,多くの物理学者が抱いているものであり,またフロイトの精神分析やマルクスの歴史理論を疑似科学であると批判したポパーの議論を踏まえたものでもあるだろう.

                %書き足す

            \subsection{p.51, l.4}

                \begin{quote}

                    「神は全体的構成に宿る.」

                \end{quote}

                無論,“神は細部に宿る”という言葉の裏返しだが,これこそが統計力学の最大の利点である.細部を知ることなく,マクロに全体を知ることによって,その全体の振舞いを統計的に知ることが出来る.全体が統計力学的な法則に束縛されているが故に,われわれは統計力学的に予言をすることが出来る.いい感じに構成すればいい感じに振舞う,というアバウトさが統計力学の魅力である.

            \subsection{p.51, l.16}

                \begin{quote}

                    「超高次元力学系の挙動」

                \end{quote}

                統計力学で記述した系の振舞いのこと.統計力学では,$N$個の粒子からなる3次元系を,1個の粒子からなる$3N$次元系の問題に還元して考える.統計力学においては,$N$は1より十分大きく,$3N$次元は超高次元と言える.

                無論,統計力学的な系を直観的に把握できる人間など存在しない.逆に,直感的に把握出来る人間がいたならば,という話が「ムーンシャイン」である.

            \subsection{p.52, l.12}

                \begin{quote}

                    「レフラーの専門が,定理自動証明と呼ばれる数学分野〜」

                \end{quote}

                数学と呼ばれる体系は,極めて厳密な体系であり,実のところ,証明の自動化はある程度可能である.

                証明プロセスが停止したならば,その命題が証明されたことになる.恐ろしいことに,世の中には,正しいのだが,証明プロセスが停止しない命題が知られている.これが停止性問題である.停止性問題は,ゲーデルの第一不完全性定理と等価であることが知られており,先述の通り,コルモゴロフ複雑性とも深遠な関係にある.

                なお,定理自動証明と形式証明は似ているが異なる分野である.定理自動証明は機械に自動的に証明させようという分野であり,ゴールを定めず行けるところにはすべて行ってみるとった感じで大量に排出されるクズ証明からいいものを拾う,といったものである.一方で,形式証明は,人間が機械支援を受けて丁寧に証明を進めようという分野であり,目指すゴールは決まっていて,それを丁寧に埋めていく.

            \subsection{p.53, l.4}

                \begin{quote}

                    「公理と推論規則が与えられて,〜」

                \end{quote}

                「考速」にも同様の文言が存在する.

            \subsection{p.53, l.14-15}

                \begin{quote}

                    「トルネドとある種の形式系に同型写像が成り立つ」

                \end{quote}

                これは数学的に非常に強力かつ興味深い(のだが,研究が間に合わなかった).

                %要検討

            \subsection{p.58, l.8}

                \begin{quote}

                    「認識と真理の共進化〜」

                \end{quote}

                共進化という語自体は物理学でも使う\footnote{時空と物質の共進化,という言い回し\cite{mtb}など.}が,少なくとも真理は進化しないのではないか.逆に,真理が進化しうるという主張が本作のSF的な嘘である

            \subsection{p.59, l.4-5}

                \begin{quote}

                    「その成果は論文集の中には登場しておらず,自伝に埋もれてあまり目立たぬ一挿話ほどのものに留まっている.」

                \end{quote}

                数理科学のあるあるというか,私の知る限りではディラックとファインマンにのみ適応されるあるあるのような気がする.この両者については,論文を読むよりも,教科書\cite{dir, fynp}や自伝\cite{fyna}を読むと,天才の天才たる所以であるような直感的な洞察が唐突に書かれていて驚くことが多い.

                あるいは,教授の退官記念の最終講義で,確かに面白いし本質的なのだが,物理学としては到底扱えないようなSF的なアイデアや宗教的信念\footnote{本当の宗教ではなく,科学では現状探索不能だが,もしそうだとしたら美しい理論になるであろう信念のこと.物理学帝国主義,サイバーパンクへの憧憬,ウルフラムの提唱のほか,後述のアティヤの夢や,ディラックの大数仮説がこれの実例.}がうかがえてものすごく面白かったりする.

            \subsection{p.59, l.6}

                \begin{quote}

                    「たとえるならその儀式は,〜」

                \end{quote}

                「Gernsback Intersection」か?

            \subsection{p.60, l.2-3}

                \begin{quote}

                    「もしくは想像力と大雑把に呼ばれる力が行う仕事に,発熱分を加えたエネルギー.」

                \end{quote}

                ランダウアーの原理のことを指す.ランダウアーの原理の主張は以下の通り.

                \begin{prn}

                    ランダウアーの原理

                    不可逆な計算は必ず熱の発生を伴う.

                \end{prn}

            \subsection{p.60, l.7}

                \begin{quote}

                    「裡へ閉じ込められて,決して外へと貫き出ることのないレフラーの視線.」

                \end{quote}

                円城塔作品には“自分自身が自分自身を規律して孤立して宙に浮かぶ系”というモチーフが多用されていることを説明した.本作におけるそれは循環レフラー群だったわけだが,ここで,レフラー自身もそのモチーフそのものであったことが明かされる.

            \subsection{p.60, l.9}

                \begin{quote}

                    「モラン変換」

                \end{quote}

                実在する数学的操作.盲目の数学者,ベルナール・モランによって構成された,三次元球殻をなめらかに裏返しにする操作のこと.この操作の途中で,球殻はボーイ曲面(Boy's surface)を経由する.

                本作において,このモラン変換は重要な要素となっているように思われる.作中の描写に従いつつ,モラン変換の文芸的解釈を試みる.

                まず注意するべきは,モラン変換が球殻からボーイ曲面を経由して裏返しの球殻へと至る変換であるということである.直後にある「頭の中にあるものをそっくり反転させて提示すること.」が可能であるのは,このモラン変換によるものである.すなわち,モラン変換を物語中で可能な操作として認めるならば,レフラー球という球殻は,レフラーという少年の表面を経由して,壊すことなく裏返しにして中身を露出させることが可能\footnote{物体の表裏を入れ替えるという概念は「良い夜を持っている」にも共通する.}である.

                一旦モラン変換をおき,本作におけるレフラーの描かれ方を考察する.本作において,レフラーは,基本的に伝聞形式で記述されている.伝聞形式ではレフラーの内面については当然知り得ず,レフラーの内部に蠢く論理や内部機構は作中を通じて真に不明\footnote{中身を覆い隠してパッケージ化する取り扱いからは,プログラミングにおける糖衣構文やカプセル化(情報隠蔽とも)が連想される.情報隠蔽と内在物理学の双方において,インターフェイスという語が共通して登場するが,これは独立して構築された同じような概念に同じ名前が当然付与されたことによるものである.この関連に特に深い意味はなく,何かしらの関連性を見出すのは妄想的であり,不適.}であり,その表層がどのように振舞うかという観測を積み重ねることでしかレフラーを理解出来ない.つまり,われわれが感得出来るのは,レフラーという少年の表面のみである.

                モラン変換に戻る.モラン変換の途中で経由するボーイ曲面は,2次元へと切開すると,メビウスの輪となる.本作はメビウスの輪を1周分だけなぞるように構成されており\footnote{「What is the Name of This Rose?」の自作改題より.},本作が“少年の表面”の紙面への自然な展開であるとする見立ては,よく説明される.

                最後に,検討すべきではあるが,時間が足りず精査出来なかった手がかりについて記す.出発点と終着点が定まっていて,その間に複数の(数学的)経路があるような場合の数学としてモデル圏が挙げられる.また,レフラー球が分岐して合流するという描写から,レフラー球の連鎖を$\lambda$計算,特にそのチャーチ--ロッサー性(合流性)によってよく記述出来る可能性が高い.

                %もっと書く

            \subsection{p.62, l.7}

                \begin{quote}

                    「重々帝網なるを即身と名づく」

                \end{quote}

                空海『即身成仏義』の一節.重々帝網とは,帝釈天堂に縦横無尽に張り巡らされた糸(帝釈網)の交点には珠玉があり,それぞれの珠玉は互いの姿を幾重にも反射し,共に照らし出している様子をいう.

                空海は,六大\footnote{宇宙の根源的な力である地,水,火,風,空,識のこと.}・四曼\footnote{宇宙の顕現形態である大,三,法,羯のこと.}・三密\footnote{身体活動のすべてである身,口,心のこと.}が網の目のように混じり合いつつも,互いが互いを尊重し合い,大きなひとつまとまりとなって成仏を目指して進んでいるということを表している\cite{kno}.

                円城塔作品には,先掲の通り,目の前に投げ出された酷く絡み合った網目という表現が頻出する.これを私は物理学的にしか捉えていなかったが,これは円城塔の仏教からの影響をも反映したものであったらしい.

            \subsection{p.65, l.3}

                \begin{quote}

                    「数学的構造だからといって自分が数学を理解しているというつもりはない.」

                \end{quote}

                正しい.形式証明で用いるThe Rocq Prover\footnote{かつてCoqと呼ばれていたものと同一.2024年に名称が変更された.}は,数学を知らないだろう.チューリング機械で用いるオートマトン自体が数学を知っているはずがないのと同様である.

            \subsection{p.65, l.16}

                \begin{quote}

                    「疑う暇があるなら研究せよ」

                \end{quote}

                その通り.

            \subsection{p.66, l.14}

                \begin{quote}

                    「リーマン予想に手を出すとはあいつも焼きが回ったらしい〜」

                \end{quote}

                リーマン予想は最も有名な数学の未解決問題であり,その主張が実に簡潔なことでも知られている.リーマン予想の主張は以下の通り\cite{sei}.

            \newpage

                \begin{hyp}

                    リーマン予想

                    リーマンゼータ関数の非自明な零点の実部はすべて$\displaystyle \frac{1}{2}$である.

                \end{hyp}

                このリーマン予想には,史上数多の数学者が挑み,そして敗れ去っていった.現代においては,その難易度の高さから,リーマン予想を真正面から真面目に証明しようということ自体が不毛なこととされており,数学上の最重要問題である\footnote{難易度が高いから最重要というわけではない.他の研究を進めるうちに,思いがけずリーマン予想に繋がってしまう事例が数多く報告された結果,リーマン予想は数学的に最も興味のある問題であるとみなされている.}にもかかわらず,それ以上に難しすぎるという評価がなされている.

                逆に,リーマン予想に挑もうという者は,自分のキャリアをかなぐり捨て発狂した数学者か,数学的能力を失って自省が効かなくなって発狂した数学者か,自身の数学的能力が足りていないということに気づかない無能かのいずれかという状況になっている.

                そのような状況下で,2018年末に,フィールズ賞を受賞した超大物の数学者・数理物理学者であるマイケル・アティヤが,物理学における微細構造定数は数学的に決定可能であり,この計算過程でリーマン予想を肯定的に証明したと主張する論文\cite{aty1, aty2}を発表し,数学者・物理学者に一時大きな衝撃を与えた.

                結論,アティヤによる論証とされるものは,そもそも大前提が派手に間違っており論外であるとして,アティヤは耄碌しきったという容赦ない評価が下され,偉大な数学者としてのアティヤは失われたとされた.当時,アティヤは奇数位数定理(ファイト--トンプソンの定理)を極めて簡潔に証明したと主張しつつ,あからさまな手違いが見受けられるなど,直近の数年の内に初歩的なミスが多く指摘されており,既に数学者としての能力を喪失しつつあると評価されていた.このリーマン予想事件でアティヤは完全に名誉と信頼を喪失し,翌2019年1月に没した\footnote{件の論文は英王立協会誌に投稿されていたものの,著者死没によって無事撤回されることとなった.}.

                最晩年のアティヤの数学的論証はおくとして,アティヤによる物理学的執着,すなわち微細構造定数が数学的に決定可能であり,基本相互作用に関わる定数はすべて数学的に決定可能であって,それらはすべて数学的構造から自然に導かれるものであるという主張は大変に興味深い\footnote{とはいえ,このような主張はかなり無理筋であるように思われる.アティヤの原著論文\cite{aty1}によれば,電弱相互作用は実数$\mathbb{R}$と複素数$\mathbb{C}$に,強い相互作用は四元数$\mathbb{H}$に.重力は八元数$\mathbb{O}$に由来する.そもそもこの対応関係が妄想的であって,十分な検証がなされていない,また,標準模型には少なくとももうひとつ追加の対称性が必要であるから,最低でも1つは基本相互作用が追加される.しかし,八元数およびそれよりも大きい十六元数$\mathbb{S}$との対応を考えると,八元数・十六元数のいずれも非可換かつ結合的でないなど,数学的な振る舞いが良くないことから,そのような対応関係が成立するというのは,やはり妄想的であるように思われる.}.アティヤは,物理学すら数学に還元可能であるという世界観\footnote{物理学者,特に20世紀半ば頃の物理学者\cite{hsy}や,あるいは現代の素粒子物理学者の多くには,すべての学問は物理学に還元可能であるとの驕りが存在する.これを物理学帝国主義,あるいは還元主義という.これは科学ではなく信仰の一種であり,実際,複雑系は物理学帝国主義が素朴には成り立たないことを示している.}を夢見ていた.このような,基本相互作用は数学的に決定可能であるというアティヤの主張を,数学帝国主義,あるいはアティヤの夢ということにする.

                アティヤによる失敗もあったが,かつて至難の未解決問題とされたフェルマーの最終予想\footnote{1995年,アンドリュー・ワイルズによって.現在ではフェルマー--ワイルズの定理と呼ばれる.}やポアンカレ予想\footnote{2006年,グレゴリー・ペレルマンによって.現在ではポアンカレ--ペレルマンの定理と呼ばれる.}が無事正常に証明されたように,リーマン予想に挑戦する人は絶えず現れるだろう.

                また,リーマン予想は決して前人未到ではなく,いくつかの重要な手がかりが残されている.そのうちで最も興味深いのは,理論物理学から逆輸入された知見だろう.すなわち,一般の物質が熱平衡するかどうかは決定不能であり,これが停止性問題と等価であり,しかもリーマン予想が偽のとき,かつそのときに限り,熱平衡化する物体が存在するということが知られている\cite{shi}.

            \subsection{p.67, l.4-}

                \begin{quote}

                    「どこかの余白にあなたのお好みの数式を〜」

                \end{quote}

                フェルマーがフェルマーの最終予想について書いた有名な文言,「わたしは驚くべき事実を発見したが,それを書くにはこの余白は狭すぎる」というものを意識しているか.人には誰だって好みの数式があるものだ\footnote{残念ながら,偽である.}.

            \subsection{p.67, l.13-14}

                \begin{quote}

                    「奴らは基盤なしに引きこもってぐるぐる回っているだけの自閉的妄想だが,それだけに摂動にはえらく強い.」

                \end{quote}

                循環レフラー群が無限連鎖する自己規律によって孤立して宙に浮く系であるという指摘を補強するものである.

                “自閉的妄想”というのはかなり攻撃的な言葉遣いに思われるが,“自閉的”あるいは“自閉症的”という語自体は,80年代から90年代にかけてのSF作品でよく使われていた言い回しでもある\footnote{例えば,士郎正宗『攻殻機動隊』\cite{srms}など.『攻殻機動隊』では,外部との通信を切断して,引きこもったりセキュリティを高めたりすることを指して“自閉症モード”と言っていた.なお,現在では“閉鎖モード”に表現が改められている.}.

                %摂動の説明

            \subsection{p.68, l.6}

                \begin{quote}

                    「レフラーの野望」

                \end{quote}

                アティヤの夢,ディラックの大数仮説\footnote{ディラックによる,物理定数は時間発展するという仮説.いくつかの物理定数同士の比が$10^{40}$あるいは$10^{80}$という特徴的なオーダーであるという発見に由来する.}の同類のようなもの.

                なお,1と$\displaystyle \frac{1}{2} + \frac{1}{4} + \frac{1}{8} + \cdots$は一致する.先述した,無限という概念を扱うのは非常に難しいということの実例でもある.

            \subsection{p.68, l.12}

                \begin{quote}

                    「誰かと,誰かを想像することを一致させる極限.」

                \end{quote}

                この発想はのちの「決定論的自由意志利用改変攻撃について」\footnote{アンソロジー『異常論文』(ハヤカワ文庫JA)収録.}に繋がる.これは同作でいうところの“分割型解決”に近く,圏論における自然変換を用いれば,文芸的に興味深い考察が得られそうな気配がする(時間が足りなかった).

	\begin{thebibliography}{99}

		\bibitem{boys} 円城塔, 『Boy's Surface』, ハヤカワ文庫JA, 早川書房, 2011

        \bibitem{ochi} 落合仁司, 『数理神学を学ぶ人のために』, 世界思想社, 2009

        \bibitem{eth} スピノザ, 『エチカ 上』, 岩波文庫, 岩波書店, 2011

        \bibitem{pa} 下村思游, コンメンタール「パリンプセストあるいは重ね書きされた八つの物語」, \textit{円城塔研究}, \textbf{2}(1), 1-20, 2024

        \bibitem{cat} レンスター, 『ベーシック圏論』, 丸善出版, 2017

        \bibitem{ssk} 佐々木敦, 『あなたは今,この文章を読んでいる. : パラフィクションの誕生』, 慶應義塾大学出版会, 2014

        \bibitem{sri} 円城塔, 『Self-Referential INTERVIEW』, 2007, \url{https://engine.sub.jp/ppt/SRI.ver2.3.pdf}

        \bibitem{myz} 宮沢賢治, 『宮沢賢治全集1』, ちくま文庫, 筑摩書房, 1986

        \bibitem{yskt} 安酸敏眞, 第七回シンポジウムのバックストーリー : 深井智朗氏の研究不正事件とそこに含まれる人文学/人文科学の重要問題, 北海学園大学人文論集, (69), 2-11, 2020, \url{https://human.hgu.jp/publication/docs/journal-humanities69.pdf}

        \bibitem{ntz} 芦名定道, 小柳敦史, 深井智朗, 会員から会員へ : 質問と応答, \textit{日本の神学}, (57), 224-232, 2018, \url{https://doi.org/10.5873/nihonnoshingaku.57.224}

        \bibitem{mtz} 松坂和夫, 『集合・位相入門』, 岩波書店, 1968

        \bibitem{tki} 東京大学教養学部統計学教室, 『統計学入門』, 東京大学出版会, 1991

        \bibitem{llm} 下村思游, 円城塔のローラ : Apple silicon専用機械学習フレームワークを用いた円城塔LLMの開発と運用, \textit{円城塔研究}, \textbf{2}(3), 1-15, 2024

        \bibitem{ott} Ichiro Tsuda, Takashi Ikegami, Endophysics : the world as a an interface, \textit{Discrete Dynamics in Nature and Society}, \textbf{7}(3), 213-214, 2002, \url{https://doi.org/10.1080/1026022021000001481}

        \bibitem{cog} Andrew Shtulman, Joshua Valcarcel, Scientific knowledge suppresses but does not supplant earlier intuitions, \textit{Cognition}, \textbf{124}(2), 209-215, 2012, \url{https://doi.org/10.1016/j.cognition.2012.04.005}

        \bibitem{end} Otto E. Rössler, \textit{Endophysics}, World Scientific, 1998

        \bibitem{goto} 円城塔, 『後藤さんのこと』, ハヤカワ文庫JA, 早川書房, 2012

        \bibitem{moon} 下村思游, コンメンタール「ムーンシャイン」, \textit{円城塔研究}, \textbf{2}(1), 21-63, 2024

        \bibitem{mct} 菊池誠, 差異こそはすべて, \textit{SFマガジン}, \textbf{50}(1), 26-30, 2009

        \bibitem{pla} C・L・ジーゲル, J・K・モーザー, 『天体力学講義』, 丸善出版, 2024

        \bibitem{sip} Michael Sipser, 『計算理論の基礎2』, 第3版, 共立出版, 2023

        \bibitem{omr} 大森荘蔵, 『時間と自我』, 新装版, 青土社, 2023

        \bibitem{tsk} 田崎晴明, 大森荘蔵の時間論のごく一部, 2011, \url{https://www.gakushuin.ac.jp/~881791/modphys/11/Ohmori.pdf}

        \bibitem{mtn} 原惟介, 松永秀章, イプシロン・デルタ論法完全攻略, 共立出版, 2011

        \bibitem{hsi} 細井勉, はじめて学ぶイプシロン・デルタ, 日本評論社, 2010

        \bibitem{sum} Raymond S. Smullyan, \textit{Logical Labyriths}, A K Peters, 2009

        \bibitem{kwb} レイモンド・スマリヤン, 『スマリヤン記号論理学 : 一般化と記号化』, 丸善出版, 2013

        \bibitem{htt} 堀田昌寛, 『入門 現代の量子力学 : 量子情報・量子測定を中心として』, 講談社, 2021

        \bibitem{smz} 清水明, 『量子論の基礎』, 新版, サイエンス社, 2004

        \bibitem{mrt} 森田紘平, 量子力学を解釈するとはどういうことだったのか, \textit{日本物理学会誌}, \textbf{76}(8), 532-534, 2021, \url{https://doi.org/10.11316/butsuri.76.8_532}

        \bibitem{dnt} ダニエル・デネット, 『解明される意識』, 青土社, 1998

        \bibitem{cut} Harald Atmanspacher, Cartesian cut, Heisenberg cut, and the concept of complexity, \textit{World Future}, (49), 333-355, 1997, \url{http://dx.doi.org/10.1080/02604027.1997.9972639}


        \bibitem{dol} John Dalton, Extraordinary facts relating to the vision of colors : with observations, \textit{Manchester memoirs}, (5), 28-45, 1798

        \bibitem{dlt} 『科学の名著(第2期)6』, 朝日出版社, 1988

        \bibitem{bby} 馬場靖人, 〈色盲〉の歴史研究から当事者研究へ : 色盲者の言葉を取り戻すために, \textit{立命館言語文化研究}, \textbf{32}(3), 63-76, 2021, \url{https://doi.org/10.34382/00014147}

        \bibitem{wig} Eugene Wigner, The unreasonable effectiveness of mathematics in the natural sciences, \textit{Communications in Pure and Applied Mathematics}, \textbf{13}(1), 1-14, 1960, \url{https://doi.org/10.1002/cpa.3160130102}

        \bibitem{rus} 下村思游, “集合の集合”が集合でないことの証明, \textit{円城塔研究}, \textbf{1}(1), 11-14, 2023

        \bibitem{oka} サミール・オカーシャ, 『科学哲学』, 新版, 岩波書店, 2023

        \bibitem{isd} 伊勢田哲治, 『疑似科学と科学の哲学』, 名古屋大学出版会, 2003

        \bibitem{rub} The craft of science fiction, Harper \& Row, 1976

        \bibitem{mtb} 松原隆彦, 『現代宇宙論』, 東京大学出版会, 2010

        \bibitem{dir} ディラック, 『量子力学』, 原書第4版改訂版, 岩波書店, 2017

        \bibitem{fynp} ファインマン, レイトン, サンズ, 『ファインマン物理学1-5』, 岩波書店, 1986

        \bibitem{fyna} リチャード・P・ファインマン, 『ご冗談でしょう,ファインマンさん 上下』, 岩波現代文庫, 岩波書店, 2000

        \bibitem{kno} 空海, 金岡秀友, 『即身成仏義』, 太陽出版, 1985

        \bibitem{sei} 小林銅蟲, 関真一朗, 『せいすうたん1』, 日本評論社, 2023

        \bibitem{aty1} Michael Atiyah, The Fine Structure Constant, 2018

        \bibitem{aty2} Michael Atiyah, The Riemann Hypothesis, 2018

        \bibitem{asd} 浅田明, Atiyahの夢, \textit{数学・物理通信}, \textbf{8}(9), 23-25, 2018, \url{https://www.phys.cs.i.nagoya-u.ac.jp/\%7etanimura/math-phys/mathphys-8-9.pdf}

        \bibitem{hsy} 細谷治夫, 「物理帝国主義」について, \textit{日本物理学会誌}, \textbf{51}(4), 265, 1996, \url{https://doi.org/10.11316/butsuri1946.51.4.265}

        \bibitem{isdb} 伊勢田哲治, 物理(学)帝国主義という言葉を使い始めたのはだれか, \textit{Daily Life}, 2023, \url{http://blog.livedoor.jp/iseda503/archives/1935568.html}

        \bibitem{shi} Naoto Shiraishi, Keiji Matsumoto, Undecidability in quantum thermalization, \textit{Nature Communications}, \textbf{12}(5084), 2021, \url{https://doi.org/10.1038/s41467-021-25053-0}

        \bibitem{srms} 士郎正宗, 『攻殻機動隊』, 講談社, 1991

    \end{thebibliography}

\end{document}